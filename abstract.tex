\begin{abstract}
A summary of survey strategy and cadence choices, simulated and evaluated by the Vera C. Rubin Observatory Legacy Survey of Space and Time (LSST) Scheduler Team, prepared for the Survey Cadence and Optimization Committee (SCOC). 

A large telescope survey, covering the entire visible sky repeatedly every few days in multiple bandpasses over the course of ten years, is the core idea of the LSST. An area of about 20,000 square degrees observed under a wide range of conditions to deep coadded limiting magnitudes in bandpasses $ugrizy$ enables cosmological studies and studies of the Milky Way structure with unprecedented precision; the same survey, when cadenced well, can serve to open new windows into our understanding of transient and variable stars, and extend our knowledge of small bodies throughout the Solar System by orders of magnitude. The outlines of, and some basic necessary requirements for these goals are outlined in the LSST Science Requirements Document (SRD)\footnote{ls.st/srd}. Finding options for the observing strategy to meet more detailed needs of an even wider range of science goals, as well as building the LSST Scheduler and Metrics Analysis Framework, has been the work of the LSST Scheduler Team with support and input from the astronomical community, including the Community Observing Strategy Evaluation Paper (COSEP)\footnote{https://github.com/LSSTScienceCollaborations/ObservingStrategy}, the Call for White Papers\footnote{Document-28382}, numerous performance metrics, and guidance from the LSST Science Advisory Committee in their Recommendations for Operations Simulator Experiments\footnote{Document-32816}. This report will help enable the SCOC to make its recommendation for the preferred survey strategy.

\end{abstract}

