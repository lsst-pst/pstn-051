\section{Introduction}

Vera C. Rubin Observatory (Rubin) will carry out the Legacy Survey of Space and Time (LSST) over the first ten years of its lifetime. The LSST is intended to meet four core science goals:
\begin{itemize}
\item constraining dark energy and dark matter
\item taking an inventory of the Solar System
\item exploring the transient optical sky, and
\item mapping the Milky Way.
\end{itemize}

The basic requirements for these goals are described in the LSST Science Requirements Document (\href{http://ls.st/srd}{SRD}\footnote{The LSST Science 
Requirements Document (SRD) is available at \href{http://ls.st/srd}{http://ls.st/srd}}). In practice, the SRD intentionally places minimal quantitative constraints 
on the observing strategy, primarily requiring:
\begin{itemize} 
\item A footprint for the `main survey' of at least 18,000 deg$^2$, which must be uniformly covered to 
a median of 825 30-second visits per 9.6 deg$^2$ field, summed over all six filters, $ugrizy$ (see SRD 
Tables 22 and 23). This places a minimum constraint on the time required to complete 
the main survey. Simulated surveys indicate that the main survey typically requires 80--90\% 
of the available time (10 years) to reach this benchmark; even with scheduling improvements, it is unlikely 
that the goals of the main survey could be met with a time allocation significantly below 80\%. 
\item Parallax and proper motion $1\sigma$ accuracies of 3~mas and 1~mas/yr per coordinate at $r=24$, 
respectively, in the main survey (see SRD Table 26), which places
a weak constraint on how visits are distributed throughout the lifetime of the survey and throughout a season.
\item Rapid revisits (40 seconds to 30 minutes) must be acquired over an area of at least 2000 deg$^2$ (see SRD table 25) for
very fast transient discovery; this requirement can usually be satisfied via simple field overlaps when surveying contiguous areas of sky. 
\end{itemize}
This leaves significant flexibility in the detailed cadence of observations within
the main survey footprint, including the distribution of visits within a year (or between seasons), the distribution between filters and 
the definition of a `visit' itself. Furthermore, these constraints apply to the main survey; the use of the 
remaining time (i.e., in mini surveys) is not constrained by the SRD.

In order to maximize the overall science impact of the LSST, in 2018 the project issued a \href{ http://ls.st/doc-28382}{call for white papers} requesting survey strategy input\footnote{The call for white papers is available at \href{ http://ls.st/doc-28382}{http://ls.st/doc-28382}}. The 46 \href{https://www.lsst.org/submitted-whitepaper-2018}{submitted white papers} represent a wide swath of the astronomical community, and work together with the \href{https://github.com/LSSTScienceCollaborations/ObservingStrategy/raw/master/whitepaper/releases/LSST_Observing_Strategy_White_Paper_v1.0.pdf}{Community Observing Strategy Evaluation Paper (COSEP)}\footnote{The github repository containing the living source for the COSEP is \href{https://github.com/LSSTScienceCollaborations/ObservingStrategy}{https://github.com/LSSTScienceCollaborations/ObservingStrategy}} to shape the next stage of the survey strategy evaluation. The contents of these white papers were distilled into several areas for investigation by the LSST Science Advisory Council (SAC) in their advisory \href{http://ls.st/doc-32816}{response} to the project\footnote{The SAC white paper report is available at \href{http://ls.st/doc-32816}{http://ls.st/doc-32816}}.

This survey strategy optimization work is starting from an existing candidate baseline strategy, driven by the basic science goals. A brief introduction to the baseline survey strategy, expanded background of the primary LSST science goals, and concise descriptions of how these goals drive the basic survey strategy and data processing requirements are provided in the \href{http://ls.st/lop}{LSST Overview paper}\footnote{The LSST Overview paper is a living document available at \href{http://ls.st/lop}{http://ls.st/lop}.}. A reference survey simulation (baseline2018a), generated by an earlier version of the LSST survey simulation tools (see Section~\ref{section:simulator}), provided an implemented example of this strategy. This starting point for the survey strategy can be described extremely briefly as follows:
\begin{itemize}
\item The {\bf main ``wide-fast-deep''  (WFD) survey}, which covers $\sim$18,000 deg$^2$ of sky within
the equatorial declination
range $-62^\circ < \delta < +2^\circ$, and excluding the central portion of the Galactic 
plane. Within the main survey, two visits\footnote{A `visit' here is an LSST default visit, which 
consists of two back-to-back 15 sec exposures, for a total of 30 sec of on-sky exposure time. These back-to-back exposures are always
in the same filter, separated only by the 2 second readout time.} 
per 9.6 deg$^2$ field (in either the same or different filters) are acquired in each night, to allow identification of moving objects and rapidly varying transients, and to improve
the reliability of the alert stream. These pairs of visits are repeated every three to four nights throughout the period the field is visible in each year (other nights are used to maximize the sky coverage). Each
field in the main survey receives about 825 visits throughout the ten years of the LSST, spread over the six LSST filters 
$ugrizy$. The quantitative SRD constraints on area coverage, number of visits, parallax and proper motion errors, and 
rapid-revisit rate (40 seconds -- 30 minutes) apply to visits obtained in the main survey. 
\item The set of five {\bf Deep Drilling Field candidate mini surveys}, consisting of five specific field pointings for a total of $\sim$ 50 deg$^2$, 
which are observed with a much denser sampling rate. These mini surveys use a similar sequence of visits; the fields
are observed every three to four days, but in a sequence of multiple $grizy$ exposures during gray and bright time, and then
multiple sequential $u$ band exposures during dark time. The current deep drilling mini survey fields are aimed at extragalactic
science, providing a `gold sample' to calibrate the main survey, and to discover Type Ia supernovae. 
\item The {\bf Galactic Plane candidate mini survey} covers the central portion of the Galactic plane that is not included in the main survey, 
centered around $|l| = 0^\circ$ and covering $\sim$ 1860 deg$^2$.  It is observed at a much reduced rate compared to the main survey, 
and with a smaller total number of observations per field (30 visits per field and per filter, in $ugrizy$), so as to
provide astrometry and photometry of stars toward the Galactic center but without reaching the confusion limit in the coadded images.
There is no requirement for pairs of visits in each night in this area.
\item The {\bf North Ecliptic Spur candidate mini survey} covers the area north of $\delta = +2^\circ$ to $10^\circ$ north of the Ecliptic plane
and is intended to observe the entire Ecliptic plane for the purpose of inventorying the minor bodies in the Solar System. This area ($\sim$ 4160 deg$^2$) 
is observed on a schedule similar to the main survey, although with a smaller total number of visits per field and only in filters $griz$. 
\item The {\bf South Celestial Pole candidate mini survey} covers the region south of the main survey, to the South Celestial Pole, $\sim$ 2315 deg$^2$,
including the Magellanic Clouds. 
This mini survey is observed with a strategy similar to the Galactic Plane mini survey, with 30 visits per field per filter in $ugrizy$, 
and without requiring pairs of visits. This provides coverage of the Magellanic clouds, but without committing extensive time as these fields are
at high airmasses from the LSST site.
\end{itemize}

This report covers the LSST Survey Strategy team's experiments with the LSST scheduler to address the optimization questions raised by the SAC. These questions include:
\begin{itemize}
\item How should the WFD footprint be defined?
\item What should the cadence of visits within the WFD look like? This includes both the intra-night cadence and the inter-night cadence throughout the season.
\item What is the impact of varying the footprint for mini-surveys?
\item Can we leverage twilight observing?
\item How should the Deep Drilling fields be distributed and what cadence should be used for their observations?
\item What are the impact of ToO proposals, particularly gravitational wave followup?
\end{itemize}

These questions are aimed at ensuring the best possible science return from the LSST. 



\section{Survey Simulator Overview}
\label{section:simulator}

In operations, the LSST needs an automated scheduler to appropriately plan and execute the $\approx1000$ visits per night. Prior to operations, we have a need to use the same scheduler to understand the range of possible survey strategies and their science potential; even in operations it is useful to run the scheduler in a `simulation' mode in order to evaluate the future impact of changes in observatory hardware or changes to the observing strategy. As such, we need a robust scheduler, together with high-fidelity model inputs for the telescope operations and observing telemetry.  

Probably need some reference to what survey scheduler was used / how it was set up for various runs, how the runs were performed, and what the input weather and telescope models were like. 

\subsection{The Model Observatory}


\subsubsection{Telescope Model}

The physical telescope operations are modeled using the LSST software package \href{https://github.com/lsst-ts/ts_observatory_model}{ts\_observatory\_model}. This package includes a kinematic model of the telescope, with appropriate acceleration/deceleration and maximum velocity limits, including requirements for sequencing (changing the filter before slewing, for example). It also enforces requirements needed before image acquisition, such as the settle time after slewing and the active optics open- and closed-loop acquisition times. Other important considerations are the extent of cable wrap due to azimuth slews or camera rotation. The parameters for the telescope model are configurable, coming from the Telescope and Site and Camera teams. These parameters are largely unchanged from \citet{Delgado14}; a subset of these parameters are described in Table~\ref{tab:tsModel}. 

\begin{table}
\begin{centering}
\begin{tabular}{lc}
\toprule
Parameter  & Value \\
Min altitude  & 20 deg \\
Max altitude & 86.5 deg \\
Camera readout & 2 sec\\
Shutter time & 1 sec \\
Filter change time & 120 sec \\
Number filters mounted & 5 \\
Azimuth slew settle time & 1 sec \\
Closed Optics Loop Delay & 36 sec (when $>9$ deg altitude change) \\
Approximate azimuth slew time &   $ t_{slew \, Az} = 0.66 \, {\rm sec/deg} * \delta Az ({\rm deg}) + C^{Az} $ ;    min = 3 sec \\
Approximate altitude slew time  &  $  t_{slew \, Alt} = 0.57 \, {\rm sec/deg} * \delta Alt ({\rm deg}) + C^{Alt} $ \\
\hline
\end{tabular}
\caption{A subset of ts\_observatory\_model parameters and slew time approximations.}
\label{tab:tsModel}
\end{centering}
\end{table}

\subsubsection{Cloud Model}

The cloud model is based on historical cloud sky coverage data from Cerro-Tololo Inter-American Observatory (CTIO), from the ten year period 1996 to 2005. 

\begin{figure}
\epsscale{0.5}
\plotone{plots/cloud_levels}
\epsscale{1}
\caption{The distribution of cloudiness as measured at CTIO. We model the observatory as closed for cloud levels above 3/10.}
\end{figure}

\begin{figure}
\epsscale{0.5}
\plotone{plots/hours_pernight}
\epsscale{1}
\caption{The average amount of time available per night over the course of a year after removing weather downtime.}
\end{figure}

The SOAR telescope reports losing 15.3-33.4\% (mean=22.9\%) of science time to weather from 2014-2018\footnote{\url{http://www.ctio.noao.edu/soar/content/soar-observing-statistics}}. This is consistent with the weather downtime reported by Gemini South (private communication). 

If we model the observatory as closed when the sky is 30\% cloudy or cloudier, we reach a weather downtime of 29.8\%. While we expect some observations will be possible in 30\% cloudy skies, this cutoff also accounts for other weather related closures (humidity, wind, dust, etc).

\subsubsection{Seeing Model}

\section{Seeing}

Simulations completed starting in 2020 use a revised database for the
atmospheric seeing. The revised database, like its predecessor, is
based on seeing measurements from the Geminin South DIMM, located at
the same site as Rubin Observatory. We derived predicted delivered
imager FWHM from the reported DIMM measurements using the
approximation of the von K\'arm/'an turbulance model given in
Tokovinin (2002) and an outer scale of 30 meters, and validated this
relationship between DIMM measurements and seeing by comparing these
derived values to the image quality measured from the Gemini South
GMOS instrument. We also tested the DIMM data by deriving a seeing
using the Kolmogerov relationship and comparing the result to the
seeing measured by the DECam imager on the Blanco telescope at CTIO, a
few miles away.

For most time samples in the simulation database, we generated seeing
data by resampling seeing derived from the DIMM into 5 minute
intervals, and shifting it ahead 4748 days (13.000 tropical years). For
example, the seeing for 2022-01-01 in the simulation database is taken
from the DIMM seeing on 2009-01-01. Thus, most of the ten simulated
years use seeing values that replay ten historical years.

There is, however, significant time for which no DIMM data is available, for
example due to clouds or equipment failure. We used a model of
$log(r_{0})$ (where $r_{0}$ is the Fried parameter) derived
from the DIMM data to generate artifical seeing values for these
times. This model has several components:
\begin{itemize}
  \item a yearly sinusoidal variation in $log(r_{0})$ to include
    seasonal variation,
  \item a smooth (years timescale) fit to the residuals with respect
    to the seasonal variation to represent multi-year trends in
    seeing,
  \item a 1st-order autoregressive series (damped random walk) to
    represent variations in the nightly seeing, and
  \item another 1st-order AR series to represent variations on a
    5-minute timescale within a night.
\end{itemize}
Artificial data generated according to this model therefore maintains
the night to night and short term distributions and correlations
present in the DIMM data, and follows seasonal variations and longer
term trends in the DIMM data surrounding it. 



\subsubsection{Skybrightness Model}

The observatory model includes a model for the sky brightness. The model is built mostly from the ESO sky brightness model which includes upper and lower atmosphere emission lines, airglow continuum, scattered lunar light, and zodiacal light. In addition, we have added a twilight model fit from all-sky camera observations at the site. The sky brightness model does not include human generated light pollution. While the ESO model does include the ability to scale the airglow component with solar activity, we use the default mean solar activity throughout. Compared to all sky camera observations, the skybrightness model has RMS deviations of $\sim$0.2-0.3 magnitudes per square arcsecond \citep{Yoachim16}. 

With so many independent components, the sky brightness is potentially the most computationally expensive aspects of the simulations. We pre-compute sky brightness maps in all six Rubin filters in 5-15 minute time steps which can then be rapidly interpolated to exact times.

\subsubsection{Maintenance Downtime Model}

\begin{figure}
\epsscale{0.5}
\plotone{plots/downtimes}
\epsscale{1}
\caption{The simulated scheduled and unscheduled downtimes over 10 years.}\label{fig:downtime}
\end{figure}

The observatory model includes both scheduled and unscheduled downtime. Figure~\ref{fig:downtime} shows we simulate approximately 10\% of time lost to maintenance. The scheduled downtime allowance is currently about 22 weeks over the full 10 year survey. This is taken in either two one week periods twice a year, or a single two week period in alternating years. The unscheduled downtime allowance is approximately 20 weeks, in variable amounts of time, often as short as a single night. The scheduled downtime is planned during the same periods that are most likely to be bad weather, when possible. 

\subsection{The Scheduler}

Optimally scheduling telescopic observations is a traditionally difficult problem. Most observatories have traditionally scheduled observations by hand. The LCO and ZTF have implemented integer linear programming techniques to optimize their scheduling \citep{Lampoudi15, Bellm19}. Integer programming is difficult to use for Rubin because we have multiple science goals which are intended to be serviced simultaneously. Thus, there is no well-defined value which can be maximized when scheduling Rubin. \citet{Rothchild19} simulated Rubin observations with a very fast deterministic scheduler, essentially repeating a fixed raster pattern mostly along the meridian. This algorithm showed great promise, but had several downsides (such as occasionally pointing at the moon). For the Rubin scheduler, we follow the example in \citet{Naghib19} and use a Markov Decision Process to select most of the observations.

The Rubin scheduler is designed to provide real-time decisions on where and how to observe. Because we expect there to be weather interruptions, we need a system that can recover quickly. Unlike other traditional telescope schedulers, we do not try to optimize a large number of observations in advance, but rather use a decision tree along with a modified Markov Decision Process. The scheduler behavior is set by a small number of free parameters that can be tuned.

Earlier attempts at simulating LSST in \citet{Rothchild19} and \citet{Naghib19}. Eric's scheduler paper. 

Our baseline scheduler uses a three tier decision tree when deciding what observations to attempt. 

\subsubsection{Tier 1:  Deep Drilling Fields}

The first tier of the decision tree is to check if there are any deep drilling fields that should be executed. We typically have five DDFs in a simulation. 

For a DDF to be eligible to send a sequence to the observing queue, it must
\begin{itemize}
\item{Not currently be twilight}
\item{Have enough time to finish a sequence before twilight begins}
\item{Be in it's target hour angle range}
\item{The moon must be down}
\item{The DDF must not have exceeded it's limit of observations (typically $\sim$1\% of the total number of visits)}
\end{itemize}

If the DDF has not fallen behind, it will space sequences by at least 1.5 days. There is also a check to see if the DDF will be feasible and better observed later in the night, in which case no observations are requested.

If the above conditions are met, the DDF sends it's sequence of observations to the queue to be executed. There are currently no attempts at recovery if a sequence is interrupted. 

The spatial position of the DDF is dithered nightly up to 0.7 degrees.  The camera rotator is also varied nightly to be between -75 and 75 degrees with respect to the telescope. 

\begin{table}
\begin{centering}
\begin{tabular}{lcc}
\toprule
    Name &      RA &     Dec \\
    &          (Deg) &  (Deg) \\
    \hline
 ELAISS1 &   9.450 & -44.000 \\
 XMM-LSS &  35.708 &  -4.750 \\
   ECDFS &  53.125 & -28.100 \\
  COSMOS & 150.100 &   2.182 \\
    EDFS &  58.970 & -49.280 \\
    EDFS &  63.600 & -47.600 \\
    \hline
\end{tabular}
\caption{The location of the deep drilling fields used in our simulations.}\label{table:ddfs}
\end{centering}
\end{table}



\subsubsection{Tier 2:  The Blobs}

If there are no DDFs requesting observations, the decision tree moves to the second tier. This tier is the survey workhorse, executing $\sim$80\% of the simulation visits.  This tier will only request observations if it is not currently twilight, and there is at least 30 minutes before twilight begins.

A modified Markov Decision Process (MDP) is used to decide what sky region and filter combination to observe given the current conditions and observation history. Briefly, the MDP balances the desire to observe areas 1) that are closest to the optimal possible in terms of 5-sigma depth, 2) which have fallen behind the specified desired survey footprint, 3) are near the current telescope pointing and 4) in the currently loaded filter to minimize filter changes.  In addition to these core components, the MDP includes a mask around zenith, a 30 degree mask around the moon, and small masks around the bright planets (Venus, Mars, Jupiter). The end product of the MDP is a reward function that ranks the desirability of every point in the sky. Because this tier does not execute in twilight, we assume the reward function is relatively stable on 40 minute timescales.

A sky area around the reward function maximum that will take $\sim$22 minutes to observe ($\sim$35 pointings) is then selected. If possible, the area is selected to be be contiguous.  The exact position of the telescope pointings are determined by the sky tessellation, which is randomly oriented for each night. The camera rotator angle (relative to the telescope) is also randomized between $\pm 80$\ degrees each night.

A traveling salesman algorithm is used to put the pointings in an order that minimizes the slew time. The list of pointings are then repeated, usually in a different filter, ensuring moving objects can be detected.  One of seven possible filter combinations is used: $u+g$, $u+r$, $g+r$, $r+i$, $i+z$, $z+y$, or $y+y$.  We use 30 second visits for the majority of simulations. The official baseline uses visits comprised of two 15 second snaps.  


\subsubsection{Tier 3:  Greedy}

If it is during morning or evening twilight, or close to morning twilight, the DDFs and Blob surveys will pass and the decision tree goes to the third and final tier, the greedy surveys. 

The greedy surveys use a similar Markov Decision Process as in Tier 2, but rather than selecting large areas of sky to observe, the survey selects a single pointing at a time.  No attempt is made to observe greedy scheduled observations in pairs.  Since this tier is primarily used in twilight time, it only schedules observations in the redder filters $r$, $i$, $z$, and $y$.  

As with the Blob tier, the sky tessellation orientation is randomized each night so the final survey is spatially dithered. 



\begin{figure}
\epsscale{0.3}
\plotone{plots/night_plots/baseline_nexp1_v1_6_Count_note_like_DD_and_night810_HEAL_SkyMap.pdf}
\plotone{plots/night_plots/baseline_nexp1_v1_6_Count_note_like_blob_and_night810_HEAL_SkyMap.pdf}
\plotone{plots/night_plots/baseline_nexp1_v1_6_Count_note_like_greedy_and_night810_HEAL_SkyMap.pdf}

\plotone{plots/night_plots/baseline_nexp1_v1_6_altAz_Count_note_like_DD_and_night810_HEAL_SkyMap.pdf}
\plotone{plots/night_plots/baseline_nexp1_v1_6_altAz_Count_note_like_blob_and_night810_HEAL_SkyMap.pdf}
\plotone{plots/night_plots/baseline_nexp1_v1_6_altAz_Count_note_like_greedy_and_night810_HEAL_SkyMap.pdf}
\epsscale{1}
\caption{Examples of how the three scheduler tiers execute during a single night. Left panels show how a DDF sequence was observed during the night. Middle panels show observations taken as part of blob pairs. Right panels show the greedy observations taken in twilight time.  The panels from left to right show the different decision tiers the scheduler uses, with the DDFs as the top tier and the greedy algorithm as the bottom tier. } \label{fig:examplenight}
\end{figure}

\subsection{Filter Mounting Schedule}

In addition to the observations scheduler, we have a separate scheduler that decides which five filters should be loaded for the start of each night.  By default, we mount redder filters ($grizy$) when the moon is more than 40\% illuminated and bluer filters ($ugriy$) closer to new moon. 


\section{Survey Requirements and Metrics}
Basic survey strategy starting point and why - in more depth? Discuss metrics related to these requirements. 

Probably should show that all survey strategies evaluated do / need to meet these requirements (but maybe later?)

\subsection{SRD Metrics}

XXX--Relevant SRD requirements. 825 observations over 18,000 square degrees, fast revisits, and astrometry


XXX--relevant requirement to publish a list of upcoming planned observations (1?2?) hours in advance. 


\subsection{Science Metrics}

\section{Simulation Analysis}

Here we present the metrics we use to illustrate the performance of different simulations. By nesessity, we use a subset of all the metics we typically run.

\subsection{Parallax}


\subsection{Proper Motion}

\subsection{fO}

\subsection{SNe Ia}

\subsection{Tidal Disruption Events (TDE)}

\subsection{Weak Lensing}

Contributed by DESC.

\subsection{3x2 point Figure of Merit}

Contributed by DESC.

\subsection{Fast Microlensing}

Light curves contributed from the community.

While we also have a slow microlensing metric, we find very little variation over different siumulations.

Because the baseline footprint has sparse coverage of the Galactic bulge, the baseline value of this metric is relatively small, making the normalized values more volitile than the other metrics here.

\subsection{Number of Galaxies}

\subsection{Number of Stars}

\subsection{Bright Near Earth Objects}

\subsection{Faint Near Earth Objects}

\subsection{Trans Neptunian Objects}

\subsection{Radar Plots}

Explain that the radar plots have values normalized (typically to a relevant baseline run). For the parallax and proper motion metrics, the inverse of the errors are compared. For magnitudes, we plot madnitude difference (with larger values indicating deeper depths).

XXX--Maybe put in a table of the raw values for the 1.5 and 1.6 baseline runs.

\section{Survey Strategy Experiments} 

Broad outline of points to evaluate for survey strategy, and our approach in running the subsequent experiments (this should help make sense of what comes next)

\section{Feedback from white papers and SAC} 
Broad outline of points to evaluate for survey strategy, and our approach in running the subsequent experiments (this should help make sense of what comes next)

Discuss basic types of SAC recommendations. 


Reformat these, can add science impact highlights, but primary evaluation later. 


\section{Individual Scheduler Experiments}

Here we look at various experiments that explore varying a single aspect of the scheduler.


% old rolling
%\subsection{alt\_roll\_dust}

%These are motivated by the DESC collaboration and their desire to observe as many galaxies as possible. The WFD area is extended north and south and the galactic plane (defined by a dust map) is decreased down to $\sim$350 visits. The footprint also includes a northern stripe which gets $\sim220$\ visits. The motivation for the northern stripe is that it can provide image differencing templates for any gravitational wave detections in the north, as well as provide overlap with other survey missions (Euclid, WFIRST).

%These simulations tend to fall just short of the fOArea benchmark (e.g., only $\sim$17,500 square degrees receive 825 visits), but meet the formal SRD requirement by having a median number of 891 visits in the WFD area.  

%The dust exclusion zones make certain parts of the year undersubscribed. This suggests it might be relatively low-impact to add some bridges to the footprint across the galactic bulge and the galactic anti-center.  


\subsubsection{Dust With Alternating}

XXX--add the alt dust plots
\begin{figure}
\caption{}\label{fig:altdust}
\end{figure}
This uses the dusty footprint, and uses a basis function to encourage the scheduler to alternate between the north and south nightly. This was originally done in the altSched \citep{Rothchild19}. This can help keep light curve sampling optimally spaced. By using a basis function, we encourage alternating north/south, but it is not absolutely enforced, making it possible for the scheduler to avoid the moon.

There is no additional NES, however there is a strip in the north observed in $g$, $r$, $i$, and $z$.


XXX--science recap


% old rolling
%\subsubsection{roll\_mod2\_dust\_sdf\_0.20\_v1.5\_10yrs}

%Now we divide the WFD area into quarters, and employ a rolling cadence. For the first 1.5 years and final 2.5 years, the full footprint is used as normal. 

%<fig of first year> <year 1.5-2.5> <2.5-3.5>

%The use of dividing the WFD into quarters will be obvious when we combine with the altSched strategy.


%\subsection{baseline}

%We divide the sky into the standard WFD, NES, SCP and GP. The NES does not get observed in the u and y filters. 

%We run 2 versions of the baseline. One with 2x15s visits and one with 1x30s visits. The extra readtime needed for the 2-snap version results in the open shutter fraction dropping from 77\% to 72\%. Over the 10-year survey, this comes out to about 60 observations fewer for a typical point in the sky. 

%The 2-snap baseline meets the SRD requirements, but has very little extra contingency.

\subsection{Bulge}

We used recommendations from the SAC for different strategies for observing the galactic bulge. These simulations use the Big Sky footprint similar to the Olsen et al white paper.  

We use three footprints for bulge coverage 1) light coverage of the bulge and entire galactic plane, 2) the bulge as deep as WFD and 3) the bulge covered similarly to WFD, but with more observations in $i$.  For each of these strategies, we run a version with natural cadence and one where we boost the priority of the bulge if it has not been observed in 2.5 days. 

\begin{figure}
\epsscale{0.35}
\plotone{plots/pulled_plots/bulges_bs_v1_5_10yrs_Count_observationStartMJD_i_HEAL_SkyMap.pdf}
\plotone{plots/pulled_plots/bulges_bulge_wfd_v1_5_10yrs_Count_observationStartMJD_i_HEAL_SkyMap.pdf}
\plotone{plots/pulled_plots/bulges_i_heavy_v1_5_10yrs_Count_observationStartMJD_i_HEAL_SkyMap.pdf}
\epsscale{1}
\caption{Series of simulations trying different bulge observing strategies.}\label{fig:bulge}
\end{figure}


xxx-science recap


\subsection{DCR}

\begin{figure}
\epsscale{0.5}
\plotone{plots/pulled_plots/dcr_nham2_ugri_v1_5_10yrs_Nvisits_as_function_of_Alt_Az_HEAL_SkyMap.pdf}
\plotone{plots/pulled_plots/dcr_nham2_ugri_v1_5_10yrs_Count_observationStartMJD_HEAL_SkyMap.pdf}
\plotone{plots/pulled_plots/dcr_nham2_ugri_v1_5_10yrs_Hourglass_year_0-1_HOUR_Hourglass.pdf}
\epsscale{1}
\caption{Intentionally taking observations at higher airmass to measure DCR.}
\end{figure}

xxx--example plots for DCR

The LSST will not have an atmospheric chromatic corrector, thus difference imaging can be complicated by differential chromatic refraction (DCR). There is also potential science opportunities by being able to measure the chromatic shift in objects with sharp features in their SEDs.

These experiments look at how we could intentionally schedule a subset of images to be at high airmass so a DCR model could be built up. We test various combinations of filters to demand DCR observations (u+g, u+g+r, and u+g+r+i), and the number of observations to take at high airmass per year (1 or 2). 

Note, even with 2 high airmass observations per year, we would still expect some area of the sky to fall in chip and raft gaps.  It is also worth noting that in our baseline simulation, we observe a spot on the sky in u typically 60 times, or 6 times per year. Taking 2 high airmass observations per year in u decreases the final coadded depth by 0.15 mags.

XXX--science recap

   
\subsection{DDF}

We have run a variety of DDF strategies. Figure~\ref{fig:ddfexamples} shows the same observing season of the DDF ELIASS1 with 5 different strategies. 

\begin{itemize}
    \item{AGN: This strategy takes shorter DDF sequences more often. Only $\sim$2.5\% of visits are spent on DDFs, making the final coadded depths much shallower than other strategies.}
    \item{DESC: a strategy that split the blue and red filters to different days, emphasizing a 3-day cadence}
    \item{Baseline:  }
    \item{Daily: Similar to the baseline, but includes short DDF sequences that can execute daily so there are no long gaps between observations}
\end{itemize}


\begin{figure}
\plottwo{plots/ddf_plots/ddf_m5_AGN.pdf}{plots/ddf_plots/gap_hist_AGN.pdf}
\plottwo{plots/ddf_plots/ddf_m5_Baseline_v1_5.pdf}{plots/ddf_plots/gap_hist_Baseline_v1_5.pdf}
\plottwo{plots/ddf_plots/ddf_m5_Baseline_v1_6.pdf}{plots/ddf_plots/gap_hist_Baseline_v1_6.pdf}
\plottwo{plots/ddf_plots/ddf_m5_DESC.pdf}{plots/ddf_plots/gap_hist_DESC.pdf}
\plottwo{plots/ddf_plots/ddf_m5_Daily.pdf}{plots/ddf_plots/gap_hist_Daily.pdf}
\caption{Plots looking at one observing season of the DDF ELIASS1. }\label{fig:ddfexamples}
\end{figure}

XXX--massive table(s) of DDF, run, filter, coadded depth?

XXX--science discussion.  Point out we desperately need more DDF-specific metrics from the community. 


\subsection{Filter Distribution}

Testing a simple WFD-only footprint, but varying the requested ratio of observations in different filters.


\subsection{Footprints}

\begin{figure}
\epsscale{.25}
\plotone{plots/pulled_plots/footprint_add_mag_cloudsv1_5_10yrs_Count_observationStartMJD_HEAL_SkyMap.pdf}
\plotone{plots/pulled_plots/footprint_big_sky_dustv1_5_10yrs_Count_observationStartMJD_HEAL_SkyMap.pdf}
\plotone{plots/pulled_plots/footprint_big_sky_nouiyv1_5_10yrs_Count_observationStartMJD_HEAL_SkyMap.pdf}
\plotone{plots/pulled_plots/footprint_big_skyv1_5_10yrs_Count_observationStartMJD_HEAL_SkyMap.pdf}
\plotone{plots/pulled_plots/footprint_big_wfdv1_5_10yrs_Count_observationStartMJD_HEAL_SkyMap.pdf}
\plotone{plots/pulled_plots/footprint_bluer_footprintv1_5_10yrs_Count_observationStartMJD_HEAL_SkyMap.pdf}
\plotone{plots/pulled_plots/footprint_gp_smoothv1_5_10yrs_Count_observationStartMJD_HEAL_SkyMap.pdf}
\plotone{plots/pulled_plots/footprint_newAv1_5_10yrs_Count_observationStartMJD_HEAL_SkyMap.pdf}
\plotone{plots/pulled_plots/footprint_newBv1_5_10yrs_Count_observationStartMJD_HEAL_SkyMap.pdf}
\plotone{plots/pulled_plots/footprint_no_gp_northv1_5_10yrs_Count_observationStartMJD_HEAL_SkyMap.pdf}
\plotone{plots/pulled_plots/footprint_standard_goalsv1_5_10yrs_Count_observationStartMJD_HEAL_SkyMap.pdf}
\plotone{plots/pulled_plots/footprint_stuck_rollingv1_5_10yrs_Count_observationStartMJD_HEAL_SkyMap.pdf}
\epsscale{1}
\caption{The different survey footprints simulated.}
\end{figure}

We test a wide variation of possible survey footprints. Some of these are more realistic than others. 



\subsection{goodseeing}\label{ss:goodseeing}

These test the ability to ensure the entire WFD area is imaged in ``good seeing" conditions every year, here defined as FWHM of 0.7 arcseconds or better.  

These runs work well and it seems to add no particular overhead to the observing. It might make it more challenging to implement in operations, simply because the baseline simulation can simulate an entire night and pass off the list to be observed. If we want to run with the goal of collecting good seeing images, we will need to update the observing queue every time the seeing conditions change significantly, which could result in changing the upcomming observations more often than is desired.

% old rolling
%\subsection{rolling}
%We test breaking the WFD region up into 2, 3, and 6 declination bands which are then ``rolled". 

\subsection{Short Exposures}

We try taking additional short exposures (1s or 5s) twice or five times per year. Taking shorter exposures is a less efficient observing mode, but it seems to have little impact on the overall open shutter fraction.

\subsection{Spiders}

We look at keeping diffraction spikes aligned along CCD rows and columns. This may result in the camera rotator angle being much less randomized than our baseline rotational dithering strategy.

xxx--Science recap: Should be almost identical to baseline. 

\subsection{Third Observation}

For early identification of transients, it can be helpful to have more than two observations in a night. In these observations, we dedicate between 15 and 120 minutes at the end of the night to attempting to observe areas of sky that already have been observed.

\subsection{Twilight NEO Survey}

This is an implementation of white paper XXX, where we use twilight time to take short exposures along the ecliptic to search for NEOs. 

If we dedicate all twilight time to NEO searches, we fail to meet the SRD requirements. Thus we also check running the NEO survey every 2, 3, or 4 days.

\subsection{u60}\label{ss:u60}
The u-band observations can often be readnoise limited. We test doubling the u-band exposure time and cutting the number of exposures in half. This results in the u-band final coadded depth reaching $\sim$0.20 mags deeper. The g-band is also 0.10 mags deeper, with the rest of the filters essentially unchanged in final depth.

Note, we assume that 1x60s visit counts as 2 30s visits for the purpose of meeting the SRD value of 825 visits in the WFD area. Adopting longer exposures in u seems like a good idea, but the SRD will probably need to be modified to ensure it is not ambiguous.

\subsection{Variable Exposure Times}


\begin{figure}
\plottwo{plots/variable_expt_plots/baseline_spot.pdf}{plots/variable_expt_plots/varexpt_spot.pdf}
\caption{XXX-variable exposure time}\label{fig:varexptime}
\end{figure}

A test where we vary the exposure time based on the current conditions so individual exposures have similar depths. There is an argument that taking a full 30s visit in ideal dark time conditions results in ``wasted depth", as more objects and transients will be detected, but then it will be impossible to identify them as later visits are unlikely to be as deep. Similarly, taking a 30s visit in poor conditions will result in a shallow image which will be of limited use.

As with doing 60s u band exposures , this may require modifying the detailed specifics of the SRD as longer exposures may need to count as multiple visits.

Having variable exposure time introduces at least 6 new free parameters to the scheduler (the target individual depth for each filter), as well as the shortest and longest acceptable exposure times.  As with \ref{ss:goodseeing}, this would be more complicated to run in operations as the scheduler would need current conditions to calculate the modified exposure times, although the predicted sky brightness may be accurate enough.



\subsection{WFD Depth}

We vary what fraction of the observing time is dedicated to the WFD area, from 60\% to 99\% with and without the standard DDF surveys. Unsurprisingly, the SRD is not met if the WFD is only given 60\%.



\subsection{Rolling Cadences}

XXX--show off the different rolling runs from 1.6

\subsection{Even Filters}

Looking at varying how aggressive we are with matching filter choice to sky brightness.  

XXX--should show a hit on the coadded depth. Maybe some improvement in SNe

\begin{figure}
\plottwo{plots/pulled_plots/even_filters_g_v1_6_10yrs_Hourglass_year_0-1_HOUR_Hourglass.pdf}{plots/pulled_plots/even_filtersv1_6_10yrs_Hourglass_year_0-1_HOUR_Hourglass.pdf}
\caption{The filter distribution for the even filter simulations. Unlike the baseline simulations, more filters are observed in bright time.}\label{fig:even_filt_hourglass}
\end{figure}

\subsection{Aliasing}

Make a note that we looked at if we are aliased. It's not as bad as the old runs, so we probably don't need to do anything extra to avoid aliasing.

XXX--maybe add some plots from the aliasing notebook I made a while back.

\begin{figure}
\plotone{plots/alias_plots/aliasing.pdf}
\caption{Aliasing at a sample position in a baseline simulation. There are peaks at harmonics of 24 hours, but this is inevitable with a ground-based telescope. The aliasing seems much lower than earlier version of OpSim.}
\end{figure}


\section{FBS release v1.6}

Here we describe the runs done as part of the FBS 1.6 release.  Unlike previous releases, here we do a select few releases that combine various aspects of previous experiments.


%#################### Baseline #############################
\subsection{Baseline}

\begin{figure}
\epsscale{.5}
\plotone{plots/pulled_plots/baseline_nexp1_v1_6_10yrs_Count_observationStartMJD_HEAL_SkyMap.pdf}
\plotone{plots/pulled_plots/baseline_nexp1_v1_6_10yrs_Nvisits_as_function_of_Alt_Az_HEAL_SkyMap.pdf}
\plotone{plots/pulled_plots/baseline_nexp1_v1_6_10yrs_Hourglass_year_0-1_HOUR_Hourglass.pdf}
\epsscale{1}
%XXX ra dec map, alt az, an hourglass
\caption{The baseline v1.6 simulation. The top panels show the distribution of visits (all filters) in RA/dec and Alt/Az. The bottom panel shows the first year of observations color-coded by what filter was loaded. White regions represent scheduled and unscheduled downtime as well as weather downtime. The black curve on the bottom shows the moon phase.}\label{fig:baseline1.6}
\end{figure}


For the baseline strategy, we set the footprint to have 18,000 square degrees dedicated the the WFD survey. The WFD has a filter distribution of u:g:r:i:z:y of 0.31:0.44:1.0:1.0:0.9:0.9.
% WFD sum = 4.55

We include coverage of the Galactic Plane (GP) and South Celestial Pole (SCP). These areas are set to have 20\% the number of counts of the WFD (if a spot in the WFD has 900 visits, points in the GP and SCP will have 180 visits). The GP and SCP are set to have equal number of visits in all filters.

The total breakdown of target observing time is 85\% for WFD, 6\% for the NES, 6\% for the GP and NES, and 5\% for DDFs.

The North Ecliptic Spur (NES) is observed with only the g, r, i, and z filters. The NES area is set to have one-third the number of visits of the WFD.  The filter distribution is set to g:r:i:z of 0.2:0.46:0.46:0.4. 

While the different survey areas are covered to different depths, the baseline scheduler treats them identically and only tries to maintain the proper ratios of area coverage. This means blocks of observations can be scheduled that cover the different regions seamlessly. It also means we have no additional constraints on how the regions are observed. For example, we currently do not reserve ``good seeing" time for the WFD area. 

The baseline survey includes the 4 announced Deep Drilling Fields as well as a pair of fields that overlap the Euclid Deep Field South. (XXX--maybe put in a table?). Each individual DDF is set to take a maximum of 1\% of the total visits (the Euclid fields are set to take 1\% combined). The DDF sequence is ux8, gx20, rx10, ix20, zx26, and yx20, all with 30s exposures. For any given sequence, only the five filters loaded in the camera are executed. By default, we remove the u filter when the moon is more than 40\% illuminated at the start of the night.


We run 2 baseline simulations, one with 1x30s visits and one with 2x15s visits.  The main difference is the additional readout time in the 2x15 version drops the open shutter fraction from 77\% to 72\%. This puts the 2x15s simulation close to failing the SRD FO metric, with some parts of the WFD region only reaching 824 observations (the median is still 892). 

For the rest of the simulations in v1.6 we use 1x30s visits.  If 2x15s is required there will be a significant drop in the number of visits, and areas outside of the WFD may need to be scaled back to still meet SRD requirements.

The baseline surveys use the following strategies:

XXX-DDF strategy:


XXX--main non-twilight time strategy: 

Observations are taken in 44 minute blocks (22 minutes for an initial area, 22 minutes to repeat the area). The size of the blocks can scale slightly to try and fill time before twilight (e.g., it will expand to a pair gap of 25 minutes if there are 50 minutes until twilight). All observations are taken in pairs, with potential combinations of u+g, u+r, g+r, r+i, i+z, z+y, or y+y. The order of observations can change depending on what filter is currently loaded (e.g., if the scheduler decides to observe a g+r sequence, the r observations will be taken first to eliminate a filter change if possible.)

The camera rotator angle (relative to the telescope) is randomly set each night between -80 and 80 degrees.  XXX-the angle is set when the block is scheduled, so there can be a few degrees of drift between when the rotator angle is computed and when the observation is actually taken.

The basis functions used are:

The 5-sigma depth (for both filters in the pair being taken), the footprint uniformity (again, in both filters), the slewtime, and a basis function that rewards staying in the current filter.  We also include a basis function that rewards taking 3 observations per year per filter over the entire survey footprint.  

The zenith is masked (to avoid long azimuth slews), and a region 30 degrees around the moon is masked. The bright planets (Venus, Mars, and Jupiter) are masked with a 3.5 degree radius. 

XXX--twilight strategy
If the sun is higher than XXX degrees, or there is not enough time remaining to take observations in pairs, 

%#################### DDF Heavy #############################
\subsection{DDF Heavy}

\begin{figure}
\epsscale{0.5}
\plotone{plots/pulled_plots/ddf_heavy_v1_6_10yrs_Count_observationStartMJD_HEAL_SkyMap}
\plotone{plots/pulled_plots/ddf_heavy_v1_6_10yrs_Nvisits_as_function_of_Alt_Az_HEAL_SkyMap}
\plotone{plots/pulled_plots/ddf_heavy_v1_6_10yrs_Hourglass_year_0-1_HOUR_Hourglass}
\epsscale{1}
\caption{DDF Heavy simulation. Nearly identical to the baseline, but giving as much time as possible to DDF observations.}\label{fig:ddfheavy}
\end{figure}


This run is nearly identical to the baseline, but gives a large fraction of time to the deep drilling fields. Each of the five DDFs takes between 2.4 and 2.9\% of the survey, with 13.4\% of all visits being used for DDF observations. The baseline has 4.6\% of visits used for DDFs.  This is enough time that the WFD area near the DDFs fails to reach 825 visits over 10 years, but the SRD requirement is formally still met because the median WFD point is observed 875 times.

XXX--For each of these 1.6 runs, maybe an include a science impact recap? Maybe a table to compare median coadded depth in each filter, a radar relative to baseline (and same scale across all of them), and a ew lines of explination of what we think happened, what metrics we need (e.g., here we could say we need AGN and other DDF relevant metrics).

%#################### Barebones #############################
\subsection{Barebones}

\begin{figure}
\epsscale{0.5}
\plotone{plots/pulled_plots/barebones_v1_6_10yrs_Count_observationStartMJD_HEAL_SkyMap}
\plotone{plots/pulled_plots/barebones_v1_6_10yrs_Nvisits_as_function_of_Alt_Az_HEAL_SkyMap}
\plotone{plots/pulled_plots/barebones_v1_6_10yrs_Hourglass_year_0-1_HOUR_Hourglass}
\epsscale{1}
\caption{The barebones simulation essentially covering just the WFD area as efficiently and deeply as possible.}\label{fig:barebones}
\end{figure}


The barebones simulation is an example of a survey where we focus exclusively on meeting the SRD requirements, with little optimization for science.

The survey footprint is restricted to the standard 18,000 square degree WFD area only. Deep drilling fields are included, but capped at $\sim2.5$\% of the total visits. Visits in u and y are unpaired, while the rest of the filters are paired in the same filter. This results in very few filter changes in a night. 

There are a wide number of reasons why this would be a terrible survey strategy--detected transients would have no color information, photometric uber-calibration would be difficult with the galactic plane gap, a lack of solar system object because the NES is not included, etc.  The main purpose is to show the scheduler can reach very near the theoretical maximum for open shutter fraction, with this run reaching 80\%. Also, we can note the fONv metric reaches 1,148 which is 40\% higher than the SRD requirement of 825. This also implies that we can observe a maximum of $\sim115$\ WFD visits per year in the event we want to adjust the scheduler to attempt to catch up on the WFD progress. 



%#################### Data Management Heavy #############################
\subsection{Data Management Heavy}

\begin{figure}
\epsscale{0.5}
\plotone{plots/pulled_plots/dm_heavy_v1_6_10yrs_Count_observationStartMJD_HEAL_SkyMap}
\plotone{plots/pulled_plots/dm_heavy_v1_6_10yrs_Nvisits_as_function_of_Alt_Az_HEAL_SkyMap}
\plotone{plots/pulled_plots/dm_heavy_v1_6_10yrs_Hourglass_year_0-1_HOUR_Hourglass}
\epsscale{1}
\caption{The DM heavy simulation. Similar to the baseline, but the alt/az plot shows how some observations are being taken at high airmass to support DCR modeling.}\label{fig:dmheavy}
\end{figure}


This is simulations includes various modifications that may be helpful for Data Management purposes. For the WFD region in u, g, and r a few images per year are taken at high airmass so that DCR correction models can be made.

The camera rotator angle is set so that diffraction spikes fall along CCD rows and columns. This helps with difference imaging so the maximum possible area can be used, but may result in weak lensing systematics.

Each year, the scheduler prioritizes taking g,r,i images of the whole sky in good seeing conditions (defined as 0.7\arcsec effective FWHM or better).

The DDF fields use larger dithers, up to 1.5 degrees, compared to the default 0.7 degree maximum.


%#################### Rolling Extragalactic #############################
\subsection{Rolling Extragalactic}

\begin{figure}
\epsscale{0.5}
\plotone{plots/pulled_plots/rolling_exgal_mod2_dust_sdf_0_80_v1_6_10yrs_Count_observationStartMJD_HEAL_SkyMap}
\plotone{plots/pulled_plots/rolling_exgal_mod2_dust_sdf_0_80_v1_6_10yrs_Nvisits_as_function_of_Alt_Az_HEAL_SkyMap}
\plotone{plots/pulled_plots/rolling_exgal_mod2_dust_sdf_0_80_v1_6_10yrs_Hourglass_year_0-1_HOUR_Hourglass}
\epsscale{1}
\caption{The Rolling Exgal simulation. }\label{fig:rollingexgal}
\end{figure}


\begin{figure}
\plottwo{plots/rolling_plot/baseline_nexp1_v1_6_Count_filter_note_not_like_DD_HEAL_SkyMap.pdf}{plots/rolling_plot/rolling_exgal_mod2_dust_sdf_0_80_v1_6_Count_filter_note_not_like_DD_HEAL_SkyMap.pdf}
\plottwo{plots/rolling_plot/baseline_nexp1_v1_6_Count_filter_night_gt_1278_375000_and_night_lt_1643_625000_and_note_not_like_DD_HEAL_SkyMap.pdf}{plots/rolling_plot/rolling_exgal_mod2_dust_sdf_0_80_v1_6_Count_filter_night_gt_1278_375000_and_night_lt_1643_625000_and_note_not_like_DD_HEAL_SkyMap.pdf}
\caption{Illustration of how rolling cadence works. The top panels show the number of observations after 10 years (all filters) for the baseline and rolling exgal simulations (excluding DDF observations). Both simulations have very smooth WFD coverage, with $\sim$900 observations.  The lower panels show the number of observations taken between 3.5 and 4.5 years into the survey.  The baseline WFD remains smooth, while the rolling exgal simulation has declination stripes of high and low counts.  }\label{fig:exgalroll}
\end{figure}


The rolling extragalactic is motivated by cosmological drivers. The footprint is modified so the 18,000 square degrees of the WFD are placed in low-extinction regions. The simulation also executes a half-sky rolling scheme, which should result in well sampled lightcurves for extragalactic transients.

xxx--add a plot showing the lines of how rolling works.  

xxx--mention that doing rolling in quarters ensures that half the alert stream is always available to northern telescopes for follow up. 


%#################### Milky Way Heavy #############################
\subsection{Milky Way Heavy}
\begin{figure}
\epsscale{0.5}
\plotone{plots/pulled_plots/mw_heavy_v1_6_10yrs_Count_observationStartMJD_HEAL_SkyMap}
\plotone{plots/pulled_plots/mw_heavy_v1_6_10yrs_Nvisits_as_function_of_Alt_Az_HEAL_SkyMap}
\plotone{plots/pulled_plots/mw_heavy_v1_6_10yrs_Hourglass_year_0-1_HOUR_Hourglass}
\epsscale{1}
\caption{The Milky Way heavy simulation. }\label{fig:mwheavy}
\end{figure}

The Milky Way heavy simulation covers the Galactic bulge, LMC, and SMC as part of the WFD area.  


%#################### Solar System Heavy #############################
\subsection{Solar System Heavy}
\begin{figure}
\epsscale{0.5}
\plotone{plots/pulled_plots/ss_heavy_v1_6_10yrs_Count_observationStartMJD_HEAL_SkyMap}
\plotone{plots/pulled_plots/ss_heavy_v1_6_10yrs_Nvisits_as_function_of_Alt_Az_HEAL_SkyMap}
\plotone{plots/pulled_plots/ss_heavy_v1_6_10yrs_Hourglass_year_0-1_HOUR_Hourglass}
\epsscale{1}
\caption{The Solar System heavy simulation. }\label{fig:ssheavy}
\end{figure}

Include ecliptic coverage through the galactic plane

NEO survey at twilight--Note, a NEO survey taking short exposures will drastically increase the data throughput. DM needs to check if this would be feasible.  We also need to check with the camera team that taking short exposures for an extended time will not be a thermal issue.

only i,z,y in twilight, making sure we observe more r-band in non-twilight and in pairs.

include r+r pairs

For regular (30s visit) twilight observations, avoid the ecliptic (ensuring they are always taken in pairs in non-twilight time)

%#################### Combo Dust #############################
\subsection{Combo Dust}

\begin{figure}
\epsscale{0.5}
\plotone{plots/pulled_plots/combo_dust_v1_6_10yrs_Count_observationStartMJD_HEAL_SkyMap}
\plotone{plots/pulled_plots/combo_dust_v1_6_10yrs_Nvisits_as_function_of_Alt_Az_HEAL_SkyMap}
\plotone{plots/pulled_plots/combo_dust_v1_6_10yrs_Hourglass_year_0-1_HOUR_Hourglass}
\epsscale{1}
\caption{The Combo Dust simulation. }\label{fig:combodust}
\end{figure}


This simulation attempts to improve several science cases compared to the baseline simultaneously. The footprint used here starts with defining the WFD area as 18,000 square degrees with low extinction. Then an additional 2,000 square degrees are added to WFD to cover the bulge, the ecliptic through the galactic plane, the LMC and SMC, and an outer disk region. Dusty areas of the sky and the South Celestial Pole are covered at about one-quarter the WFD depth. The NES is covered in $g$, $r$, $i$, and $z$. The footprint also includes very light coverage to the northern limit of the telescope so there can be templates for ToO events on the entire visible sky. 

The footprint has 35 free parameters for setting region locations and filter ratios. Many of these have have been set by eye or use historical values of questionable providence. Moving forward with such a footprint 

XXX--this one is also rolling.

% Let's make a table comparing everything. Maybe one for the basics, then one for science?


\section{Science impacts}

The broad categories of experiments covered in the FBS 1.4, 1.5 and 1.6 releases address different aspects of survey strategy. While each family of simulations maintained the approach of varying a single kind of parameter (such as the amount of time devoted to triplets of visits in the `third\_visit' runs or the footprint coverage in the `footprint' runs), the underlying science optimization questions can cover multiple families. In addition, when looking at individual science cases, there can be effects that cover multiple families but have the same underlying cause -- preferring more visits in the WFD or needing more visits in $u$ band, for example .. which can be the result of variations in survey strategy in multiple different families (i.e. we get to the same place via different means). 

\subsection{Individual Visit Length}
What to do - 1x30s vs. 2x15s? 1x30s much more efficient (show rough calculation of overhead) than 2x15s, but may have drawbacks due to cosmic ray rejection and potential to miss very rapid transients (or WD detection .. ref white paper). Subtle drawback that 2x15s gives the same "midpoint exposure time" across FOV, 1x30s does not. 

Show difference in 1x30s vs. 2x15s in whatever is our 'standard baseline' at this point. 

There has been thought of using a variety of exposure times if we use two snaps (e.g., 5s + 25s). Because there are not plans to release catalogs from individual snaps, it's not clear if this would enable much new science.

Show effect of 7\% loss in efficiency when attempting to combine minisurveys in various configurations (assume we will find some combinations possible with single exposure visits that are impossible with two snaps). 

Also possible to use variable exposure time depending on seeing and sky brightness conditions. Shorter exposures in good conditions keeps us from observing ``wasted" depth, letting us take longer exposures in poor conditions. This does introduce a host of new free parameters (an ideal target depth for each filter and minimum and maximum exposure times).  This would might require rewording the SRD to ensure, e.g., that 20s visits in good conditions count for the number of visit requirement.

Relevant metrics: total number of visits, number of visits per field/filter

\subsection{Intra-night Cadence}

What to do for visit sequence within a night? White paper support for multiple filters within a night (except TNOs maybe?). Potential drawbacks - less efficient (show effect on efficiency). This applies to WFD primarily, but we've applied to any survey that did not have their own specifications (so, everywhere). 

Extension of pairs to $u$ band and $y$ band (show effect). 

Relevant metrics: inter-night visit gaps and SN discovery, SSO discovery/characterization, transient and variable discovery (??), number of visits

\subsection{Survey Footprint}
What to do for WFD footprint? SRD not specific, DESC want low-extinction sky (and depth), but WFD is generally the area of sky that receives the most visits, so generally other science will also benefit from more visits to their relevant areas (particularly galactic plane .. for time-domain studies primarily, not depth)

Relevant metrics: area of sky with 825 visits (under particular restrictions, like total coadded depth and individual image seeing and dust extinction), number of galaxies, number of resolved galaxies, SSO discovery, transient and variable star discovery, astrometry in the galactic plane (?)

\subsubsection{Northern minisurveys}
Add extension to cover Euclid/DESI with various numbers of visits

Observing NES 

Effect of adding or removing these minisurveys

Relevant metrics: SSO discovery and characterization (particularly active asteroids), depth and number of visits through remainder of North

\subsubsection{Southern minisurveys}
Add extension over south celestial pole, LMC/SMC with various numbers of visits

Effect of adding or removing these minisurveys

Relevant metrics: number of visits and coadded depth over SCP, discovery of variables in LMC/SMC (see Olsen white paper for metrics?)

\subsubsection{Low Galactic Latitudes}
Discussion of definitions from SAC and recommendations for visits

Effect of adding or removing these minisurveys

Relevant metrics: number of visits, astrometry in bulge, discovery of variables/transients/microlensing in bulge (?)


\subsection{Rolling cadence}
Motivation for a rolling cadence (more frequent visits in some years)

Different options for rolling and explanation of how implemented

Should really include discussion of recovery from bad weather years and simulation of same

Relevant metrics: Maintain astrometry requirements, SN discovery, SSO discovery and characterization,  Transient and variable discovery, uniformity of coadded depth / number of visits, 



\subsection{Twilight Observing}
Discuss need for twilight observing to meet SRD goals (weather, total amount of time available)

Add NEO twilight survey, add DCR white paper (season extension visits?)

Effect of adding or removing these minisurveys

Relevant metrics: NEO discovery, number of visits and coadded depth (and uniformity) in WFD, measurement of DCR, season length

\subsection{Deep Drilling Fields}
Discuss purpose and how these are scheduled (very different from other fields)

Discuss potential cadences (AGN/ DESC) and how these differ, and our combination of the two

Discuss timing issues with oversubscription (and how much of a problem this could be, what if worse weather?) -- include location of fifth DD field

Effect of adding or removing these minisurveys

Relevant metrics: number of visits and coadded depth for DD, SN detection in DDFs, AGN detection in DDFs
*[solar system minisurvey DDF?]

\subsection{ToO modes}
Discuss impact of ToO, and how we could implement ToOs in scheduler (various modes: straight to queue by hand or set up known program and supply trigger, etc. -- that we're evaluating the second?)

Any ToO survey should also take into account that chip and raft gaps mean full sky coverage will require multiple images with spatial dithering.

Discuss how we can have a low coverage region to the north to maintain templates for all possible ToOs, or we could decide ot only search for ToOs that are likely to be in the WFD area.

Relevant metrics: frequency of achieving ToO observations, number of visits and coadded depth in other surveys (WFD or other minisurveys that may be in particular contention)

\subsection{Number of visits in WFD}
Overall survey number of visits vs. number of visits in WFD (see twilight survey, DCRham surveys, variable exposure, shortexp surveys)


\section{Conclusions}
Hopefully here we pare down the evaluation of 100s of runs (like promised) to a set of between 10 to 20 (if this is possible, after combining along different axes). 
The results should come with some basic comments about what's particularly good or bad in each of these areas and how we arrived at these general options. 

Metrics we know we need to get from the community:
\begin{itemize}
    \item{Photometric redshift performance, especially as it relates to filter distribution}
    \item{Weak Lensing systematics, especially as related to camera rotator angle}
    \item{Deep Drilling Field metrics beyond coadded depth (e.g., AGN performance)}
    \item{Deep Drilling metrics that are sensitive to the spatial dither strategy}
    \item{Transient early classification metric}
    \item{More populations in the Galactic plane beyond the simple number of stars.}
\end{itemize}


\section{Outstanding Questions}\label{sec:questions}

Here we go through some of the outstanding questions that the SCOC and scientific community can help resolve in order to converge on a final scheduler strategy for the Rubin Observatory. 

\subsection{Exposure Time(s)}

We will probably need on-sky data to make a final answer to this question, but we need to eventually decide how many snaps to take in a visit. We have run the baseline simulation with both 1x30s visits and 2x15s visits. Another possibility is using variable exposure times to make the single visit depths more uniform.

Other questions related to exposure time
\begin{itemize}
    \item{Should we change the u-band to default to 60 second exposures to ensure they are not readnoise dominated? This might require decreasing the SRD 825 visit value. This choice would also severely limit $u$\ band time domain science (e.g., TDE early detection)}
    \item{Should we include some very short exposure time exposures. That would let us have better tie-in with other surveys (e.g., Gaia).  It is relatively little exposure time, but the readout time means it is a low-efficiency way to operate the telescope.}
    \item{Should we decrease the exposure time in twilight to keep the saturation level reasonable?}
    \item{Should we use variable exposure times so individual exposures have more uniform depth? In poor observing conditions, we would have fewer exposures that were londer and in good conditions we would have more observations that are shorter.}
\end{itemize}


\subsection{Pairs and Filter Choice}

There is a strong preference to take observations in pairs. Closely spaced observations let the pipeline identify moving objects. Similarly, observations in different filters are essential for transient classification. 

The baseline survey (and most of our other experiments), take pairs in neighboring filters (e.g., u+g, g+r, etc).  We should verify that this is a good pairing strategy. Similarly, we have done experiments where we attempt to observe a third observation in a night. 

Taking pairs in different filters does increase time spent changing filters, but it's something like a 4\% hit that seems totally worth the science gain so far. 

Our baseline strategies heavily prefer to take y-band observations in bright time. While this is optimal for the possible SNR, it can result in long gaps between observations in bluer filters. 

Similarly, we could expand or constrict the filters we pursue in twilight time. XXX--some 1.5 runs I think that vary which filters we run in twilight.

\subsection{Survey Contingency}

How much contingency should we aim for when designing the survey strategy?  Currently, with what we believe is a conservative weather closure policy, we can meet SRD requirements with 2x15s visits, but can cover a larger footprint and do more science cases with 1x30s snaps.  

\subsection{Deep Drilling Fields}

We have run a variety of Deep Drilling strategies. The DDF strategy is largely separable from the rest of the survey design, and we have a number of proposals for DDFs that we have yet to explore (e.g., rolling DDFs where a single DDF is completed in one observing season).  We have started experimenting with pre-scheduling DDF observations. 

\begin{itemize}
    \item{What fraction of the survey should be dedicated to the DDFs?}
    \item{Should DDFs be preferentially executed in dark time, or is it more important to maintain cadence?}
    \item{Where should the DDFs be placed (can we finalize the 5th DDF as a Euclid double-pointing)?}
    \item{What is the preferred dithering strategy (spatially and rotationally) for the DDFs? There is tension in that DM generally prefers larger dithers for calibration and co-addition purposes, while science cases prefer smaller dithers to preserve the area that reaches the deepest levels.}
    \item{Should we try ``rolling" the DDFs, completing DDF observations in a field in only a few years?}
\end{itemize}

\subsection{Rotational Dithering}

By default, we select a random camera rotation angle (wrt the telescope) nightly. This creates minimal additional slewtime, and seems to provide adequate angular randomization.  We currently have no science metrics that depend on the angular distribution, and this should be something very important to weak lensing science (although we do not have a metric to measure this).

We have also experimented with setting the camera rotation angle to ensure stellar diffraction spikes fall preferentially along rows and columns. 

\begin{itemize}
    \item{How should we rotationally dither visits?}
\end{itemize}

\subsection{Spatial Dithering}

For the wide area regions we have had excellent results randomizing the tessellation orientation nightly. This does result in a small percent of time being spent observing outside the desired survey footprint. The alternative would be to limit the amount one dithers out of the footprint, but then one risks imprinting systematics on objects near the footprint border (e.g., an object is never observed in the center of the focal plane, only by outer rafts).

% XXX--There's no question here, the SCOC doesn't have to say anything about dithering in the wide area surveys.

\subsection{Survey Footprint}

Perhaps the biggest question, what should we set the survey footprint to be?

\begin{itemize}
    \item{How should we cover the Galactic plane?}
    \item{How should we observe the Galactic bulge?}
    \item{Should we avoid areas of high dust extinction for the WFD area?}
    \item{What is the ideal filter distribution to use? It would be nice to have a photo-z metric to help make this decision.}
    \item{What is the ideal filter distribution in the GP and SCP?}
    \item{Should we cover the LMC and SMC as part of the WFD survey? As their own DDF-like survey? We have few metrics that touch on LMC/SMC science directly.}
    \item{Should we add area in the north to overlap with Euclid, WFIRST, and/or DESI?}
\end{itemize}

Once the general survey footprint is decided, we can fine-tune the footprint (e.g., tapering the WFD region slightly around the RA with multiple DDFs, and flaring at under-subscribed RAs).

\subsection{Rolling Cadence}

We have gone through several iterations of rolling cadence, and now have started to converge on a technique that does not seem to impact the final survey depth. 

\begin{itemize}
    \item{Should we use a rolling cadence strategy?}
    \item{Should we roll just the WFD area, or other regions as well?}
\end{itemize}

\subsection{Best Use of Twilight Time}

Our baseline simulation uses twilight time to fill in WFD observations in redder filters ($rizy$). We can use some of the time to conduct a NEO survey. We can also vary which filters get used in twilight time. The baseline greedy algorithm used in twilight is known to be rather unstable, so we could also try running more contiguous blocks in twilight. We could also emphasize targeting areas that have already been observed 4 or more times in the night, potentially gathering important color information for a small number of transients.


\subsection{Target of Opportunity}

Currently, the only expected ToO use of Rubin observatory is follow up of gravitational wave detections.

\begin{itemize}
    \item{When should Rubin interrupt observations to look for GW optical counterparts?}
    \item{Do we look for GW events in the WFD area, or anywhere on the sky?}
    \item{Should we expand the survey footprint so we have image differencing templates over the entire accessible sky, in at least a few filters?}
    \item{Should Rubin plan on observing the entire light curve of ToO events, or make observations primarily for detection/classification and leave detailed follow up to other observatories?}
    \item{What filter combination and dither strategy (filling chip and raft gaps) should be used for observing ToO triggers? } 
\end{itemize}


\subsection{Image Differencing Templates, DCR}

Do we need to do anything special to ensure we have adequate image templates? A certain number of observations per year? A certain fraction of images taken in good seeing conditions? 

If we need to start considering image quality, that makes it more difficult to simulate a night ahead of time and maintain the list of upcoming observations.

Should we intentionally extend to high airmass to facilitate DCR modeling? Note that in the baseline, we only image a location in the WFD region $\sim$9 times per year in $g$ and $\sim$6 times in $u$. Also, we have chip and raft gaps, so if we want to build a DCR model for the entire sky in $g$, we might be dedicating 1/3 of the $g$ observations in a year to DCR. If we switch to 60s $u$ band exposures, there would be no observations beyond building the DCR model. 

There have been claims that measuring DCR can be used for science.  We do not have any metrics that demonstrate any gains, and the loss of depth is noticeable. In theory, we could combine the DCR measurements to extend the season length of observations as well (e.g., only take DCR template images near twilight in the direction of the sun).

\subsection{Satellite Megaconstellations}

Starlink is poised to launch thousands of LEO satellites. Observations so far imply that final-orbit Starlink satellites should not saturate Rubin exposures, and thus can be masked fairly easily in the image reduction pipeline. 

Do we need any further satellite mitigations? Will NEO twilight surveys still be viable in the presence of megaconstellations, or should we use twilight strategies that avoid the horizon?

Figure~\ref{fig:megasat} shows how illuminated megaconstellations in LEO would leave numerous streaks on Rubin images.

% from https://github.com/yoachim/satellite_collisions
\begin{figure}
\plottwo{plots/sat_plots/ten_min_12k.pdf}{plots/sat_plots/tenmin_example.pdf}
\caption{Alt/az projection of simulated satellite megaconstellations as seen from the Rubin Observatory site after twilight has ended. } \label{fig:megasat}
\end{figure}




% Make sure lsst-texmf/bin/generateAcronyms.py is in your path
\section{Acronyms} \label{sec:acronyms}
\input{acronyms.tex}
