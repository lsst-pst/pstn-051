% replace large sections with \include files 

\section{Introduction}

Note: This paper needs to focus on survey strategies and their evaluation. 

Introduction - cover basic idea of survey simulator, scheduler and weather/telescope models. 

Cover basic survey strategy starting point - wide area, frequent coverage, ten year timespan - and why. 

Mention COSEP and call for white papers - idea is to do the best science we can, add last 10\% "best" science. 

Earlier attempts at simulating LSST in \citet{Rothchild19} and \citet{Naghib19}.

\section{Survey Simulator Overview}
Probably need some reference to what survey scheduler was used / how it was set up for various runs, how the runs were performed, and what the input weather and telescope models were like. 

\section{Basic Survey Requirements}
Basic survey strategy starting point and why - in more depth? Discuss metrics related to these requirements. 

Probably should show that all survey strategies evaluated do / need to meet these requirements (but maybe later?)

\section{Feedback from white papers and SAC} 
Broad outline of points to evaluate for survey strategy, and our approach in running the subsequent experiments (this should help make sense of what comes next)

Discuss basic types of SAC recommendations. 

%\section{Overview of Metrics}
\section{Metrics}

There are many options for evaluating the output of the survey strategy experiments, including high-level science-oriented metrics and more basic metrics measuring simpler changes in survey characteristic. One of the primary goals for the LSST Metrics Analysis Framework (MAF) package was to make it easier for both the project and community members to write metrics to evaluate these outputs. This effort has had some significant successes; SRD-level metrics have been written that cover the primary requirements for the SRD, the DESC working groups have made good progress in writing metrics for their evaluation of the simulations, and the Solar System collaboration has contributed substantial metrics. In other areas, it has been more difficult for the community to engage and contribute directly to MAF; for some of these areas, we have been able to help get metrics running, but clearly there are areas which are lacking definitive metrics. Many of the areas which are lacking relate directly to time domain studies, a critical area for the LSST. We acknowledge this issue and encourage further work by the community, particularly over the next year. 

Here we make a brief summary of some of the top-level science-related metrics. There are thousands of metrics which are run as part of standard MAF analysis (many of which relate to simple analysis of observation metadata like seeing, airmass, sky brightness, etc.); for broad comparisons between simulations we pick a very limited subset of these metrics intended to discover or highlight differences between the simulation survey strategies or to cover major areas of science. 

\subsection{SRD Metrics}

The SRD metrics are designed to cover the primary science requirements laid out in the SRD; the most relevant of these relate to the number of visits per pointing across the WFD region, the area of the WFD region, the parallax and proper motion errors and the number of rapid revisits (on timescales between a few to 40 seconds) per point on the sky. While we check all of these metrics for all runs, the most sensitive to changes in the survey strategy is the number of visits across the WFD, tracked in the fO metric, since we are often attempting to distribute visits into other parts of the sky for other science. 

The fO metric calculates the total number of visits per point on the sky, then calculates how much area is covered with how many visits. This can be summarized across two axes; the amount of area that receives at least 825 visits per pointing (`fO Area') or the median (or minimum) number of visits that the most frequently visited 18,000 square degrees receives (`fONv MedianNvisits' or `fONv MinNvisits'). The first version, fO Area, tends to be somewhat unstable; the survey hardly ever observes more than 18,000 square degrees to at least 825 visits, because we don't program in larger WFD areas, but if the number of visits across the WFD area falls below 825, the resulting fO Area value will fall rapidly (because we cover the sky uniformly). While fO Area is useful to check, a more useful number is fONv MedianNvisits or MinNvisits. The value of fONv MedianNvisits tells us how many visits the typical field in the top 18,000 square degrees receives; fONv MinimumNvisits tells us the fewest number of visits any of those top 18,000 square degrees received. Typically we see fONv MedianNvisits scales more smoothly with the fraction of visits devoted to WFD and likely represents science metrics that depend on having a reasonably large amount of visits over the entire WFD well. 

The radar plots use fONv MedianNvisits, the Median Parallax Error, and the Median Proper Motion Error. The astrometry metrics both assume an $r$=20 magnitude star with a flat SED. When plotted in radar plots, we compare the reciprical of the astrometry uncertainties so that larger values on the radar plot can always be interpreted as ``better".

\subsection{Solar System Science Metrics}

Solar System science metrics include \href{https://github.com/lsst/sims_maf/blob/master/python/lsst/sims/maf/metrics/moMetrics.py#L215}{discovery metrics} (with various discovery criteria, such as detections in 3 nights with pairs of visits within a 15 night window) and characterization metrics (ie. how many colors for objects can we measure, and can we determine a light curve or even shape measurement from the lightcurve), contributed by both project and science collaboration. The most important metric for solar system objects is discovery; finding the objects is the first priority. Characterization metrics are secondary metrics. For each of these metrics, we generate input observations using an appropriate solar system population: Potentially Hazardous Asteroids (PHAs) and Near Earth Objects (NEOs) based on a model by \citet{2018Icar..312..181G}, Main Belt Asteroids (MBAs) and Jovian Trojans based on the S3M model from \citet{2011PASP..123..423G}, and TransNeptunian Objects (TNOs) based on the L7 model from the Canada France Ecliptic Plane Survey (CFEPS) \citep{2009AJ....137.4917K, 2011AJ....142..131P}. These populations move at varying rates and cover varying amounts of the sky. NEOs move over much of the sky during the lifetime of the survey, so are less sensitive to footprint variations, but tend to have much more strongly varying brightnesses, thus are sensitive to the number and timing of visits (must be observed when they are bright). TNOs move very slowly, not more than a few fields of view over the lifetime of the survey, so are quite sensitive to footprint, however they are relatively consistent in their brightness; thus they are less sensitive to the overall number of visits at a particular point in the sky, once a threshold has been met. 

For each of these populations, we calculate the population completeness due to discovery with the LSST at the end of 10 years (not including previous surveys) with the currently Moving Object Pipeline baseline criteria; 3 nights with pairs of visits within 15 nights at a range of absolute magnitude $H$ (approximately the size of the object) and then take the completeness at an $H$ value near peak completeness and an $H$ value that is relatively close to 50\% completeness in the baseline; these completeness values are the summary metrics we track across various runs to compare them here. 

The radar plots use the completeness for bright (H=16) NEOs, faint (H=22) NEOs, and bright (H=4) TNOs. 

\subsection{Number of Stars}

We use a simulated MW stellar catalog from Galfast along with the survey coadded depth to estimate the number of stars that would be detected at the 5$\sigma$\ level in $i$. Comparison with the TRILEGAL galaxy model gives similar results.

The number of stars is primarily sensitive to the footprint definition, and decreases dramatically (by about a factor of 2) when there is no coverage of the galactic plane. An extended survey footprint (such as increased coverage toward the north) also increases the number of stars. Because we are only computing the metric in the $i$\ filter, the metric is also sensitive to the depth in $i$, and thus we can see some variation if, e.g., a simulation pushes more $i$\ observations to twilight time.

The radar plots use the total number of stars over the entire footprint down to the coadded limiting magnitude. We do not include a crowding correction.

\subsection{Tidal Disruption Events (TDE)}

\begin{figure}
\epsscale{0.5}
\plotone{plots/tde_lc}
\epsscale{1}
\caption{Simulated TDE lightcurve shapes.}\label{fig:tdelc}
\end{figure}

We use TDE lightcurves from the community to generate a sample of TDE events distributed uniformly on the sky and uniformly over the 10 year survey.  Figure~\ref{fig:tdelc} shows the lightcurve shapes. When analyzing a detected light curve, we test three crtiera
\begin{itemize}
    \item{If it is detected twice pre-peak in any filters}
    \item{If there is one detection pre-peak and at least 3 filters within 10 days of peak}
    \item{If there is one detection pre-peak, one detection in u and any other band near peak, and u plus any other filter post-peak.}
\end{itemize}

When requiring both a color in any filter and $u$ band measurements during the TDE event, this metric is exceedingly sensitive to the number and cadence of $u$ band visits, with the number of detected TDEs scaling linearly with the number of $u$ band visits and preferring visits spread more uniformly over time. In other configurations, when requiring observations pre-peak or just a color in any filters, it is primarily sensitive to the frequency of observations and whether pairs are obtained in the same or mixed filters.

The radar plot uses the TDE some color plus $u$ band metric output.

\subsection{Fast Microlensing}

We use microlensing light curves contributed from the community. For all the events, we assume an $r$=22 magnitude star with a flat SED is being magnified. 

We calculate both a Fast (crossing times of 1-10 days) and Slow (crossing times 100-1,500 days) microlensing metric. They are distributed on the sky proportionally to stellar density squared as measured from TRILEGAL galaxy model.  Due to this spatial distribution, both the Slow and Fast microlensing metrics are primarily sensitive to survey footprint. Footprints without galactic plane coverage cut the number of detected microlenses by approximately 75\% while footprints with heavy galactic plane coverage can increase the number of microlenses by a factor of 2 or more. The Fast microlensing metric is also sensitive to the number and cadence of $u$ band visits, preferring $u$ band visits spread more uniformly over time. 

The radar plot uses the Fast microlensing metric. We find the slow microlensing events are so slow they are detected at a very high rate regardless of survey strategy. 

\subsection{Number of Galaxies}

The estimated expected number of galaxies, across the entire survey footprint, is calculated using \href{https://github.com/LSST-nonproject/sims_maf_contrib/blob/master/mafContrib/LSSObsStrategy/galaxyCountsMetric_extended.py#L26}{GalaxyCountsMetric\_extended}, from \href{https://github.com/LSST-nonproject/sims_maf_contrib}{sims\_maf\_contrib}. The number of galaxies is estimated based on the coadded depth using redshift-bin-specific powerlaws, based on mock catalogs from \citet{2003MNRAS.343..796P}. The overall number of galaxies tends to increase with increased depth, and more so when more of the survey footprint is distributed in lower dust extinction areas. The number of galaxies also increases when the survey filter distribution is redder, rather than bluer.

The radar plots use the total number of galaxies down to the coadded limiting magnitude over the entire survey footprint.

\subsection{DESC WFD Metrics}

The DESC has contributed several metrics evaluating the performance of the WFD for various areas of relevant science. Many of these metrics are built on calculating a subset of the survey footprint that meets the requirements of coverage in all 6 filters, less than a specified level of dust extinction (E(B-V) $<$ 0.2) and greater than a specified coadded depth in $i$ band ($i$ $>$ 25.9 at 10 years), calculated using \href{https://github.com/lsst/sims_maf/blob/master/python/lsst/sims/maf/metrics/weakLensingSystematicsMetric.py#L8}{ExgalM5\_with\_cuts}. This represents the extragalactic science footprint. 

\subsubsection{Static Science}
Over this extragalactic footprint the following metrics are calculated for general `static science'.
\begin{itemize}
\item Median coadded depth in $i$ band
\item Standard deviation of the coadded depth in $i$ band
\item The area of the selected footprint
\item A \href{https://github.com/lsst/sims_maf/blob/master/python/lsst/sims/maf/metrics/summaryMetrics.py#L231}{3x2 point Figure of Merit} emulator
\end{itemize}. 
These metrics are very sensitive to footprint coverage and depth, as well as desiring uniformity in the coadded depth to minimize corrections during later analysis.
The radar plot uses the 3x2point FoM. 

\subsubsection{Weak Lensing}
The same footprint is used to calculate the number of visits per point in the footprint (\href{https://github.com/lsst/sims_maf/blob/master/python/lsst/sims/maf/metrics/weakLensingSystematicsMetric.py#L59}{WeakLensingNvisits}); this is used as an approximate metric evaluating weak lensing systematics.This metric is sensitive to footprint coverage and depth. 
The radar plot uses the mean number of visits across the extragalactic footprint. 

\subsubsection{Large Scale Structure}
The number of galaxies within this same footprint is used as a metric to approximate large scale structure results (DepthLimitedNumGalaxies), using the same GalaxyCountsMetric\_extended as above, but limiting the result to the selected footprint. 

\subsubsection{SNe Ia}

We use SNe Ia light curves from the PLAsTiCC challenge. SNe are distributed uniformly on the sky. 

For each supernova, we check:
\begin{enumerate}
    \item{Is the supernova detected in any filter?}
    \item{Is there a color detected (detected in 2 filters within 0.5 days)?}
    \item{Is it possible to measure the rise slope (detect an increase of 0.3 mags in a filter pre-peak)?}
    \item{Is the light curve "well sampled" (if the light curve duration is divided into tenths, are there detections in 5 unique bins)?}
\end{enumerate}

There are several versions of this metric, using different criteria for observations. We call the supernova ``Detected" if it meets criteria 1, it is ``Pre-peak" if it meets criteria 2 and 3, and is ``Well-sampled" if it meets criteria 4.
The simple Detected metric is sensitive to a combination of survey footprint and number of visits, preferring more area as long as a minimum number of visits spaced uniformly over time are available. The Pre-peak metric, which may be more useful for detections before follow-up, is most sensitive to requiring visits to be obtained with mixed filter pairs, with a lesser preference for visits being spaced more evenly over time (such as in non-rolling cadences). The Well-sampled metric is primarily sensitive to the cadence of visits, preferring visits to be spaced evenly over time. 

The radar plot uses the metric which demands criteria 2 and 3 from above. Thus, we are mostly measuring how well we are producing SNe alerts that can act as triggers for others to follow up. The DESC group has developed metrics for measuring how well SNe are observed by Rubin alone, and we will be incorporating these into MAF soon.


\subsection{Radar Plots}

To help compare multiple science and SRD metrics across runs, we make use of radar plots. In each radar plot, we typically normalize values to a baseline run and plot the fractional change in metric values in the radial direction. For metrics that are measured over the entire sky (e.g., Parallax, Proper Motion, Weak Lensing), we use the median. For the parallax and proper motion metrics, the inverse of the errors are compared. When there are particularly large changes in metrics, we will generate a pair of radar plots with different radial ranges to make comparisons easier.

In some cases we make radar plots of the median coadded depth in each filter. For coadded depth, we plot magnitude difference in the radial direction, with larger values indicating deeper depths. 

Almost all of the metrics in the radar plot show highly statistically significant changes as the survey strategy changes. It is worth remembering that the simulations themselves have some level of uncertainty, as change in the `real-life' weather or status of the observatory will lead to changes in the observing history and then changes in the scheduler choices to maintain the overall survey strategy guidelines. For some metrics with particularly small measured values, these small differences between runs can result in large metric differences, effectively making the metric results somewhat noisy. The TDE metric is one of these such metric; out of 10,000 simulated TDE lightcurves, the baseline run only observes about 200 of these with visits that meet the `some color plus $u$' criteria, thus implying that variations of up to about 7\% in the TDE metric value can be expected even if the simulations are statistically similar.





\section{Individual Scheduler Experiments}

Here we look at various experiments that explore varying a single aspect of the scheduler.


% old rolling
%\subsection{alt\_roll\_dust}

%These are motivated by the DESC collaboration and their desire to observe as many galaxies as possible. The WFD area is extended north and south and the galactic plane (defined by a dust map) is decreased down to $\sim$350 visits. The footprint also includes a northern stripe which gets $\sim220$\ visits. The motivation for the northern stripe is that it can provide image differencing templates for any gravitational wave detections in the north, as well as provide overlap with other survey missions (Euclid, WFIRST).

%These simulations tend to fall just short of the fOArea benchmark (e.g., only $\sim$17,500 square degrees receive 825 visits), but meet the formal SRD requirement by having a median number of 891 visits in the WFD area.  

%The dust exclusion zones make certain parts of the year undersubscribed. This suggests it might be relatively low-impact to add some bridges to the footprint across the galactic bulge and the galactic anti-center.  


\subsubsection{Dust With Alternating}

XXX--add the alt dust plots
\begin{figure}
\caption{}\label{fig:altdust}
\end{figure}
This uses the dusty footprint, and uses a basis function to encourage the scheduler to alternate between the north and south nightly. This was originally done in the altSched \citep{Rothchild19}. This can help keep light curve sampling optimally spaced. By using a basis function, we encourage alternating north/south, but it is not absolutely enforced, making it possible for the scheduler to avoid the moon.

There is no additional NES, however there is a strip in the north observed in $g$, $r$, $i$, and $z$.


XXX--science recap


% old rolling
%\subsubsection{roll\_mod2\_dust\_sdf\_0.20\_v1.5\_10yrs}

%Now we divide the WFD area into quarters, and employ a rolling cadence. For the first 1.5 years and final 2.5 years, the full footprint is used as normal. 

%<fig of first year> <year 1.5-2.5> <2.5-3.5>

%The use of dividing the WFD into quarters will be obvious when we combine with the altSched strategy.


%\subsection{baseline}

%We divide the sky into the standard WFD, NES, SCP and GP. The NES does not get observed in the u and y filters. 

%We run 2 versions of the baseline. One with 2x15s visits and one with 1x30s visits. The extra readtime needed for the 2-snap version results in the open shutter fraction dropping from 77\% to 72\%. Over the 10-year survey, this comes out to about 60 observations fewer for a typical point in the sky. 

%The 2-snap baseline meets the SRD requirements, but has very little extra contingency.

\subsection{Bulge}

We used recommendations from the SAC for different strategies for observing the galactic bulge. These simulations use the Big Sky footprint similar to the Olsen et al white paper.  

We use three footprints for bulge coverage 1) light coverage of the bulge and entire galactic plane, 2) the bulge as deep as WFD and 3) the bulge covered similarly to WFD, but with more observations in $i$.  For each of these strategies, we run a version with natural cadence and one where we boost the priority of the bulge if it has not been observed in 2.5 days. 

\begin{figure}
\epsscale{0.35}
\plotone{plots/pulled_plots/bulges_bs_v1_5_10yrs_Count_observationStartMJD_i_HEAL_SkyMap.pdf}
\plotone{plots/pulled_plots/bulges_bulge_wfd_v1_5_10yrs_Count_observationStartMJD_i_HEAL_SkyMap.pdf}
\plotone{plots/pulled_plots/bulges_i_heavy_v1_5_10yrs_Count_observationStartMJD_i_HEAL_SkyMap.pdf}
\epsscale{1}
\caption{Series of simulations trying different bulge observing strategies.}\label{fig:bulge}
\end{figure}


xxx-science recap


\subsection{DCR}

\begin{figure}
\epsscale{0.5}
\plotone{plots/pulled_plots/dcr_nham2_ugri_v1_5_10yrs_Nvisits_as_function_of_Alt_Az_HEAL_SkyMap.pdf}
\plotone{plots/pulled_plots/dcr_nham2_ugri_v1_5_10yrs_Count_observationStartMJD_HEAL_SkyMap.pdf}
\plotone{plots/pulled_plots/dcr_nham2_ugri_v1_5_10yrs_Hourglass_year_0-1_HOUR_Hourglass.pdf}
\epsscale{1}
\caption{Intentionally taking observations at higher airmass to measure DCR.}
\end{figure}

xxx--example plots for DCR

The LSST will not have an atmospheric chromatic corrector, thus difference imaging can be complicated by differential chromatic refraction (DCR). There is also potential science opportunities by being able to measure the chromatic shift in objects with sharp features in their SEDs.

These experiments look at how we could intentionally schedule a subset of images to be at high airmass so a DCR model could be built up. We test various combinations of filters to demand DCR observations (u+g, u+g+r, and u+g+r+i), and the number of observations to take at high airmass per year (1 or 2). 

Note, even with 2 high airmass observations per year, we would still expect some area of the sky to fall in chip and raft gaps.  It is also worth noting that in our baseline simulation, we observe a spot on the sky in u typically 60 times, or 6 times per year. Taking 2 high airmass observations per year in u decreases the final coadded depth by 0.15 mags.

XXX--science recap

   
\subsection{DDF}

We have run a variety of DDF strategies. Figure~\ref{fig:ddfexamples} shows the same observing season of the DDF ELIASS1 with 5 different strategies. 

\begin{itemize}
    \item{AGN: This strategy takes shorter DDF sequences more often. Only $\sim$2.5\% of visits are spent on DDFs, making the final coadded depths much shallower than other strategies.}
    \item{DESC: a strategy that split the blue and red filters to different days, emphasizing a 3-day cadence}
    \item{Baseline:  }
    \item{Daily: Similar to the baseline, but includes short DDF sequences that can execute daily so there are no long gaps between observations}
\end{itemize}


\begin{figure}
\plottwo{plots/ddf_plots/ddf_m5_AGN.pdf}{plots/ddf_plots/gap_hist_AGN.pdf}
\plottwo{plots/ddf_plots/ddf_m5_Baseline_v1_5.pdf}{plots/ddf_plots/gap_hist_Baseline_v1_5.pdf}
\plottwo{plots/ddf_plots/ddf_m5_Baseline_v1_6.pdf}{plots/ddf_plots/gap_hist_Baseline_v1_6.pdf}
\plottwo{plots/ddf_plots/ddf_m5_DESC.pdf}{plots/ddf_plots/gap_hist_DESC.pdf}
\plottwo{plots/ddf_plots/ddf_m5_Daily.pdf}{plots/ddf_plots/gap_hist_Daily.pdf}
\caption{Plots looking at one observing season of the DDF ELIASS1. }\label{fig:ddfexamples}
\end{figure}

XXX--massive table(s) of DDF, run, filter, coadded depth?

XXX--science discussion.  Point out we desperately need more DDF-specific metrics from the community. 


\subsection{Filter Distribution}

Testing a simple WFD-only footprint, but varying the requested ratio of observations in different filters.


\subsection{Footprints}

\begin{figure}
\epsscale{.25}
\plotone{plots/pulled_plots/footprint_add_mag_cloudsv1_5_10yrs_Count_observationStartMJD_HEAL_SkyMap.pdf}
\plotone{plots/pulled_plots/footprint_big_sky_dustv1_5_10yrs_Count_observationStartMJD_HEAL_SkyMap.pdf}
\plotone{plots/pulled_plots/footprint_big_sky_nouiyv1_5_10yrs_Count_observationStartMJD_HEAL_SkyMap.pdf}
\plotone{plots/pulled_plots/footprint_big_skyv1_5_10yrs_Count_observationStartMJD_HEAL_SkyMap.pdf}
\plotone{plots/pulled_plots/footprint_big_wfdv1_5_10yrs_Count_observationStartMJD_HEAL_SkyMap.pdf}
\plotone{plots/pulled_plots/footprint_bluer_footprintv1_5_10yrs_Count_observationStartMJD_HEAL_SkyMap.pdf}
\plotone{plots/pulled_plots/footprint_gp_smoothv1_5_10yrs_Count_observationStartMJD_HEAL_SkyMap.pdf}
\plotone{plots/pulled_plots/footprint_newAv1_5_10yrs_Count_observationStartMJD_HEAL_SkyMap.pdf}
\plotone{plots/pulled_plots/footprint_newBv1_5_10yrs_Count_observationStartMJD_HEAL_SkyMap.pdf}
\plotone{plots/pulled_plots/footprint_no_gp_northv1_5_10yrs_Count_observationStartMJD_HEAL_SkyMap.pdf}
\plotone{plots/pulled_plots/footprint_standard_goalsv1_5_10yrs_Count_observationStartMJD_HEAL_SkyMap.pdf}
\plotone{plots/pulled_plots/footprint_stuck_rollingv1_5_10yrs_Count_observationStartMJD_HEAL_SkyMap.pdf}
\epsscale{1}
\caption{The different survey footprints simulated.}
\end{figure}

We test a wide variation of possible survey footprints. Some of these are more realistic than others. 



\subsection{goodseeing}\label{ss:goodseeing}

These test the ability to ensure the entire WFD area is imaged in ``good seeing" conditions every year, here defined as FWHM of 0.7 arcseconds or better.  

These runs work well and it seems to add no particular overhead to the observing. It might make it more challenging to implement in operations, simply because the baseline simulation can simulate an entire night and pass off the list to be observed. If we want to run with the goal of collecting good seeing images, we will need to update the observing queue every time the seeing conditions change significantly, which could result in changing the upcomming observations more often than is desired.

% old rolling
%\subsection{rolling}
%We test breaking the WFD region up into 2, 3, and 6 declination bands which are then ``rolled". 

\subsection{Short Exposures}

We try taking additional short exposures (1s or 5s) twice or five times per year. Taking shorter exposures is a less efficient observing mode, but it seems to have little impact on the overall open shutter fraction.

\subsection{Spiders}

We look at keeping diffraction spikes aligned along CCD rows and columns. This may result in the camera rotator angle being much less randomized than our baseline rotational dithering strategy.

xxx--Science recap: Should be almost identical to baseline. 

\subsection{Third Observation}

For early identification of transients, it can be helpful to have more than two observations in a night. In these observations, we dedicate between 15 and 120 minutes at the end of the night to attempting to observe areas of sky that already have been observed.

\subsection{Twilight NEO Survey}

This is an implementation of white paper XXX, where we use twilight time to take short exposures along the ecliptic to search for NEOs. 

If we dedicate all twilight time to NEO searches, we fail to meet the SRD requirements. Thus we also check running the NEO survey every 2, 3, or 4 days.

\subsection{u60}\label{ss:u60}
The u-band observations can often be readnoise limited. We test doubling the u-band exposure time and cutting the number of exposures in half. This results in the u-band final coadded depth reaching $\sim$0.20 mags deeper. The g-band is also 0.10 mags deeper, with the rest of the filters essentially unchanged in final depth.

Note, we assume that 1x60s visit counts as 2 30s visits for the purpose of meeting the SRD value of 825 visits in the WFD area. Adopting longer exposures in u seems like a good idea, but the SRD will probably need to be modified to ensure it is not ambiguous.

\subsection{Variable Exposure Times}


\begin{figure}
\plottwo{plots/variable_expt_plots/baseline_spot.pdf}{plots/variable_expt_plots/varexpt_spot.pdf}
\caption{XXX-variable exposure time}\label{fig:varexptime}
\end{figure}

A test where we vary the exposure time based on the current conditions so individual exposures have similar depths. There is an argument that taking a full 30s visit in ideal dark time conditions results in ``wasted depth", as more objects and transients will be detected, but then it will be impossible to identify them as later visits are unlikely to be as deep. Similarly, taking a 30s visit in poor conditions will result in a shallow image which will be of limited use.

As with doing 60s u band exposures , this may require modifying the detailed specifics of the SRD as longer exposures may need to count as multiple visits.

Having variable exposure time introduces at least 6 new free parameters to the scheduler (the target individual depth for each filter), as well as the shortest and longest acceptable exposure times.  As with \ref{ss:goodseeing}, this would be more complicated to run in operations as the scheduler would need current conditions to calculate the modified exposure times, although the predicted sky brightness may be accurate enough.



\subsection{WFD Depth}

We vary what fraction of the observing time is dedicated to the WFD area, from 60\% to 99\% with and without the standard DDF surveys. Unsurprisingly, the SRD is not met if the WFD is only given 60\%.



\subsection{Rolling Cadences}

XXX--show off the different rolling runs from 1.6

\subsection{Even Filters}

Looking at varying how aggressive we are with matching filter choice to sky brightness.  

XXX--should show a hit on the coadded depth. Maybe some improvement in SNe

\begin{figure}
\plottwo{plots/pulled_plots/even_filters_g_v1_6_10yrs_Hourglass_year_0-1_HOUR_Hourglass.pdf}{plots/pulled_plots/even_filtersv1_6_10yrs_Hourglass_year_0-1_HOUR_Hourglass.pdf}
\caption{The filter distribution for the even filter simulations. Unlike the baseline simulations, more filters are observed in bright time.}\label{fig:even_filt_hourglass}
\end{figure}

\subsection{Aliasing}

Make a note that we looked at if we are aliased. It's not as bad as the old runs, so we probably don't need to do anything extra to avoid aliasing.

XXX--maybe add some plots from the aliasing notebook I made a while back.

\begin{figure}
\plotone{plots/alias_plots/aliasing.pdf}
\caption{Aliasing at a sample position in a baseline simulation. There are peaks at harmonics of 24 hours, but this is inevitable with a ground-based telescope. The aliasing seems much lower than earlier version of OpSim.}
\end{figure}


\section{FBS release v1.6}

Here we describe the runs done as part of the FBS 1.6 release.  Unlike previous releases, here we do a select few releases that combine various aspects of previous experiments.


%#################### Baseline #############################
\subsection{Baseline}

\begin{figure}
\epsscale{.5}
\plotone{plots/pulled_plots/baseline_nexp1_v1_6_10yrs_Count_observationStartMJD_HEAL_SkyMap.pdf}
\plotone{plots/pulled_plots/baseline_nexp1_v1_6_10yrs_Nvisits_as_function_of_Alt_Az_HEAL_SkyMap.pdf}
\plotone{plots/pulled_plots/baseline_nexp1_v1_6_10yrs_Hourglass_year_0-1_HOUR_Hourglass.pdf}
\epsscale{1}
%XXX ra dec map, alt az, an hourglass
\caption{The baseline v1.6 simulation. The top panels show the distribution of visits (all filters) in RA/dec and Alt/Az. The bottom panel shows the first year of observations color-coded by what filter was loaded. White regions represent scheduled and unscheduled downtime as well as weather downtime. The black curve on the bottom shows the moon phase.}\label{fig:baseline1.6}
\end{figure}


For the baseline strategy, we set the footprint to have 18,000 square degrees dedicated the the WFD survey. The WFD has a filter distribution of u:g:r:i:z:y of 0.31:0.44:1.0:1.0:0.9:0.9.
% WFD sum = 4.55

We include coverage of the Galactic Plane (GP) and South Celestial Pole (SCP). These areas are set to have 20\% the number of counts of the WFD (if a spot in the WFD has 900 visits, points in the GP and SCP will have 180 visits). The GP and SCP are set to have equal number of visits in all filters.

The total breakdown of target observing time is 85\% for WFD, 6\% for the NES, 6\% for the GP and NES, and 5\% for DDFs.

The North Ecliptic Spur (NES) is observed with only the g, r, i, and z filters. The NES area is set to have one-third the number of visits of the WFD.  The filter distribution is set to g:r:i:z of 0.2:0.46:0.46:0.4. 

While the different survey areas are covered to different depths, the baseline scheduler treats them identically and only tries to maintain the proper ratios of area coverage. This means blocks of observations can be scheduled that cover the different regions seamlessly. It also means we have no additional constraints on how the regions are observed. For example, we currently do not reserve ``good seeing" time for the WFD area. 

The baseline survey includes the 4 announced Deep Drilling Fields as well as a pair of fields that overlap the Euclid Deep Field South. (XXX--maybe put in a table?). Each individual DDF is set to take a maximum of 1\% of the total visits (the Euclid fields are set to take 1\% combined). The DDF sequence is ux8, gx20, rx10, ix20, zx26, and yx20, all with 30s exposures. For any given sequence, only the five filters loaded in the camera are executed. By default, we remove the u filter when the moon is more than 40\% illuminated at the start of the night.


We run 2 baseline simulations, one with 1x30s visits and one with 2x15s visits.  The main difference is the additional readout time in the 2x15 version drops the open shutter fraction from 77\% to 72\%. This puts the 2x15s simulation close to failing the SRD FO metric, with some parts of the WFD region only reaching 824 observations (the median is still 892). 

For the rest of the simulations in v1.6 we use 1x30s visits.  If 2x15s is required there will be a significant drop in the number of visits, and areas outside of the WFD may need to be scaled back to still meet SRD requirements.

The baseline surveys use the following strategies:

XXX-DDF strategy:


XXX--main non-twilight time strategy: 

Observations are taken in 44 minute blocks (22 minutes for an initial area, 22 minutes to repeat the area). The size of the blocks can scale slightly to try and fill time before twilight (e.g., it will expand to a pair gap of 25 minutes if there are 50 minutes until twilight). All observations are taken in pairs, with potential combinations of u+g, u+r, g+r, r+i, i+z, z+y, or y+y. The order of observations can change depending on what filter is currently loaded (e.g., if the scheduler decides to observe a g+r sequence, the r observations will be taken first to eliminate a filter change if possible.)

The camera rotator angle (relative to the telescope) is randomly set each night between -80 and 80 degrees.  XXX-the angle is set when the block is scheduled, so there can be a few degrees of drift between when the rotator angle is computed and when the observation is actually taken.

The basis functions used are:

The 5-sigma depth (for both filters in the pair being taken), the footprint uniformity (again, in both filters), the slewtime, and a basis function that rewards staying in the current filter.  We also include a basis function that rewards taking 3 observations per year per filter over the entire survey footprint.  

The zenith is masked (to avoid long azimuth slews), and a region 30 degrees around the moon is masked. The bright planets (Venus, Mars, and Jupiter) are masked with a 3.5 degree radius. 

XXX--twilight strategy
If the sun is higher than XXX degrees, or there is not enough time remaining to take observations in pairs, 

%#################### DDF Heavy #############################
\subsection{DDF Heavy}

\begin{figure}
\epsscale{0.5}
\plotone{plots/pulled_plots/ddf_heavy_v1_6_10yrs_Count_observationStartMJD_HEAL_SkyMap}
\plotone{plots/pulled_plots/ddf_heavy_v1_6_10yrs_Nvisits_as_function_of_Alt_Az_HEAL_SkyMap}
\plotone{plots/pulled_plots/ddf_heavy_v1_6_10yrs_Hourglass_year_0-1_HOUR_Hourglass}
\epsscale{1}
\caption{DDF Heavy simulation. Nearly identical to the baseline, but giving as much time as possible to DDF observations.}\label{fig:ddfheavy}
\end{figure}


This run is nearly identical to the baseline, but gives a large fraction of time to the deep drilling fields. Each of the five DDFs takes between 2.4 and 2.9\% of the survey, with 13.4\% of all visits being used for DDF observations. The baseline has 4.6\% of visits used for DDFs.  This is enough time that the WFD area near the DDFs fails to reach 825 visits over 10 years, but the SRD requirement is formally still met because the median WFD point is observed 875 times.

XXX--For each of these 1.6 runs, maybe an include a science impact recap? Maybe a table to compare median coadded depth in each filter, a radar relative to baseline (and same scale across all of them), and a ew lines of explination of what we think happened, what metrics we need (e.g., here we could say we need AGN and other DDF relevant metrics).

%#################### Barebones #############################
\subsection{Barebones}

\begin{figure}
\epsscale{0.5}
\plotone{plots/pulled_plots/barebones_v1_6_10yrs_Count_observationStartMJD_HEAL_SkyMap}
\plotone{plots/pulled_plots/barebones_v1_6_10yrs_Nvisits_as_function_of_Alt_Az_HEAL_SkyMap}
\plotone{plots/pulled_plots/barebones_v1_6_10yrs_Hourglass_year_0-1_HOUR_Hourglass}
\epsscale{1}
\caption{The barebones simulation essentially covering just the WFD area as efficiently and deeply as possible.}\label{fig:barebones}
\end{figure}


The barebones simulation is an example of a survey where we focus exclusively on meeting the SRD requirements, with little optimization for science.

The survey footprint is restricted to the standard 18,000 square degree WFD area only. Deep drilling fields are included, but capped at $\sim2.5$\% of the total visits. Visits in u and y are unpaired, while the rest of the filters are paired in the same filter. This results in very few filter changes in a night. 

There are a wide number of reasons why this would be a terrible survey strategy--detected transients would have no color information, photometric uber-calibration would be difficult with the galactic plane gap, a lack of solar system object because the NES is not included, etc.  The main purpose is to show the scheduler can reach very near the theoretical maximum for open shutter fraction, with this run reaching 80\%. Also, we can note the fONv metric reaches 1,148 which is 40\% higher than the SRD requirement of 825. This also implies that we can observe a maximum of $\sim115$\ WFD visits per year in the event we want to adjust the scheduler to attempt to catch up on the WFD progress. 



%#################### Data Management Heavy #############################
\subsection{Data Management Heavy}

\begin{figure}
\epsscale{0.5}
\plotone{plots/pulled_plots/dm_heavy_v1_6_10yrs_Count_observationStartMJD_HEAL_SkyMap}
\plotone{plots/pulled_plots/dm_heavy_v1_6_10yrs_Nvisits_as_function_of_Alt_Az_HEAL_SkyMap}
\plotone{plots/pulled_plots/dm_heavy_v1_6_10yrs_Hourglass_year_0-1_HOUR_Hourglass}
\epsscale{1}
\caption{The DM heavy simulation. Similar to the baseline, but the alt/az plot shows how some observations are being taken at high airmass to support DCR modeling.}\label{fig:dmheavy}
\end{figure}


This is simulations includes various modifications that may be helpful for Data Management purposes. For the WFD region in u, g, and r a few images per year are taken at high airmass so that DCR correction models can be made.

The camera rotator angle is set so that diffraction spikes fall along CCD rows and columns. This helps with difference imaging so the maximum possible area can be used, but may result in weak lensing systematics.

Each year, the scheduler prioritizes taking g,r,i images of the whole sky in good seeing conditions (defined as 0.7\arcsec effective FWHM or better).

The DDF fields use larger dithers, up to 1.5 degrees, compared to the default 0.7 degree maximum.


%#################### Rolling Extragalactic #############################
\subsection{Rolling Extragalactic}

\begin{figure}
\epsscale{0.5}
\plotone{plots/pulled_plots/rolling_exgal_mod2_dust_sdf_0_80_v1_6_10yrs_Count_observationStartMJD_HEAL_SkyMap}
\plotone{plots/pulled_plots/rolling_exgal_mod2_dust_sdf_0_80_v1_6_10yrs_Nvisits_as_function_of_Alt_Az_HEAL_SkyMap}
\plotone{plots/pulled_plots/rolling_exgal_mod2_dust_sdf_0_80_v1_6_10yrs_Hourglass_year_0-1_HOUR_Hourglass}
\epsscale{1}
\caption{The Rolling Exgal simulation. }\label{fig:rollingexgal}
\end{figure}


\begin{figure}
\plottwo{plots/rolling_plot/baseline_nexp1_v1_6_Count_filter_note_not_like_DD_HEAL_SkyMap.pdf}{plots/rolling_plot/rolling_exgal_mod2_dust_sdf_0_80_v1_6_Count_filter_note_not_like_DD_HEAL_SkyMap.pdf}
\plottwo{plots/rolling_plot/baseline_nexp1_v1_6_Count_filter_night_gt_1278_375000_and_night_lt_1643_625000_and_note_not_like_DD_HEAL_SkyMap.pdf}{plots/rolling_plot/rolling_exgal_mod2_dust_sdf_0_80_v1_6_Count_filter_night_gt_1278_375000_and_night_lt_1643_625000_and_note_not_like_DD_HEAL_SkyMap.pdf}
\caption{Illustration of how rolling cadence works. The top panels show the number of observations after 10 years (all filters) for the baseline and rolling exgal simulations (excluding DDF observations). Both simulations have very smooth WFD coverage, with $\sim$900 observations.  The lower panels show the number of observations taken between 3.5 and 4.5 years into the survey.  The baseline WFD remains smooth, while the rolling exgal simulation has declination stripes of high and low counts.  }\label{fig:exgalroll}
\end{figure}


The rolling extragalactic is motivated by cosmological drivers. The footprint is modified so the 18,000 square degrees of the WFD are placed in low-extinction regions. The simulation also executes a half-sky rolling scheme, which should result in well sampled lightcurves for extragalactic transients.

xxx--add a plot showing the lines of how rolling works.  

xxx--mention that doing rolling in quarters ensures that half the alert stream is always available to northern telescopes for follow up. 


%#################### Milky Way Heavy #############################
\subsection{Milky Way Heavy}
\begin{figure}
\epsscale{0.5}
\plotone{plots/pulled_plots/mw_heavy_v1_6_10yrs_Count_observationStartMJD_HEAL_SkyMap}
\plotone{plots/pulled_plots/mw_heavy_v1_6_10yrs_Nvisits_as_function_of_Alt_Az_HEAL_SkyMap}
\plotone{plots/pulled_plots/mw_heavy_v1_6_10yrs_Hourglass_year_0-1_HOUR_Hourglass}
\epsscale{1}
\caption{The Milky Way heavy simulation. }\label{fig:mwheavy}
\end{figure}

The Milky Way heavy simulation covers the Galactic bulge, LMC, and SMC as part of the WFD area.  


%#################### Solar System Heavy #############################
\subsection{Solar System Heavy}
\begin{figure}
\epsscale{0.5}
\plotone{plots/pulled_plots/ss_heavy_v1_6_10yrs_Count_observationStartMJD_HEAL_SkyMap}
\plotone{plots/pulled_plots/ss_heavy_v1_6_10yrs_Nvisits_as_function_of_Alt_Az_HEAL_SkyMap}
\plotone{plots/pulled_plots/ss_heavy_v1_6_10yrs_Hourglass_year_0-1_HOUR_Hourglass}
\epsscale{1}
\caption{The Solar System heavy simulation. }\label{fig:ssheavy}
\end{figure}

Include ecliptic coverage through the galactic plane

NEO survey at twilight--Note, a NEO survey taking short exposures will drastically increase the data throughput. DM needs to check if this would be feasible.  We also need to check with the camera team that taking short exposures for an extended time will not be a thermal issue.

only i,z,y in twilight, making sure we observe more r-band in non-twilight and in pairs.

include r+r pairs

For regular (30s visit) twilight observations, avoid the ecliptic (ensuring they are always taken in pairs in non-twilight time)

%#################### Combo Dust #############################
\subsection{Combo Dust}

\begin{figure}
\epsscale{0.5}
\plotone{plots/pulled_plots/combo_dust_v1_6_10yrs_Count_observationStartMJD_HEAL_SkyMap}
\plotone{plots/pulled_plots/combo_dust_v1_6_10yrs_Nvisits_as_function_of_Alt_Az_HEAL_SkyMap}
\plotone{plots/pulled_plots/combo_dust_v1_6_10yrs_Hourglass_year_0-1_HOUR_Hourglass}
\epsscale{1}
\caption{The Combo Dust simulation. }\label{fig:combodust}
\end{figure}


This simulation attempts to improve several science cases compared to the baseline simultaneously. The footprint used here starts with defining the WFD area as 18,000 square degrees with low extinction. Then an additional 2,000 square degrees are added to WFD to cover the bulge, the ecliptic through the galactic plane, the LMC and SMC, and an outer disk region. Dusty areas of the sky and the South Celestial Pole are covered at about one-quarter the WFD depth. The NES is covered in $g$, $r$, $i$, and $z$. The footprint also includes very light coverage to the northern limit of the telescope so there can be templates for ToO events on the entire visible sky. 

The footprint has 35 free parameters for setting region locations and filter ratios. Many of these have have been set by eye or use historical values of questionable providence. Moving forward with such a footprint 

XXX--this one is also rolling.

% Let's make a table comparing everything. Maybe one for the basics, then one for science?



\section{Individual Visit Length}
What to do - 1x30s vs. 2x15s? 1x30s much more efficient (show rough calculation of overhead) than 2x15s, but may have drawbacks due to cosmic ray rejection and potential to miss very rapid transients (or WD detection .. ref white paper). Subtle drawback that 2x15s gives the same "midpoint exposure time" across FOV, 1x30s does not. 

Show difference in 1x30s vs. 2x15s in whatever is our 'standard baseline' at this point. 

There has been thought of using a variety of exposure times if we use two snaps (e.g., 5s + 25s). Because there are not plans to release catalogs from individual snaps, it's not clear if this would enable much new science.

Show effect of 7\% loss in efficiency when attempting to combine minisurveys in various configurations (assume we will find some combinations possible with single exposure visits that are impossible with two snaps). 

Also possible to use variable exposure time depending on seeing and sky brightness conditions. Shorter exposures in good conditions keeps us from observing ``wasted" depth, letting us take longer exposures in poor conditions. This does introduce a host of new free parameters (an ideal target depth for each filter and minimum and maximum exposure times).  This would might require rewording the SRD to ensure, e.g., that 20s visits in good conditions count for the number of visit requirement.

Relevant metrics: total number of visits, number of visits per field/filter

\section{Intra-night Cadence}
What to do for visit sequence within a night? White paper support for multiple filters within a night (except TNOs maybe?). Potential drawbacks - less efficient (show effect on efficiency). This applies to WFD primarily, but we've applied to any survey that did not have their own specifications (so, everywhere). 

Extension of pairs to $u$ band and $y$ band (show effect). 

Relevant metrics: inter-night visit gaps and SN discovery, SSO discovery/characterization, transient and variable discovery (??), number of visits

\section{Wide-Fast-Deep Footprint}
What to do for WFD footprint? SRD not specific, DESC want low-extinction sky (and depth), but WFD is generally the area of sky that receives the most visits, so generally other science will also benefit from more visits to their relevant areas (particularly galactic plane .. for time-domain studies primarily, not depth)

Relevant metrics: area of sky with 825 visits (under particular restrictions, like total coadded depth and individual image seeing and dust extinction), number of galaxies, number of resolved galaxies, SSO discovery, transient and variable star discovery, astrometry in the galactic plane (?)

\section{Dithering}

The spatial dithering strategy for the large area surveys seems to work well.

We need to decide on a rotational dithering strategy. Should we try to randomize, or keep diffraction spikes aligned along rows and columns?

The dithering strategy for the DDFs pose a tension. DDF science is best served by small spatial dithers, but DM would prefer large spatial dithers. Also need to do something unique for the Euclid pair of fields, and any fields with nearby bright stars.


\section{Rolling cadence}
Motivation for a rolling cadence (more frequent visits in some years)

Different options for rolling and explanation of how implemented

Should really include discussion of recovery from bad weather years and simulation of same

Relevant metrics: Maintain astrometry requirements, SN discovery, SSO discovery and characterization,  Transient and variable discovery, uniformity of coadded depth / number of visits, 

\section{Northern minisurveys}
Add extension to cover Euclid/DESI with various numbers of visits

Observing NES 

Effect of adding or removing these minisurveys

Relevant metrics: SSO discovery and characterization (particularly active asteroids), depth and number of visits through remainder of North

\section{Southern minisurveys}
Add extension over south celestial pole, LMC/SMC with various numbers of visits

Effect of adding or removing these minisurveys

Relevant metrics: number of visits and coadded depth over SCP, discovery of variables in LMC/SMC (see Olsen white paper for metrics?)

\section{Low Galactic Latitudes}
Discussion of definitions from SAC and recommendations for visits

Effect of adding or removing these minisurveys

Relevant metrics: number of visits, astrometry in bulge, discovery of variables/transients/microlensing in bulge (?)

\section{Twilight Observing}
Discuss need for twilight observing to meet SRD goals (weather, total amount of time available)

Add NEO twilight survey, add DCR white paper (season extension visits?)

Effect of adding or removing these minisurveys

Relevant metrics: NEO discovery, number of visits and coadded depth (and uniformity) in WFD, measurement of DCR, season length

\section{Deep Drilling Fields}
Discuss purpose and how these are scheduled (very different from other fields)

Discuss potential cadences (AGN/ DESC) and how these differ, and our combination of the two

Discuss timing issues with oversubscription (and how much of a problem this could be, what if worse weather?) -- include location of fifth DD field

Effect of adding or removing these minisurveys

Relevant metrics: number of visits and coadded depth for DD, SN detection in DDFs, AGN detection in DDFs
*[solar system minisurvey DDF?]

\section{ToO modes}
Discuss impact of ToO, and how we could implement ToOs in scheduler (various modes: straight to queue by hand or set up known program and supply trigger, etc. -- that we're evaluating the second?)

Any ToO survey should also take into account that chip and raft gaps mean full sky coverage will require multiple images with spatial dithering.

Discuss how we can have a low coverage region to the north to maintain templates for all possible ToOs, or we could decide ot only search for ToOs that are likely to be in the WFD area.

Relevant metrics: frequency of achieving ToO observations, number of visits and coadded depth in other surveys (WFD or other minisurveys that may be in particular contention)

\section{Making it all work}
Discuss combinations of the above that work together or don't 

Relevant metrics: all

\section{Optimizing parameters}
Somewhere in here we probably ought to talk about optimizing the parameters for each run, and doing bigger sweeps across parameter space. That would easily expand each of the above options by many factors.

\section{Conclusions}
Hopefully here we pare down the evaluation of 100s of runs (like promised) to a set of between 10 to 20 (if this is possible, after combining along different axes). 
The results should come with some basic comments about what's particularly good or bad in each of these areas and how we arrived at these general options. 


% Make sure lsst-texmf/bin/generateAcronyms.py is in your path
\section{Acronyms} \label{sec:acronyms}
\input{acronyms.tex}
