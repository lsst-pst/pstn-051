% replace large sections with \include files 

\section{Introduction}

Note: This paper needs to focus on survey strategies and their evaluation. 

Introduction - cover basic idea of survey simulator, scheduler and weather/telescope models. 

Cover basic survey strategy starting point - wide area, frequent coverage, ten year timespan - and why. 

Mention COSEP and call for white papers - idea is to do the best science we can, add last 10\% "best" science. 

Earlier attempts at simulating LSST in \citet{Rothchild19} and \citet{Naghib19}.

\section{Survey Simulator Overview}
Probably need some reference to what survey scheduler was used / how it was set up for various runs, how the runs were performed, and what the input weather and telescope models were like. 

\section{Basic Survey Requirements}
Basic survey strategy starting point and why - in more depth? Discuss metrics related to these requirements. 

Probably should show that all survey strategies evaluated do / need to meet these requirements (but maybe later?)

\section{Feedback from white papers and SAC} 
Broad outline of points to evaluate for survey strategy, and our approach in running the subsequent experiments (this should help make sense of what comes next)

Discuss basic types of SAC recommendations. 

%\section{Overview of Metrics}
\section{Survey Requirements and Metrics}

There are many, many options for evaluating the output of the survey strategy experiments. One of the primary goals for the LSST Metrics Analysis Framework (MAF) package was to make it easier for both the project and community members to write metrics to evaluate these outputs. This effort has had some significant successes; SRD-level metrics have been written that cover the primary requirements for the SRD, the DESC working groups have made good progress in writing metrics for their evaluation of the simulations, and the Solar System collaboration has contributed substantial metrics. In other areas, it has been more difficult for the community to engage and contribute directly to MAF; for some of these areas, we have been able to help get metrics running, but clearly there are areas which are lacking definitive metrics. Many of the areas which are lacking relate directly to time domain studies, a critical area for the LSST. We acknowledge this problem and encourage further work by the community, particularly over the next year. 

Here we make a brief summary of some of the top-level science-related metrics. There are thousands of metrics which are run as part of standard MAF analysis; for broad comparisons between simulations we pick a very limited subset of these metrics intended to discover or highlight differences between the simulation survey strategies or to cover major areas of science. 

\subsection{SRD Metrics}

The SRD metrics are designed to cover the primary science requirements laid out in the SRD; the most relevant of these relate to the number of visits per pointing across the WFD region, the area of the WFD region, the parallax and proper motion errors and the number of rapid revisits (on timescales between a few to 40 seconds) per point on the sky. While we check all of these metrics for all runs, the most sensitive to changes in the survey strategy is the number of visits across the WFD, tracked in the fO metric, since we are often attempting to distribute visits into other parts of the sky for other science. 

The fO metric calculates the total number of visits per point on the sky, then calculates how much area is covered with how many visits. This can be summarized across a two axes; the amount of area that receives at least 825 visits per pointing (`fO Area') or the median (or minimum) number of visits that the most frequently visits 18,000 square degrees receives (`fONv MedianNvisits' or `fONv MinNvisits'). The first version, fO Area, tends to be somewhat unstable; the survey hardly ever observes more than18k sq deg to at least 825 visits, because we don't program in larger WFD areas, but if the number of visits across the WFD area falls below 825, the resulting fO Area value will fall rapidly (because we cover the sky uniformly). While fO Area is useful to check, a more useful number is fONv MedianNvisits or MinNvisits. The value of fONv MedianNvisits tells us how many visits the typical field in the top 18k sq degrees receives; fONv MinimumNvisits tells us the fewest number of visits any of those top 18k sq deg received. Typically we see fONv MedianNvisits scales more smoothly with the fraction of visits devoted to WFD and likely represents science metrics that depend on having a reasonably large amount of visits over the entire WFD well. 

The radar plots use fONv MedianNvisits, the Median Parallax Error at XXX, and the Median Proper Motion Error at XXX.

\subsection{Solar System Science Metrics}

Solar System science metrics include discovery metrics (with various discovery criteria, such as detections in 3 nights with pairs of visits within a 15 night window) and characterization metrics (ie. how many colors for objects can we measure, and can we determine a light curve or even shape measurement from the lightcurve), contributed by both project and science collaboration. The most important metric for solar system objects is discovery; finding the objects is the first priority. Characterization metrics are secondary metrics. For each of these metrics, we generate input observations using an appropriate solar system population: Potentially Hazardous Asteroids (PHAs) and Near Earth Objects (NEOs) based on a model by \citet{2018Icar..312..181G}, Main Belt Asteroids (MBAs) and Jovian Trojans based on the S3M model from \citet{2011PASP..123..423G}, and TransNeptunian Objects (TNOs) based on the L7 model from the Canda France Ecliptic Plane Survey (CFEPS) \citep{2009AJ....137.4917K, 2011AJ....142..131P}. These populations move at varying rates and cover varying amounts of the sky. NEOs move over much of the sky during the lifetime of the survey, so are less sensitive to footprint variations, but tend to have much more strongly varying brightnesses, thus are sensitive to the number and timing of visits (must be observed when they are bright). TNOs move very slowly, not more than a few fields of view over the lifetime of the survey, so are quite sensitive to footprint, however they are relatively consistent in their brightness; thus they are less sensitive to the overall number of visits at a particular point in the sky, once a threshold has been met. 

For each of these populations, we calculate the population completeness due to discovery with the LSST at the end of 10 years (not including previous surveys) with the currently Moving Object Pipeline baseline criteria; 3 nights with pairs of visits within 15 nights at a range of absolute magnitude $H$ (approximately the size of the object) and then take the completeness at an $H$ value near peak completeness and an $H$ value that is relatively close to 50\% completeness in the baseline; these completeness values are the summary metrics we track across various runs to compare them here. 

The radar plots use the completeness for bright NEOs, faint NEOs, and XXX TNOs. 

\subsection{Number of Galaxies}

The estimated expected number of galaxies, across the entire survey footprint, is calculated using \href{https://github.com/LSST-nonproject/sims_maf_contrib/blob/master/mafContrib/LSSObsStrategy/galaxyCountsMetric_extended.py#L26}{GalaxyCountsMetric\_extended}, from \href{https://github.com/LSST-nonproject/sims_maf_contrib}{sims\_maf\_contrib}. The number of galaxies is estimated based on the coadded depth using redshift-bin-specific powerlaws, based on mock catalogs from \citet{2003MNRAS.343..796P}. 

The radar plots use the total number of galaxies down to the coadded limiting magnitude over the entire survey footprint.

\subsection{Number of Stars}

XXX

The radar plots use the total number of stars over the entire footprint down to the coadded limiting magnitude. (crowding?)

\subsection{DESC WFD Metrics}

The DESC has contributed several metrics evaluating the performance of the WFD for various areas of relevant science. Many of these metrics are built on calculating a subset of the survey footprint that meets the requirements of coverage in all 6 filters, less than a specified level of dust extinction (E(B-V) $<$ 0.2) and greater than a specified coadded depth in $i$ band ($i$ $>$ 25.9 at 10 years), calculated using \href{https://github.com/lsst/sims_maf/blob/master/python/lsst/sims/maf/metrics/weakLensingSystematicsMetric.py#L8}{ExgalM5\_with\_cuts}. This represents the extragalactic science footprint. 

\subsubsection{Static Science}
Over this extragalactic footprint the following metrics are calculated for general `static science'.
\begin{itemize}
\item Median coadded depth in $i$ band
\item Standard deviation of the coadded depth in $i$ band
\item The area of the selected footprint
\item A 3x2 point Figure of Merit emulator
\end{itemize}. 
The radar plot uses the 3x2point FoM. 

\subsubsection{Weak Lensing}
The same footprint is used to calculate the number of visits per point in the footprint (WeakLensingNvisits); this is used as an approximate metric evaluating weak lensing systematics.
The radar plot uses the mean number of visits across the extragalactic footprint. 

\subsubsection{Large Scale Structure}
The number of galaxies within this same footprint is used to as a metric to approximate large scale structure results (DepthLimitedNumGalaxies), using the same GalaxyCountsMetric\_extended as above, but limiting the result to the selected footprint. 

\subsubsection{SNe Ia}

XXX

\subsection{Tidal Disruption Events (TDE)}

XXX

\subsection{Fast Microlensing}

Light curves contributed from the community.

While we also have a slow microlensing metric, we find very little variation over different siumulations.

Because the baseline footprint has sparse coverage of the Galactic bulge, the baseline value of this metric is relatively small, making the normalized values more volitile than the other metrics here.


\subsection{Radar Plots}

Explain that the radar plots have values normalized (typically to a relevant baseline run). For the parallax and proper motion metrics, the inverse of the errors are compared. For magnitudes, we plot magnitude difference (with larger values indicating deeper depths).

XXX--Maybe put in a table of the raw values for the 1.5 and 1.6 baseline runs. (appendix??)



\section{Individual Scheduler Experiments}

Here we look at various experiments that explore varying a single aspect of the scheduler. 


%############# varying u-band #############
\subsection{$u$\ Filter Pairing}\label{ss:ufilt}

\begin{figure}
\plotone{plots/radar_plots/upairs30_radar}
\plotone{plots/radar_plots/upairs60_radar}
\caption{Varying when the $u$\ filter is swapped out of the camera as well as adding additional weight to the $u$\ footprint.}
\end{figure}

For this family of simulations, we only take $u$\ observations paired $\sim$22 minutes later with $g$\ or with $r$. We vary when the $u$\ filter is loaded into the camera (at lunar illuminations of 15, 30, 40, and 60\%) and vary the strength of the $u$\ footprint (1, 2, or 4 times the baseline footprint).

There is a tension between TDEs and most other science cases, with TDEs benefitting from more $u$\ visits and other science cases staying the same or dropping with increased $u$.


%############# Filter Loading #############
\subsection{Filter Loading}

\begin{figure}
\epsscale{0.5}
\plotone{plots/radar_plots/filter_load_radar}
\epsscale{1}
\caption{Varying when the $u$\ filter is loaded.}
\end{figure}

Similar to the experiment in \S\ref{ss:ufilt}, only $u$\ visits can be paired with $u$, $g$, or $r$. We vary when the $u$ filter is loaded into the camera (lunar illumination of 5, 10, 15, 20, 30, 45, or 60\%). 

As before, the TDE metric seems to be the most sensitive to the $u$\ observing strategy.


%############ Dust With Alternating ############
\subsection{Dust With Alternating}

\begin{figure}
\epsscale{0.5}
\plotone{plots/pulled_plots/alt_roll_mod2_dust_sdf_0_20_v1_5_10yrs_Count_observationStartMJD_HEAL_SkyMap}
\plotone{plots/pulled_plots/alt_roll_mod2_dust_sdf_0_20_v1_5_10yrs_Nvisits_as_function_of_Alt_Az_HEAL_SkyMap}
\plotone{plots/pulled_plots/alt_roll_mod2_dust_sdf_0_20_v1_5_10yrs_Hourglass_year_0-1_HOUR_Hourglass}
\epsscale{1}
\caption{The alt\_roll\_dust simulation that uses a footprint to avoid high extinction and tries to drive an every-other-day cadence.}\label{fig:altdust}
\end{figure}

This uses the dusty footprint and a basis function to encourage the scheduler to alternate between the north and south nightly. This is similar to what was originally done in the altSched simulations \citep{Rothchild19}. This can help keep light curve sampling optimally spaced. By using a basis function, we encourage alternating north/south, but it is not absolutely enforced, making it possible for the scheduler to avoid the moon. Note we have improved the rolling cadence implementation to eliminate the over-exposed stripes and high airmass observations.

There is no additional NES, however there is a strip in the north observed in $g$, $r$, $i$, and $z$.

The science impact of this strategy is fairly minimal. By avoiding extinction regions, we have more stars and galaxies. The coverage of the LMC also increases the number of fast microlensing events. 

\begin{figure}
\epsscale{0.65}
\plotone{plots/radar_plots/alt_dust_radar}
\epsscale{1}
\caption{The science impact for alt\_roll\_dust.}
\end{figure}

%############ Bulge ############
\subsection{Bulge}

We used recommendations from the SAC for different strategies for observing the galactic bulge. These simulations use the Big Sky footprint similar to the Olsen et al white paper.  

We use three footprints for bulge coverage 1) light coverage of the bulge and entire galactic plane, 2) the bulge as deep as WFD and 3) the bulge covered similarly to WFD, but with more observations in $i$.  For each of these strategies, we run a version with natural cadence and one where we boost the priority of the bulge if it has not been observed in 2.5 days. 

\begin{figure}
\epsscale{0.35}
\plotone{plots/pulled_plots/bulges_bs_v1_5_10yrs_Count_observationStartMJD_i_HEAL_SkyMap.pdf}
\plotone{plots/pulled_plots/bulges_bulge_wfd_v1_5_10yrs_Count_observationStartMJD_i_HEAL_SkyMap.pdf}
\plotone{plots/pulled_plots/bulges_i_heavy_v1_5_10yrs_Count_observationStartMJD_i_HEAL_SkyMap.pdf}
\epsscale{1}
\caption{Series of simulations trying different bulge observing strategies.}\label{fig:bulge}
\end{figure}

\begin{figure}
\epsscale{0.65}
\plotone{plots/radar_plots/bulge_radar}
\epsscale{1}
\caption{Science impact of our different bulge strategy simulations. The right panel is a zoom in of the left.}\label{fig:bulgeradar}
\end{figure}

Covering the bulge more deeply, we see an increase in the number of stars and fast microlensing events, with a slight decrease in the SRD metrics.

%############ DCR ############
\subsection{DCR}

\begin{figure}
\epsscale{0.5}
\plotone{plots/pulled_plots/dcr_nham2_ugri_v1_5_10yrs_Nvisits_as_function_of_Alt_Az_HEAL_SkyMap.pdf}
\plotone{plots/pulled_plots/dcr_nham2_ugri_v1_5_10yrs_Count_observationStartMJD_HEAL_SkyMap.pdf}
\plotone{plots/pulled_plots/dcr_nham2_ugri_v1_5_10yrs_Hourglass_year_0-1_HOUR_Hourglass.pdf}
\epsscale{1}
\caption{Intentionally taking observations at higher airmass to measure DCR.}
\end{figure}


The LSST will not have an atmospheric chromatic corrector, thus difference imaging can be complicated by differential chromatic refraction (DCR). There is also potential science opportunities by being able to measure the chromatic shift in objects with sharp features in their SEDs (e.g., AGN with large emission lines).

These experiments look at how we could intentionally schedule a subset of images to be at high airmass so a DCR model could be built up. We test various combinations of filters to demand DCR observations (u+g, u+g+r, and u+g+r+i), and the number of observations to take at high airmass per year (1 or 2). 

Even with 2 high airmass observations per year, we would still expect some area of the sky to fall in chip and raft gaps.  It is also worth noting that in our baseline simulation, we observe a spot on the sky in u typically 60 times, or 6 times per year. Taking 2 high airmass observations per year in u decreases the final coadded depth by 0.15 mags.

Figure~\ref{fig:dcr_radar} shows the science impact is fairly minimal, but we tend to lose $\sim0.1-0.2$\ magnitudes of final coadded depth.

\begin{figure}
\plottwo{plots/radar_plots/dcr_radar}{plots/radar_plots/dcr_mags_radar}
\caption{Science impact of including observations at high airmass for DCR. As expected, pushing observations to high airmass lowers the coadded depths (right) and has as slight negative impact on most science metrics (left).}\label{fig:dcr_radar}
\end{figure}
   
%############ Deep Drilling Fields ############
\subsection{Deep Drilling Fields}

We have run a variety of DDF strategies. Figure~\ref{fig:ddfexamples} shows the same observing season of the DDF ELIASS1 with 5 different strategies. We have run DDF strategies based on white papers from the AGN group and DESC, as well as several other variations. 

\begin{itemize}
    \item{AGN: This strategy takes shorter DDF sequences more often. Only $\sim$2.5\% of visits are spent on DDFs, making the final coadded depths much shallower than other strategies.}
    \item{DESC: a strategy that split the blue and red filters to different days, emphasizing a 3-day cadence}
    \item{Baseline:  Our baseline strategy where 5\% of observations are allocated to DDF observations.}
    \item{Daily: Similar to the baseline, but includes short DDF sequences that can execute daily so there are no long gaps between observations}
    \item{DDF Heavy:  Similar to the baseline, but 13.4\% of visits are allocated to DDF observations}
\end{itemize}


\begin{figure}
\plottwo{plots/radar_plots/ddf1_radar.pdf}{plots/radar_plots/ddf2_radar.pdf}
\caption{On the left, we show the coadded depth in each filter for a representative Deep Drilling Field. Larger values mean deeper coadded depth. On the right we show the standard science metrics.  Because the DDFs take only a small fraction of the total time, the science impacts are fairly minimal.}\label{fig:ddf_differences}
\end{figure}

\begin{figure}
\epsscale{.9}
\plottwo{plots/ddf_plots/ddf_m5_AGN.pdf}{plots/ddf_plots/gap_hist_AGN.pdf}
%\plottwo{plots/ddf_plots/ddf_m5_Baseline_v1_5.pdf}{plots/ddf_plots/gap_hist_Baseline_v1_5.pdf}
\plottwo{plots/ddf_plots/ddf_m5_Baseline_v1_6.pdf}{plots/ddf_plots/gap_hist_Baseline_v1_6.pdf}
\plottwo{plots/ddf_plots/ddf_m5_DESC.pdf}{plots/ddf_plots/gap_hist_DESC.pdf}
\plottwo{plots/ddf_plots/ddf_m5_Daily.pdf}{plots/ddf_plots/gap_hist_Daily.pdf}
\plottwo{plots/ddf_plots/ddf_m5_DDF_Heavy.pdf}{plots/ddf_plots/gap_hist_DDF_Heavy.pdf}
\epsscale{1}
\caption{One observing season of the DDF ELIASS1 from 5 different DDF strategies. }\label{fig:ddfexamples}
\end{figure}

Figure~\ref{fig:ddf_differences} shows the different coadded depths and science impact of the different DDF strategies. Overall, the sceince impact is minimal because all the DDF strategies use a limited amount of the total time, leaving the WFD region relatively unaffected. 

%############ Filter Distribution ############
\subsection{Filter Distribution}

Testing a simple WFD-only footprint, but varying the requested ratio of observations in different filters. The different filter distributions simulated are listed in Table~\ref{table:filtdist}.  

\begin{table}
\begin{centering}
\begin{tabular}{lrrrrrr}
              Name &     $u$ &     $g$ &  $r$ &     $i$ &     $z$ &     $y$ \\
\hline
           Uniform & 1.00 & 1.00 &  1 & 1.00 & 1.00 & 1.00 \\
          Baseline & 0.31 & 0.44 &  1 & 1.00 & 0.90 & 0.90 \\
         $g$ heavy & 0.31 & 1.00 &  1 & 1.00 & 0.90 & 0.90 \\
         $u$ heavy & 0.90 & 0.44 &  1 & 1.00 & 0.90 & 0.90 \\
        $z$ and $y$ heavy & 0.31 & 0.44 &  1 & 1.00 & 1.50 & 1.50 \\
         $i$ heavy & 0.31 & 0.44 &  1 & 1.50 & 0.90 & 0.90 \\
             Bluer & 0.50 & 0.60 &  1 & 1.00 & 0.90 & 0.90 \\
            Redder & 0.31 & 0.44 &  1 & 1.10 & 1.10 & 1.10 \\
\hline
\end{tabular}
\caption{Variations of the filter distribution simulated.}\label{table:filtdist}
\end{centering}
\end{table}

\begin{figure}
\epsscale{0.85}
\plotone{plots/radar_plots/filter_dist_radar}
\epsscale{1}
\caption{Science impact of varying the filter distribution}\label{}
\end{figure}


Varying the filter distribution reveals a slight tension between SNe science and solar system science, with SNe benefiting from more observations in bluer filters. Perhaps most relevant, we do not currently have a photometric redshift metric, which should be very sensitive to the filter distribution.


%############ Footprints ############
\subsection{Footprints}

\begin{figure}
\epsscale{.25}
\plotone{plots/pulled_plots/footprint_add_mag_cloudsv1_5_10yrs_Count_observationStartMJD_HEAL_SkyMap.pdf}
\plotone{plots/pulled_plots/footprint_big_sky_dustv1_5_10yrs_Count_observationStartMJD_HEAL_SkyMap.pdf}
\plotone{plots/pulled_plots/footprint_big_sky_nouiyv1_5_10yrs_Count_observationStartMJD_HEAL_SkyMap.pdf}
\plotone{plots/pulled_plots/footprint_big_skyv1_5_10yrs_Count_observationStartMJD_HEAL_SkyMap.pdf}
\plotone{plots/pulled_plots/footprint_big_wfdv1_5_10yrs_Count_observationStartMJD_HEAL_SkyMap.pdf}
\plotone{plots/pulled_plots/footprint_bluer_footprintv1_5_10yrs_Count_observationStartMJD_HEAL_SkyMap.pdf}
\plotone{plots/pulled_plots/footprint_gp_smoothv1_5_10yrs_Count_observationStartMJD_HEAL_SkyMap.pdf}
\plotone{plots/pulled_plots/footprint_newAv1_5_10yrs_Count_observationStartMJD_HEAL_SkyMap.pdf}
\plotone{plots/pulled_plots/footprint_newBv1_5_10yrs_Count_observationStartMJD_HEAL_SkyMap.pdf}
\plotone{plots/pulled_plots/footprint_no_gp_northv1_5_10yrs_Count_observationStartMJD_HEAL_SkyMap.pdf}
\plotone{plots/pulled_plots/footprint_standard_goalsv1_5_10yrs_Count_observationStartMJD_HEAL_SkyMap.pdf}
\plotone{plots/pulled_plots/footprint_stuck_rollingv1_5_10yrs_Count_observationStartMJD_HEAL_SkyMap.pdf}
\epsscale{1}
\caption{The different survey footprints simulated.}
\end{figure}

We test a wide variation of possible survey footprints. Some of these are more realistic than others. 


\begin{figure}
\plotone{plots/radar_plots/footprints_radar}
\caption{Science impact of varying the survey footprint.}
\end{figure}

As we have come to learn, the fast microlensing rate depends strongly on the footprint. Similarly, the number of stars and number of stars can very greatly on the footprint depending on how much of the galactic plane is covered or how much dusty regions are avoided. The one slightly surprising result is how the number of TNOs can vary with the footprints.

%############ Good Seeing ############

\subsection{Good Seeing}\label{ss:goodseeing}

These test the ability to ensure the entire WFD area is imaged in ``good seeing" conditions every year, here defined as FWHM of 0.7 arcseconds or better.  

These runs work well and it seems to add no particular overhead to the observing. It might make it more challenging to implement in operations, simply because the baseline simulation can simulate an entire night and pass off the list to be observed. If we want to run with the goal of collecting good seeing images, we will need to update the observing queue every time the seeing conditions change significantly, which could result in changing the upcoming observations more often than is desired.

\begin{figure}
\epsscale{0.65}
\plotone{plots/radar_plots/goodseeing_radar}
\epsscale{1}
\caption{The science impact of making sure the sky has template images in good seeing conditions.}
\end{figure}

The science impact of ensuring we have good seeing templates seems to be very minimal, with science metrics varying by only a few percent. 

%############ Short Exposures ############
\subsection{Short Exposures}

\begin{figure}
\plottwo{plots/short_exp_plots/opsim_Count_filter_visitexposuretime_gt_10_and_note_not_like_DD_HEAL_SkyMap.pdf}{plots/short_exp_plots/opsim_Count_filter_visitexposuretime_lt_10_HEAL_SkyMap.pdf}
\caption{Results from including 5s exposures (up to 5 per year). The left shows the number of regular 30s visits (excluding DDF observations) and the right shows the number of 5s visits.}
\end{figure}

We try taking additional short exposures (1s or 5s) twice or five times per year. Taking shorter exposures is a less efficient observing mode, but it seems to have little impact on the overall open shutter fraction. Similar to taking exposures in good seeing conditions, including short exposures each year has only a few percent impact on our science metrics.

\begin{figure}
\epsscale{0.65}
\plotone{plots/radar_plots/shortexp_radar}
\epsscale{1}
\caption{Science impact of covering the sky in short exposures. }
\end{figure}

%############ Spiders ############
\subsection{Spiders}

We look at keeping diffraction spikes aligned along CCD rows and columns. This may result in the camera rotator angle being much less randomized than our baseline rotational dithering strategy. There is little impact on our science metrics, but we note we do not currently have a metric the measures weak lensing systematics.

\begin{figure}
\epsscale{0.65}
\plotone{plots/radar_plots/spider_radar}
\epsscale{1}
\caption{Science impact of keeping diffraction spikes aligned along rows and columns. }
\end{figure}

%############ Third Observation ############
\subsection{Third Observation}

For early identification of transients, it can be helpful to have more than two observations in a night. In these observations, we dedicate between 15 and 120 minutes at the end of the night to attempting to observe areas of sky that already have been observed.  The science impact of adding third observations seems to be minimal. This highlights our need for a metric that quantifies how well we will be able to classify new transients.

\begin{figure}
\epsscale{0.65}
\plotone{plots/radar_plots/third_radar}
\epsscale{1}
\caption{The science impact of dedicating the end of the night to gathering observations of areas that already have pairs.  }
\end{figure}

%############ Twilight NEO Survey ############
\subsection{Twilight NEO Survey}

This is an implementation of Seaman et al. white paper where we use twilight time to take short exposures along the ecliptic to search for NEOs.  If we dedicate all twilight time to NEO searches, we fail to meet the SRD requirements. Thus we also check running the NEO survey every 2, 3, or 4 days. Despite being designed to discover more NEOs, we find that we only discover a few more bright NEOs than the baseline and lose detections of faint NEOs. 

\begin{figure}
\epsscale{0.85}
\plotone{plots/radar_plots/twineo_radar}
\epsscale{1}
\caption{The science impact of using some or all of twilight time for a NEO survey.}\label{fig:neoradar}
\end{figure}

%############ Longer u ############
\subsection{Longer $u$\ Exposure Time}\label{ss:u60}

The u-band observations are often expected be readnoise limited. We test doubling the u-band exposure time and cutting the number of exposures in half. This results in the u-band final coadded depth reaching $\sim$0.20 mags deeper. The $g$-band is also 0.10 mags deeper, with the rest of the filters essentially unchanged in final depth. The $g$\ depth increases because 60s $u$\ exposures decrease the overhead time, freeing up more dark time for $g$\ observations.

Note, we assume that 1x60s visit counts as 2 30s visits for the purpose of meeting the SRD value of 825 visits in the WFD area. Adopting longer exposures in u seems like a good idea, but the SRD will probably need to be modified to ensure it is not ambiguous.

\begin{figure}
\plottwo{plots/radar_plots/u60_radar.pdf}{plots/radar_plots/u60_mags_radar.pdf}
\caption{Increasing the $u$\ exposure time to 60s.  As expected, this results in a substantial gain in $u$\ coadded depth.}
\end{figure}

%############ Variable Exposure Times ############
\subsection{Variable Exposure Times}

\begin{figure}
\plottwo{plots/variable_expt_plots/baseline_spot.pdf}{plots/variable_expt_plots/varexpt_spot.pdf}
\caption{Comparison of a sample WFD point in the baseline and when we vary the exposure time. The individual observations depths become more uniform, especially in the redder filters that can be observed in bright time and twilight.}\label{fig:varexptime}
\end{figure}

We vary the exposure time based on the current conditions so individual exposures have similar depths. There is an argument that taking a full 30s visit in ideal dark time conditions results in ``wasted depth", as more objects and transients will be detected, but then it will be impossible to identify them as later visits are unlikely to be as deep. Similarly, taking a 30s visit in poor conditions will result in a shallow image which will be of limited use. In good conditions, the expsoure time is allowed to shrink to 20s, and in poor conditions it can extend to 100s.

As with doing 60s u band exposures, this may require modifying the detailed specifics of the SRD as longer exposures may need to count as multiple visits.

Having variable exposure time introduces at least 8 new free parameters to the scheduler (the target individual depth for each filter), as well as the shortest and longest acceptable exposure times.  As with \ref{ss:goodseeing}, this would be more complicated to run in operations as the scheduler would need current conditions to calculate the modified exposure times, although the predicted sky brightness may be accurate enough.

Figure~\ref{fig:var_radar} shows the science impact of varying the exposure time is fairly minimal. 

\begin{figure}
\epsscale{0.65}
\plotone{plots/radar_plots/var_exp_radar.pdf}
\epsscale{1}
\caption{Science impact of using variable exposure times.}\label{fig:var_radar}
\end{figure}


%############ WFD Depth ############
\subsection{WFD Depth}

\begin{figure}
\epsscale{0.35}
\plotone{plots/pulled_plots/wfd_depth_scale0_65_noddf_v1_5_10yrs_Count_observationStartMJD_HEAL_SkyMap.pdf}
\plotone{plots/pulled_plots/wfd_depth_scale0_70_noddf_v1_5_10yrs_Count_observationStartMJD_HEAL_SkyMap.pdf}
\plotone{plots/pulled_plots/wfd_depth_scale0_75_noddf_v1_5_10yrs_Count_observationStartMJD_HEAL_SkyMap.pdf}
\plotone{plots/pulled_plots/wfd_depth_scale0_80_noddf_v1_5_10yrs_Count_observationStartMJD_HEAL_SkyMap.pdf}
\plotone{plots/pulled_plots/wfd_depth_scale0_85_noddf_v1_5_10yrs_Count_observationStartMJD_HEAL_SkyMap.pdf}
\plotone{plots/pulled_plots/wfd_depth_scale0_90_noddf_v1_5_10yrs_Count_observationStartMJD_HEAL_SkyMap.pdf}
\plotone{plots/pulled_plots/wfd_depth_scale0_95_noddf_v1_5_10yrs_Count_observationStartMJD_HEAL_SkyMap.pdf}
\plotone{plots/pulled_plots/wfd_depth_scale0_99_noddf_v1_5_10yrs_Count_observationStartMJD_HEAL_SkyMap.pdf}
\epsscale{1}
\caption{Varying the amount of time dedicated to the WFD region between 65\% and 99\% of the visits.}
\end{figure}


We vary what fraction of the observing time is dedicated to the WFD area, from 60\% to 99\% with and without the standard DDF surveys. Unsurprisingly, the SRD is not met if the WFD is only given 60\%.


\begin{figure}
\epsscale{0.85}
\plotone{plots/radar_plots/wfd_depth_radar.pdf}
\epsscale{1}
\caption{The Science impact of varying the WFD depth.}\label{fig:wfd_depth_radar}
\end{figure}

%############ Rolling Cadences ############
\subsection{Rolling Cadences}

Rolling cadence is the term we have given to executing the survey in a non-uniform manner, emphasizing some region of sky one year, then deemphasizing it the next.  Because the SRD includes requirements on stellar proper motion measurements, we are constrained to cover the sky uniformly in at least year 1 and year 10.  We experiment with using rolling cadences where the WFD region is divided in 2, 3, and 6 declination bands. We also scale the rolling strength to be 80, 90, and 99\%. 

\begin{figure}
\epsscale{.35}
\plotone{plots/rolling16/rolling_2_0_8_Count_filter_night_gt_1278_375000_and_night_lt_1643_625000_and_note_not_like_DD_HEAL_SkyMap.pdf}
\plotone{plots/rolling16/rolling_2_0_9_Count_filter_night_gt_1278_375000_and_night_lt_1643_625000_and_note_not_like_DD_HEAL_SkyMap.pdf}
\plotone{plots/rolling16/rolling_2_1_0_Count_filter_night_gt_1278_375000_and_night_lt_1643_625000_and_note_not_like_DD_HEAL_SkyMap.pdf}
\plotone{plots/rolling16/rolling_3_0_8_Count_filter_night_gt_1278_375000_and_night_lt_1643_625000_and_note_not_like_DD_HEAL_SkyMap.pdf}
\plotone{plots/rolling16/rolling_3_0_9_Count_filter_night_gt_1278_375000_and_night_lt_1643_625000_and_note_not_like_DD_HEAL_SkyMap.pdf}
\plotone{plots/rolling16/rolling_3_1_0_Count_filter_night_gt_1278_375000_and_night_lt_1643_625000_and_note_not_like_DD_HEAL_SkyMap.pdf}
\plotone{plots/rolling16/rolling_6_0_8_Count_filter_night_gt_1278_375000_and_night_lt_1643_625000_and_note_not_like_DD_HEAL_SkyMap.pdf}
\plotone{plots/rolling16/rolling_6_0_9_Count_filter_night_gt_1278_375000_and_night_lt_1643_625000_and_note_not_like_DD_HEAL_SkyMap.pdf}
\plotone{plots/rolling16/rolling_6_1_0_Count_filter_night_gt_1278_375000_and_night_lt_1643_625000_and_note_not_like_DD_HEAL_SkyMap.pdf}
\epsscale{1}
\caption{Rolling cadence simulations with 2 (top), 3 (middle), and 6 (bottom) rolling stripes. Here we show the observations taken from 3.5-4.5 years in the survey, excluding the DDF observations.}
\end{figure}


Figure~\ref{fig:rolling_radar} shows the science impact of the different rolling cadence simulations. Overall, the rolling has fairly negligible impact on the science metrics. Metrics from DESC show rolling can be beneficial to SNe lightcurves. 

\begin{figure}
\epsscale{0.65}
\plotone{plots/radar_plots/rolling_radar.pdf}
\epsscale{1}
\caption{Science impact of different rolling simulations. The overall impact seems to be very small. While fast microlensing events can be impacted, that can be made up for by including more of the bulge in the WFD footprint. }\label{fig:rolling_radar}
\end{figure}


%############ Even Filters ############
\subsection{Even Filters}

The baseline simulation is fairly aggressive in switching to redder filters in bright time. This can create long gaps in light curves with no bluer observations. We have run a simulation where only the $u$, and $g$\ filters avoid bright time, and a simulation where only $u$\ avoids bright time. Figure~\ref{fig:even_filt_hourglass} shows the resulting filter distributions in year one. Unlike the baseline simulations, there are no longer sections of several days where only $y$\ is observed.

While the goal of these simulations was to improve SNe Ia lightcurves, the gains appear to be minimal over the baseline strategy.


\begin{figure}
\label{fig:even_filt_hourglass}
\plottwo{plots/pulled_plots/even_filters_g_v1_6_10yrs_Hourglass_year_0-1_HOUR_Hourglass.pdf}{plots/pulled_plots/even_filtersv1_6_10yrs_Hourglass_year_0-1_HOUR_Hourglass.pdf}
\caption{The filter distribution for the even filter simulations. Unlike the baseline simulations, bluer filters are observed in bright time.}
\end{figure}



\begin{figure}
\epsscale{0.65}
\plotone{plots/radar_plots/even_filt_radar.pdf}
\epsscale{1}
\caption{Science performance for the Even Filters runs.  Taking bluer filters in bright time can improve SNe performance and fast transients, but is detrimental to Solar System science.  The loss of depth shows up in most of the other metrics as well.}\label{fig:even_filt_radar}
\end{figure}



%############## Greedy 
\subsection{Ecliptic Pairs}

This simulation prohibits the twilight greedy algorithm from observing near the ecliptic, thus ensuring that all observations near the ecliptic are taken in pairs. This results in modest gains for NEO detection. 


\begin{figure}
\epsscale{0.65}
\plotone{plots/radar_plots/greedy_radar}
\epsscale{1}
\caption{Science impact of not permitting greedy observations near the ecliptic. }
\end{figure}

%############ Aliasing ############
\subsection{Aliasing}

There was concern that if observations were too uniformly placed on the meridian, periodic sources would be aliased. Figure~\ref{fig:alias} shows the FFT of observations at a sample WFD point in the baseline simulation. There is some aliasing at $\sim1$\ day which is inevitable for any ground-based telescope.  The aliasing is much lower than the minion\_1016 simulation that was analyzed in the Bell et al. cadence white paper. 

\begin{figure}
\label{fig:alias}
\epsscale{0.65}
\plotone{plots/alias_plots/aliasing.pdf}
\epsscale{1}
\caption{Aliasing at a sample position in a baseline simulation. There are peaks at harmonics of 24 hours, but this is inevitable with a ground-based telescope. The aliasing seems much lower than earlier version of OpSim where harmonic peaks could be seen past 200 $\mu$Hz.}
\end{figure}

\section{Individual Visit Length}
What to do - 1x30s vs. 2x15s? 1x30s much more efficient (show rough calculation of overhead) than 2x15s, but may have drawbacks due to cosmic ray rejection and potential to miss very rapid transients (or WD detection .. ref white paper). Subtle drawback that 2x15s gives the same "midpoint exposure time" across FOV, 1x30s does not. 

Show difference in 1x30s vs. 2x15s in whatever is our 'standard baseline' at this point. 

Show effect of 7\% loss in efficiency when attempting to combine minisurveys in various configurations (assume we will find some combinations possible with single exposure visits that are impossible with two snaps). 

Relevant metrics: total number of visits, number of visits per field/filter

\section{Intra-night Cadence}
What to do for visit sequence within a night? White paper support for multiple filters within a night (except TNOs maybe?). Potential drawbacks - less efficient (show effect on efficiency). This applies to WFD primarily, but we've applied to any survey that did not have their own specifications (so, everywhere). 

Extension of pairs to $u$ band and $y$ band (show effect). 

Relevant metrics: inter-night visit gaps and SN discovery, SSO discovery/characterization, transient and variable discovery (??), number of visits

\section{Wide-Fast-Deep Footprint}
What to do for WFD footprint? SRD not specific, DESC want low-extinction sky (and depth), but WFD is generally the area of sky that receives the most visits, so generally other science will also benefit from more visits to their relevant areas (particularly galactic plane .. for time-domain studies primarily, not depth)

Relevant metrics: area of sky with 825 visits (under particular restrictions, like total coadded depth and individual image seeing and dust extinction), number of galaxies, number of resolved galaxies, SSO discovery, transient and variable star discovery, astrometry in the galactic plane (?)

\section{Rolling cadence}
Motivation for a rolling cadence (more frequent visits in some years)

Different options for rolling and explanation of how implemented

Should really include discussion of recovery from bad weather years and simulation of same

Relevant metrics: Maintain astrometry requirements, SN discovery, SSO discovery and characterization,  Transient and variable discovery, uniformity of coadded depth / number of visits, 

\section{Northern minisurveys}
Add extension to cover Euclid/DESI with various numbers of visits

Observing NES 

Effect of adding or removing these minisurveys

Relevant metrics: SSO discovery and characterization (particularly active asteroids), depth and number of visits through remainder of North

\section{Southern minisurveys}
Add extension over south celestial pole, LMC/SMC with various numbers of visits

Effect of adding or removing these minisurveys

Relevant metrics: number of visits and coadded depth over SCP, discovery of variables in LMC/SMC (see Olsen white paper for metrics?)

\section{Low Galactic Latitudes}
Discussion of definitions from SAC and recommendations for visits

Effect of adding or removing these minisurveys

Relevant metrics: number of visits, astrometry in bulge, discovery of variables/transients/microlensing in bulge (?)

\section{Twilight Observing}
Discuss need for twilight observing to meet SRD goals (weather, total amount of time available)

Add NEO twilight survey, add DCR white paper (season extension visits?)

Effect of adding or removing these minisurveys

Relevant metrics: NEO discovery, number of visits and coadded depth (and uniformity) in WFD, measurement of DCR, season length

\section{Deep Drilling Fields}
Discuss purpose and how these are scheduled (very different from other fields)

Discuss potential cadences (AGN/ DESC) and how these differ, and our combination of the two

Discuss timing issues with oversubscription (and how much of a problem this could be, what if worse weather?) -- include location of fifth DD field

Effect of adding or removing these minisurveys

Relevant metrics: number of visits and coadded depth for DD, SN detection in DDFs, AGN detection in DDFs
*[solar system minisurvey DDF?]

\section{ToO modes}
Discuss impact of ToO, and how we could implement ToOs in scheduler (various modes: straight to queue by hand or set up known program and supply trigger, etc. -- that we're evaluating the second?)

Relevant metrics: frequency of achieving ToO observations, number of visits and coadded depth in other surveys (WFD or other minisurveys that may be in particular contention)

\section{Making it all work}
Discuss combinations of the above that work together or don't 

Relevant metrics: all

\section{Optimizing parameters}
Somewhere in here we probably ought to talk about optimizing the parameters for each run, and doing bigger sweeps across parameter space. That would easily expand each of the above options by many factors.

\section{Conclusions}
Hopefully here we pare down the evaluation of 100s of runs (like promised) to a set of between 10 to 20 (if this is possible, after combining along different axes). 
The results should come with some basic comments about what's particularly good or bad in each of these areas and how we arrived at these general options. 


% Make sure lsst-texmf/bin/generateAcronyms.py is in your path
\section{Acronyms} \label{sec:acronyms}
\input{acronyms.tex}
