% replace large sections with \include files 

\section{Introduction}

Note: This paper needs to focus on survey strategies and their evaluation. 

Introduction - cover basic idea of survey simulator, scheduler and weather/telescope models. 

Cover basic survey strategy starting point - wide area, frequent coverage, ten year timespan - and why. 

Mention COSEP and call for white papers - idea is to do the best science we can, add last 10\% "best" science. 

Earlier attempts at simulating LSST in \citet{Rothchild19} and \citet{Naghib19}.

\section{Survey Simulator Overview}
Probably need some reference to what survey scheduler was used / how it was set up for various runs, how the runs were performed, and what the input weather and telescope models were like. 

\subsection{The Model Observatory}
Discuss kinematic model, seeing model, weather model. 

\subsection{The Scheduler}

The scheduler is designed to provide real-time decisions on where and how to observe. Because we expect there to be things like weather interruptions, we need a system that can recover quickly. Unlike other traditional telescope schedulers, we do not try to optimize a large number of observations in advance, but rather use a decision tree along with a modified Markov Decision Process. The scheduler behavior is set by a small number of free parameters that can be tuned.

Our baseline scheduler uses a three tier decision tree when deciding what observations to attempt. 

\subsubsection{Tier 1:  Deep Drilling Fields}

The first tier of the decision tree is to check if there are any deep drilling fields that should be executed. We typically have five DDFs in a simulation. 

For a DDF to be eligible to send a sequence to the observing queue, it must
\begin{itemize}
\item{Not currently be twilight}
\item{Have enough time to finish a sequence before twilight begins}
\item{Be in it's target hour angle range}
\item{The moon must be down}
\item{The DDF must not have exceeded it's limit of observations (typically $\sim$1\% of the total number of visits)}
\end{itemize}

If the DDF has not fallen behind, it will space sequences by at least 1.5 days. There is also a check to see if the DDF will be feasible and better observed later in the night, in which case no observations are requested.

If the above conditions are met, the DDF sends it's sequence of observations to the queue to be executed. There are currently no attempts at recovery if a sequence is interrupted. 

The spatial position of the DDF is dithered nightly up to 0.7 degrees.  The camera rotator is also varied nightly to be between -75 and 75 degrees with respect to the telescope. 


\subsubsection{Tier 2:  The Blobs}

If there are no DDFs requesting observations, the decision tree moves to the second tier. This tier is the survey workhorse, executing $\sim$80\% of the simulation visits.  This tier will only request observations if it is not currently twilight, and there is at least 30 minutes before twilight begins.

A modified Markov Decision Process is used to decide what sky region and filter combination to observe given the current conditions and observation history.  We describe the specifics of the MDP in more detail in \S\ref{xxx}.  Briefly, the MDP balances the desire to observe areas 1) that are closest to the optimal possible in terms of 5-sigma depth, 2) which have fallen behind the specified desired survey footprint, 3) are near the current telescope pointing and 4) in the currently loaded filter to minimize filter changes.  In addition to these core components, the MDP includes a mask around zenith, a 30 degree mask around the moon, and small masks around the bright planets (Venus, Mars, Jupiter). The end product of the MDP is a reward function that ranks the desirability of every point in the sky. Because this tier does not execute in twilight, we assume the reward function is relatively stable on 40 minute timescales.

A sky area around the reward function maximum that will take $\sim$22 minutes to observe ($\sim$35 pointings) is then selected. If possible, the area is selected to be be contiguous.  The exact position of the telescope pointings are determined by the sky tessellation, which is randomly oriented for each night. The camera rotator angle (relative to the telescope) is also randomized between $\pm 80$\ degrees each night.

A traveling salesman algorithm is used to put the pointings in an order that minimizes the slew time. The list of pointings are then repeated, usually in a different filter, ensuring moving objects can be detected.  One of seven possible filter combinations is used: $u+g$, $u+r$, $g+r$, $r+i$, $i+z$, $z+y$, or $y+y$.  We use 30 second visits for the majority of simulations. The official baseline uses visits comprised of two 15 second snaps.  


\subsubsection{Tier 3:  Greedy}

If it is during morning or evening twilight, or close to morning twilight, the DDFs and Blob surveys will pass and the decision tree goes to the third and final tier, the greedy surveys. 

The greedy surveys use a similar Markov Decision Process as in Tier 2, but rather than selecting large areas of sky to observe, the survey selects a single pointing at a time.  No attempt is made to observe greedy scheduled observations in pairs.  Since this tier is primarily used in twilight time, it only schedules observations in the redder filters $r$, $i$, $z$, and $y$.  

As with the Blob tier, the sky tessellation orientation is randomized each night so the final survey is spatially dithered. 


\section{Basic Survey Requirements}
Basic survey strategy starting point and why - in more depth? Discuss metrics related to these requirements. 

Probably should show that all survey strategies evaluated do / need to meet these requirements (but maybe later?)

\section{Feedback from white papers and SAC} 
Broad outline of points to evaluate for survey strategy, and our approach in running the subsequent experiments (this should help make sense of what comes next)

Discuss basic types of SAC recommendations. 

%\section{Overview of Metrics}
\section{Survey Requirements and Metrics}

There are many, many options for evaluating the output of the survey strategy experiments. One of the primary goals for the LSST Metrics Analysis Framework (MAF) package was to make it easier for both the project and community members to write metrics to evaluate these outputs. This effort has had some significant successes; SRD-level metrics have been written that cover the primary requirements for the SRD, the DESC working groups have made good progress in writing metrics for their evaluation of the simulations, and the Solar System collaboration has contributed substantial metrics. In other areas, it has been more difficult for the community to engage and contribute directly to MAF; for some of these areas, we have been able to help get metrics running, but clearly there are areas which are lacking definitive metrics. Many of the areas which are lacking relate directly to time domain studies, a critical area for the LSST. We acknowledge this problem and encourage further work by the community, particularly over the next year. 

Here we make a brief summary of some of the top-level science-related metrics. There are thousands of metrics which are run as part of standard MAF analysis; for broad comparisons between simulations we pick a very limited subset of these metrics intended to discover or highlight differences between the simulation survey strategies or to cover major areas of science. 

\subsection{SRD Metrics}

The SRD metrics are designed to cover the primary science requirements laid out in the SRD; the most relevant of these relate to the number of visits per pointing across the WFD region, the area of the WFD region, the parallax and proper motion errors and the number of rapid revisits (on timescales between a few to 40 seconds) per point on the sky. While we check all of these metrics for all runs, the most sensitive to changes in the survey strategy is the number of visits across the WFD, tracked in the fO metric, since we are often attempting to distribute visits into other parts of the sky for other science. 

The fO metric calculates the total number of visits per point on the sky, then calculates how much area is covered with how many visits. This can be summarized across a two axes; the amount of area that receives at least 825 visits per pointing (`fO Area') or the median (or minimum) number of visits that the most frequently visits 18,000 square degrees receives (`fONv MedianNvisits' or `fONv MinNvisits'). The first version, fO Area, tends to be somewhat unstable; the survey hardly ever observes more than18k sq deg to at least 825 visits, because we don't program in larger WFD areas, but if the number of visits across the WFD area falls below 825, the resulting fO Area value will fall rapidly (because we cover the sky uniformly). While fO Area is useful to check, a more useful number is fONv MedianNvisits or MinNvisits. The value of fONv MedianNvisits tells us how many visits the typical field in the top 18k sq degrees receives; fONv MinimumNvisits tells us the fewest number of visits any of those top 18k sq deg received. Typically we see fONv MedianNvisits scales more smoothly with the fraction of visits devoted to WFD and likely represents science metrics that depend on having a reasonably large amount of visits over the entire WFD well. 

The radar plots use fONv MedianNvisits, the Median Parallax Error at XXX, and the Median Proper Motion Error at XXX.

\subsection{Solar System Science Metrics}

Solar System science metrics include discovery metrics (with various discovery criteria, such as detections in 3 nights with pairs of visits within a 15 night window) and characterization metrics (ie. how many colors for objects can we measure, and can we determine a light curve or even shape measurement from the lightcurve), contributed by both project and science collaboration. The most important metric for solar system objects is discovery; finding the objects is the first priority. Characterization metrics are secondary metrics. For each of these metrics, we generate input observations using an appropriate solar system population: Potentially Hazardous Asteroids (PHAs) and Near Earth Objects (NEOs) based on a model by \citet{2018Icar..312..181G}, Main Belt Asteroids (MBAs) and Jovian Trojans based on the S3M model from \citet{2011PASP..123..423G}, and TransNeptunian Objects (TNOs) based on the L7 model from the Canda France Ecliptic Plane Survey (CFEPS) \citep{2009AJ....137.4917K, 2011AJ....142..131P}. These populations move at varying rates and cover varying amounts of the sky. NEOs move over much of the sky during the lifetime of the survey, so are less sensitive to footprint variations, but tend to have much more strongly varying brightnesses, thus are sensitive to the number and timing of visits (must be observed when they are bright). TNOs move very slowly, not more than a few fields of view over the lifetime of the survey, so are quite sensitive to footprint, however they are relatively consistent in their brightness; thus they are less sensitive to the overall number of visits at a particular point in the sky, once a threshold has been met. 

For each of these populations, we calculate the population completeness due to discovery with the LSST at the end of 10 years (not including previous surveys) with the currently Moving Object Pipeline baseline criteria; 3 nights with pairs of visits within 15 nights at a range of absolute magnitude $H$ (approximately the size of the object) and then take the completeness at an $H$ value near peak completeness and an $H$ value that is relatively close to 50\% completeness in the baseline; these completeness values are the summary metrics we track across various runs to compare them here. 

The radar plots use the completeness for bright NEOs, faint NEOs, and XXX TNOs. 

\subsection{Number of Galaxies}

The estimated expected number of galaxies, across the entire survey footprint, is calculated using \href{https://github.com/LSST-nonproject/sims_maf_contrib/blob/master/mafContrib/LSSObsStrategy/galaxyCountsMetric_extended.py#L26}{GalaxyCountsMetric\_extended}, from \href{https://github.com/LSST-nonproject/sims_maf_contrib}{sims\_maf\_contrib}. The number of galaxies is estimated based on the coadded depth using redshift-bin-specific powerlaws, based on mock catalogs from \citet{2003MNRAS.343..796P}. 

The radar plots use the total number of galaxies down to the coadded limiting magnitude over the entire survey footprint.

\subsection{Number of Stars}

XXX

The radar plots use the total number of stars over the entire footprint down to the coadded limiting magnitude. (crowding?)

\subsection{DESC WFD Metrics}

The DESC has contributed several metrics evaluating the performance of the WFD for various areas of relevant science. Many of these metrics are built on calculating a subset of the survey footprint that meets the requirements of coverage in all 6 filters, less than a specified level of dust extinction (E(B-V) $<$ 0.2) and greater than a specified coadded depth in $i$ band ($i$ $>$ 25.9 at 10 years), calculated using \href{https://github.com/lsst/sims_maf/blob/master/python/lsst/sims/maf/metrics/weakLensingSystematicsMetric.py#L8}{ExgalM5\_with\_cuts}. This represents the extragalactic science footprint. 

\subsubsection{Static Science}
Over this extragalactic footprint the following metrics are calculated for general `static science'.
\begin{itemize}
\item Median coadded depth in $i$ band
\item Standard deviation of the coadded depth in $i$ band
\item The area of the selected footprint
\item A 3x2 point Figure of Merit emulator
\end{itemize}. 
The radar plot uses the 3x2point FoM. 

\subsubsection{Weak Lensing}
The same footprint is used to calculate the number of visits per point in the footprint (WeakLensingNvisits); this is used as an approximate metric evaluating weak lensing systematics.
The radar plot uses the mean number of visits across the extragalactic footprint. 

\subsubsection{Large Scale Structure}
The number of galaxies within this same footprint is used to as a metric to approximate large scale structure results (DepthLimitedNumGalaxies), using the same GalaxyCountsMetric\_extended as above, but limiting the result to the selected footprint. 

\subsubsection{SNe Ia}

XXX

\subsection{Tidal Disruption Events (TDE)}

XXX

\subsection{Fast Microlensing}

Light curves contributed from the community.

While we also have a slow microlensing metric, we find very little variation over different siumulations.

Because the baseline footprint has sparse coverage of the Galactic bulge, the baseline value of this metric is relatively small, making the normalized values more volitile than the other metrics here.


\subsection{Radar Plots}

Explain that the radar plots have values normalized (typically to a relevant baseline run). For the parallax and proper motion metrics, the inverse of the errors are compared. For magnitudes, we plot magnitude difference (with larger values indicating deeper depths).

XXX--Maybe put in a table of the raw values for the 1.5 and 1.6 baseline runs. (appendix??)



\section{Individual Scheduler Experiments}

Here we look at various experiments that explore varying a single aspect of the scheduler. 


%############# varying u-band #############
\subsection{$u$\ Filter Pairing}\label{ss:ufilt}

\begin{figure}
\plotone{plots/radar_plots/upairs30_radar}
\plotone{plots/radar_plots/upairs60_radar}
\caption{Varying when the $u$\ filter is swapped out of the camera as well as adding additional weight to the $u$\ footprint.}
\end{figure}

For this family of simulations, we only take $u$\ observations paired $\sim$22 minutes later with $g$\ or with $r$. We vary when the $u$\ filter is loaded into the camera (at lunar illuminations of 15, 30, 40, and 60\%) and vary the strength of the $u$\ footprint (1, 2, or 4 times the baseline footprint).

There is a tension between TDEs and most other science cases, with TDEs benefitting from more $u$\ visits and other science cases staying the same or dropping with increased $u$.


%############# Filter Loading #############
\subsection{Filter Loading}

\begin{figure}
\epsscale{0.5}
\plotone{plots/radar_plots/filter_load_radar}
\epsscale{1}
\caption{Varying when the $u$\ filter is loaded.}
\end{figure}

Similar to the experiment in \S\ref{ss:ufilt}, only $u$\ visits can be paired with $u$, $g$, or $r$. We vary when the $u$ filter is loaded into the camera (lunar illumination of 5, 10, 15, 20, 30, 45, or 60\%). 

As before, the TDE metric seems to be the most sensitive to the $u$\ observing strategy.


%############ Dust With Alternating ############
\subsection{Dust With Alternating}

\begin{figure}
\epsscale{0.5}
\plotone{plots/pulled_plots/alt_roll_mod2_dust_sdf_0_20_v1_5_10yrs_Count_observationStartMJD_HEAL_SkyMap}
\plotone{plots/pulled_plots/alt_roll_mod2_dust_sdf_0_20_v1_5_10yrs_Nvisits_as_function_of_Alt_Az_HEAL_SkyMap}
\plotone{plots/pulled_plots/alt_roll_mod2_dust_sdf_0_20_v1_5_10yrs_Hourglass_year_0-1_HOUR_Hourglass}
\epsscale{1}
\caption{The alt\_roll\_dust simulation that uses a footprint to avoid high extinction and tries to drive an every-other-day cadence.}\label{fig:altdust}
\end{figure}

This uses the dusty footprint and a basis function to encourage the scheduler to alternate between the north and south nightly. This is similar to what was originally done in the altSched simulations \citep{Rothchild19}. This can help keep light curve sampling optimally spaced. By using a basis function, we encourage alternating north/south, but it is not absolutely enforced, making it possible for the scheduler to avoid the moon. Note we have improved the rolling cadence implementation to eliminate the over-exposed stripes and high airmass observations.

There is no additional NES, however there is a strip in the north observed in $g$, $r$, $i$, and $z$.

The science impact of this strategy is fairly minimal. By avoiding extinction regions, we have more stars and galaxies. The coverage of the LMC also increases the number of fast microlensing events. 

\begin{figure}
\epsscale{0.65}
\plotone{plots/radar_plots/alt_dust_radar}
\epsscale{1}
\caption{The science impact for alt\_roll\_dust.}
\end{figure}

%############ Bulge ############
\subsection{Bulge}

We used recommendations from the SAC for different strategies for observing the galactic bulge. These simulations use the Big Sky footprint similar to the Olsen et al white paper.  

We use three footprints for bulge coverage 1) light coverage of the bulge and entire galactic plane, 2) the bulge as deep as WFD and 3) the bulge covered similarly to WFD, but with more observations in $i$.  For each of these strategies, we run a version with natural cadence and one where we boost the priority of the bulge if it has not been observed in 2.5 days. 

\begin{figure}
\epsscale{0.35}
\plotone{plots/pulled_plots/bulges_bs_v1_5_10yrs_Count_observationStartMJD_i_HEAL_SkyMap.pdf}
\plotone{plots/pulled_plots/bulges_bulge_wfd_v1_5_10yrs_Count_observationStartMJD_i_HEAL_SkyMap.pdf}
\plotone{plots/pulled_plots/bulges_i_heavy_v1_5_10yrs_Count_observationStartMJD_i_HEAL_SkyMap.pdf}
\epsscale{1}
\caption{Series of simulations trying different bulge observing strategies.}\label{fig:bulge}
\end{figure}

\begin{figure}
\epsscale{0.65}
\plotone{plots/radar_plots/bulge_radar}
\epsscale{1}
\caption{Science impact of our different bulge strategy simulations. The right panel is a zoom in of the left.}\label{fig:bulgeradar}
\end{figure}

Covering the bulge more deeply, we see an increase in the number of stars and fast microlensing events, with a slight decrease in the SRD metrics.

%############ DCR ############
\subsection{DCR}

\begin{figure}
\epsscale{0.5}
\plotone{plots/pulled_plots/dcr_nham2_ugri_v1_5_10yrs_Nvisits_as_function_of_Alt_Az_HEAL_SkyMap.pdf}
\plotone{plots/pulled_plots/dcr_nham2_ugri_v1_5_10yrs_Count_observationStartMJD_HEAL_SkyMap.pdf}
\plotone{plots/pulled_plots/dcr_nham2_ugri_v1_5_10yrs_Hourglass_year_0-1_HOUR_Hourglass.pdf}
\epsscale{1}
\caption{Intentionally taking observations at higher airmass to measure DCR.}
\end{figure}


The LSST will not have an atmospheric chromatic corrector, thus difference imaging can be complicated by differential chromatic refraction (DCR). There is also potential science opportunities by being able to measure the chromatic shift in objects with sharp features in their SEDs (e.g., AGN with large emission lines).

These experiments look at how we could intentionally schedule a subset of images to be at high airmass so a DCR model could be built up. We test various combinations of filters to demand DCR observations (u+g, u+g+r, and u+g+r+i), and the number of observations to take at high airmass per year (1 or 2). 

Even with 2 high airmass observations per year, we would still expect some area of the sky to fall in chip and raft gaps.  It is also worth noting that in our baseline simulation, we observe a spot on the sky in u typically 60 times, or 6 times per year. Taking 2 high airmass observations per year in u decreases the final coadded depth by 0.15 mags.

Figure~\ref{fig:dcr_radar} shows the science impact is fairly minimal, but we tend to lose $\sim0.1-0.2$\ magnitudes of final coadded depth.

\begin{figure}
\plottwo{plots/radar_plots/dcr_radar}{plots/radar_plots/dcr_mags_radar}
\caption{Science impact of including observations at high airmass for DCR. As expected, pushing observations to high airmass lowers the coadded depths (right) and has as slight negative impact on most science metrics (left).}\label{fig:dcr_radar}
\end{figure}
   
%############ Deep Drilling Fields ############
\subsection{Deep Drilling Fields}

We have run a variety of DDF strategies. Figure~\ref{fig:ddfexamples} shows the same observing season of the DDF ELIASS1 with 5 different strategies. We have run DDF strategies based on white papers from the AGN group and DESC, as well as several other variations. 

\begin{itemize}
    \item{AGN: This strategy takes shorter DDF sequences more often. Only $\sim$2.5\% of visits are spent on DDFs, making the final coadded depths much shallower than other strategies.}
    \item{DESC: a strategy that split the blue and red filters to different days, emphasizing a 3-day cadence}
    \item{Baseline:  Our baseline strategy where 5\% of observations are allocated to DDF observations.}
    \item{Daily: Similar to the baseline, but includes short DDF sequences that can execute daily so there are no long gaps between observations}
    \item{DDF Heavy:  Similar to the baseline, but 13.4\% of visits are allocated to DDF observations}
\end{itemize}


\begin{figure}
\plottwo{plots/radar_plots/ddf1_radar.pdf}{plots/radar_plots/ddf2_radar.pdf}
\caption{On the left, we show the coadded depth in each filter for a representative Deep Drilling Field. Larger values mean deeper coadded depth. On the right we show the standard science metrics.  Because the DDFs take only a small fraction of the total time, the science impacts are fairly minimal.}\label{fig:ddf_differences}
\end{figure}

\begin{figure}
\epsscale{.9}
\plottwo{plots/ddf_plots/ddf_m5_AGN.pdf}{plots/ddf_plots/gap_hist_AGN.pdf}
%\plottwo{plots/ddf_plots/ddf_m5_Baseline_v1_5.pdf}{plots/ddf_plots/gap_hist_Baseline_v1_5.pdf}
\plottwo{plots/ddf_plots/ddf_m5_Baseline_v1_6.pdf}{plots/ddf_plots/gap_hist_Baseline_v1_6.pdf}
\plottwo{plots/ddf_plots/ddf_m5_DESC.pdf}{plots/ddf_plots/gap_hist_DESC.pdf}
\plottwo{plots/ddf_plots/ddf_m5_Daily.pdf}{plots/ddf_plots/gap_hist_Daily.pdf}
\plottwo{plots/ddf_plots/ddf_m5_DDF_Heavy.pdf}{plots/ddf_plots/gap_hist_DDF_Heavy.pdf}
\epsscale{1}
\caption{One observing season of the DDF ELIASS1 from 5 different DDF strategies. }\label{fig:ddfexamples}
\end{figure}

Figure~\ref{fig:ddf_differences} shows the different coadded depths and science impact of the different DDF strategies. Overall, the sceince impact is minimal because all the DDF strategies use a limited amount of the total time, leaving the WFD region relatively unaffected. 

%############ Filter Distribution ############
\subsection{Filter Distribution}

Testing a simple WFD-only footprint, but varying the requested ratio of observations in different filters. The different filter distributions simulated are listed in Table~\ref{table:filtdist}.  

\begin{table}
\begin{centering}
\begin{tabular}{lrrrrrr}
              Name &     $u$ &     $g$ &  $r$ &     $i$ &     $z$ &     $y$ \\
\hline
           Uniform & 1.00 & 1.00 &  1 & 1.00 & 1.00 & 1.00 \\
          Baseline & 0.31 & 0.44 &  1 & 1.00 & 0.90 & 0.90 \\
         $g$ heavy & 0.31 & 1.00 &  1 & 1.00 & 0.90 & 0.90 \\
         $u$ heavy & 0.90 & 0.44 &  1 & 1.00 & 0.90 & 0.90 \\
        $z$ and $y$ heavy & 0.31 & 0.44 &  1 & 1.00 & 1.50 & 1.50 \\
         $i$ heavy & 0.31 & 0.44 &  1 & 1.50 & 0.90 & 0.90 \\
             Bluer & 0.50 & 0.60 &  1 & 1.00 & 0.90 & 0.90 \\
            Redder & 0.31 & 0.44 &  1 & 1.10 & 1.10 & 1.10 \\
\hline
\end{tabular}
\caption{Variations of the filter distribution simulated.}\label{table:filtdist}
\end{centering}
\end{table}

\begin{figure}
\epsscale{0.85}
\plotone{plots/radar_plots/filter_dist_radar}
\epsscale{1}
\caption{Science impact of varying the filter distribution}\label{}
\end{figure}


Varying the filter distribution reveals a slight tension between SNe science and solar system science, with SNe benefiting from more observations in bluer filters. Perhaps most relevant, we do not currently have a photometric redshift metric, which should be very sensitive to the filter distribution.


%############ Footprints ############
\subsection{Footprints}

\begin{figure}
\epsscale{.25}
\plotone{plots/pulled_plots/footprint_add_mag_cloudsv1_5_10yrs_Count_observationStartMJD_HEAL_SkyMap.pdf}
\plotone{plots/pulled_plots/footprint_big_sky_dustv1_5_10yrs_Count_observationStartMJD_HEAL_SkyMap.pdf}
\plotone{plots/pulled_plots/footprint_big_sky_nouiyv1_5_10yrs_Count_observationStartMJD_HEAL_SkyMap.pdf}
\plotone{plots/pulled_plots/footprint_big_skyv1_5_10yrs_Count_observationStartMJD_HEAL_SkyMap.pdf}
\plotone{plots/pulled_plots/footprint_big_wfdv1_5_10yrs_Count_observationStartMJD_HEAL_SkyMap.pdf}
\plotone{plots/pulled_plots/footprint_bluer_footprintv1_5_10yrs_Count_observationStartMJD_HEAL_SkyMap.pdf}
\plotone{plots/pulled_plots/footprint_gp_smoothv1_5_10yrs_Count_observationStartMJD_HEAL_SkyMap.pdf}
\plotone{plots/pulled_plots/footprint_newAv1_5_10yrs_Count_observationStartMJD_HEAL_SkyMap.pdf}
\plotone{plots/pulled_plots/footprint_newBv1_5_10yrs_Count_observationStartMJD_HEAL_SkyMap.pdf}
\plotone{plots/pulled_plots/footprint_no_gp_northv1_5_10yrs_Count_observationStartMJD_HEAL_SkyMap.pdf}
\plotone{plots/pulled_plots/footprint_standard_goalsv1_5_10yrs_Count_observationStartMJD_HEAL_SkyMap.pdf}
\plotone{plots/pulled_plots/footprint_stuck_rollingv1_5_10yrs_Count_observationStartMJD_HEAL_SkyMap.pdf}
\epsscale{1}
\caption{The different survey footprints simulated.}
\end{figure}

We test a wide variation of possible survey footprints. Some of these are more realistic than others. 


\begin{figure}
\plotone{plots/radar_plots/footprints_radar}
\caption{Science impact of varying the survey footprint.}
\end{figure}

As we have come to learn, the fast microlensing rate depends strongly on the footprint. Similarly, the number of stars and number of stars can very greatly on the footprint depending on how much of the galactic plane is covered or how much dusty regions are avoided. The one slightly surprising result is how the number of TNOs can vary with the footprints.

%############ Good Seeing ############

\subsection{Good Seeing}\label{ss:goodseeing}

These test the ability to ensure the entire WFD area is imaged in ``good seeing" conditions every year, here defined as FWHM of 0.7 arcseconds or better.  

These runs work well and it seems to add no particular overhead to the observing. It might make it more challenging to implement in operations, simply because the baseline simulation can simulate an entire night and pass off the list to be observed. If we want to run with the goal of collecting good seeing images, we will need to update the observing queue every time the seeing conditions change significantly, which could result in changing the upcoming observations more often than is desired.

\begin{figure}
\epsscale{0.65}
\plotone{plots/radar_plots/goodseeing_radar}
\epsscale{1}
\caption{The science impact of making sure the sky has template images in good seeing conditions.}
\end{figure}

The science impact of ensuring we have good seeing templates seems to be very minimal, with science metrics varying by only a few percent. 

%############ Short Exposures ############
\subsection{Short Exposures}

\begin{figure}
\plottwo{plots/short_exp_plots/opsim_Count_filter_visitexposuretime_gt_10_and_note_not_like_DD_HEAL_SkyMap.pdf}{plots/short_exp_plots/opsim_Count_filter_visitexposuretime_lt_10_HEAL_SkyMap.pdf}
\caption{Results from including 5s exposures (up to 5 per year). The left shows the number of regular 30s visits (excluding DDF observations) and the right shows the number of 5s visits.}
\end{figure}

We try taking additional short exposures (1s or 5s) twice or five times per year. Taking shorter exposures is a less efficient observing mode, but it seems to have little impact on the overall open shutter fraction. Similar to taking exposures in good seeing conditions, including short exposures each year has only a few percent impact on our science metrics.

\begin{figure}
\epsscale{0.65}
\plotone{plots/radar_plots/shortexp_radar}
\epsscale{1}
\caption{Science impact of covering the sky in short exposures. }
\end{figure}

%############ Spiders ############
\subsection{Spiders}

We look at keeping diffraction spikes aligned along CCD rows and columns. This may result in the camera rotator angle being much less randomized than our baseline rotational dithering strategy. There is little impact on our science metrics, but we note we do not currently have a metric the measures weak lensing systematics.

\begin{figure}
\epsscale{0.65}
\plotone{plots/radar_plots/spider_radar}
\epsscale{1}
\caption{Science impact of keeping diffraction spikes aligned along rows and columns. }
\end{figure}

%############ Third Observation ############
\subsection{Third Observation}

For early identification of transients, it can be helpful to have more than two observations in a night. In these observations, we dedicate between 15 and 120 minutes at the end of the night to attempting to observe areas of sky that already have been observed.  The science impact of adding third observations seems to be minimal. This highlights our need for a metric that quantifies how well we will be able to classify new transients.

\begin{figure}
\epsscale{0.65}
\plotone{plots/radar_plots/third_radar}
\epsscale{1}
\caption{The science impact of dedicating the end of the night to gathering observations of areas that already have pairs.  }
\end{figure}

%############ Twilight NEO Survey ############
\subsection{Twilight NEO Survey}

This is an implementation of Seaman et al. white paper where we use twilight time to take short exposures along the ecliptic to search for NEOs.  If we dedicate all twilight time to NEO searches, we fail to meet the SRD requirements. Thus we also check running the NEO survey every 2, 3, or 4 days. Despite being designed to discover more NEOs, we find that we only discover a few more bright NEOs than the baseline and lose detections of faint NEOs. 

\begin{figure}
\epsscale{0.85}
\plotone{plots/radar_plots/twineo_radar}
\epsscale{1}
\caption{The science impact of using some or all of twilight time for a NEO survey.}\label{fig:neoradar}
\end{figure}

%############ Longer u ############
\subsection{Longer $u$\ Exposure Time}\label{ss:u60}

The u-band observations are often expected be readnoise limited. We test doubling the u-band exposure time and cutting the number of exposures in half. This results in the u-band final coadded depth reaching $\sim$0.20 mags deeper. The $g$-band is also 0.10 mags deeper, with the rest of the filters essentially unchanged in final depth. The $g$\ depth increases because 60s $u$\ exposures decrease the overhead time, freeing up more dark time for $g$\ observations.

Note, we assume that 1x60s visit counts as 2 30s visits for the purpose of meeting the SRD value of 825 visits in the WFD area. Adopting longer exposures in u seems like a good idea, but the SRD will probably need to be modified to ensure it is not ambiguous.

\begin{figure}
\plottwo{plots/radar_plots/u60_radar.pdf}{plots/radar_plots/u60_mags_radar.pdf}
\caption{Increasing the $u$\ exposure time to 60s.  As expected, this results in a substantial gain in $u$\ coadded depth.}
\end{figure}

%############ Variable Exposure Times ############
\subsection{Variable Exposure Times}

\begin{figure}
\plottwo{plots/variable_expt_plots/baseline_spot.pdf}{plots/variable_expt_plots/varexpt_spot.pdf}
\caption{Comparison of a sample WFD point in the baseline and when we vary the exposure time. The individual observations depths become more uniform, especially in the redder filters that can be observed in bright time and twilight.}\label{fig:varexptime}
\end{figure}

We vary the exposure time based on the current conditions so individual exposures have similar depths. There is an argument that taking a full 30s visit in ideal dark time conditions results in ``wasted depth", as more objects and transients will be detected, but then it will be impossible to identify them as later visits are unlikely to be as deep. Similarly, taking a 30s visit in poor conditions will result in a shallow image which will be of limited use. In good conditions, the expsoure time is allowed to shrink to 20s, and in poor conditions it can extend to 100s.

As with doing 60s u band exposures, this may require modifying the detailed specifics of the SRD as longer exposures may need to count as multiple visits.

Having variable exposure time introduces at least 8 new free parameters to the scheduler (the target individual depth for each filter), as well as the shortest and longest acceptable exposure times.  As with \ref{ss:goodseeing}, this would be more complicated to run in operations as the scheduler would need current conditions to calculate the modified exposure times, although the predicted sky brightness may be accurate enough.

Figure~\ref{fig:var_radar} shows the science impact of varying the exposure time is fairly minimal. 

\begin{figure}
\epsscale{0.65}
\plotone{plots/radar_plots/var_exp_radar.pdf}
\epsscale{1}
\caption{Science impact of using variable exposure times.}\label{fig:var_radar}
\end{figure}


%############ WFD Depth ############
\subsection{WFD Depth}

\begin{figure}
\epsscale{0.35}
\plotone{plots/pulled_plots/wfd_depth_scale0_65_noddf_v1_5_10yrs_Count_observationStartMJD_HEAL_SkyMap.pdf}
\plotone{plots/pulled_plots/wfd_depth_scale0_70_noddf_v1_5_10yrs_Count_observationStartMJD_HEAL_SkyMap.pdf}
\plotone{plots/pulled_plots/wfd_depth_scale0_75_noddf_v1_5_10yrs_Count_observationStartMJD_HEAL_SkyMap.pdf}
\plotone{plots/pulled_plots/wfd_depth_scale0_80_noddf_v1_5_10yrs_Count_observationStartMJD_HEAL_SkyMap.pdf}
\plotone{plots/pulled_plots/wfd_depth_scale0_85_noddf_v1_5_10yrs_Count_observationStartMJD_HEAL_SkyMap.pdf}
\plotone{plots/pulled_plots/wfd_depth_scale0_90_noddf_v1_5_10yrs_Count_observationStartMJD_HEAL_SkyMap.pdf}
\plotone{plots/pulled_plots/wfd_depth_scale0_95_noddf_v1_5_10yrs_Count_observationStartMJD_HEAL_SkyMap.pdf}
\plotone{plots/pulled_plots/wfd_depth_scale0_99_noddf_v1_5_10yrs_Count_observationStartMJD_HEAL_SkyMap.pdf}
\epsscale{1}
\caption{Varying the amount of time dedicated to the WFD region between 65\% and 99\% of the visits.}
\end{figure}


We vary what fraction of the observing time is dedicated to the WFD area, from 60\% to 99\% with and without the standard DDF surveys. Unsurprisingly, the SRD is not met if the WFD is only given 60\%.


\begin{figure}
\epsscale{0.85}
\plotone{plots/radar_plots/wfd_depth_radar.pdf}
\epsscale{1}
\caption{The Science impact of varying the WFD depth.}\label{fig:wfd_depth_radar}
\end{figure}

%############ Rolling Cadences ############
\subsection{Rolling Cadences}

Rolling cadence is the term we have given to executing the survey in a non-uniform manner, emphasizing some region of sky one year, then deemphasizing it the next.  Because the SRD includes requirements on stellar proper motion measurements, we are constrained to cover the sky uniformly in at least year 1 and year 10.  We experiment with using rolling cadences where the WFD region is divided in 2, 3, and 6 declination bands. We also scale the rolling strength to be 80, 90, and 99\%. 

\begin{figure}
\epsscale{.35}
\plotone{plots/rolling16/rolling_2_0_8_Count_filter_night_gt_1278_375000_and_night_lt_1643_625000_and_note_not_like_DD_HEAL_SkyMap.pdf}
\plotone{plots/rolling16/rolling_2_0_9_Count_filter_night_gt_1278_375000_and_night_lt_1643_625000_and_note_not_like_DD_HEAL_SkyMap.pdf}
\plotone{plots/rolling16/rolling_2_1_0_Count_filter_night_gt_1278_375000_and_night_lt_1643_625000_and_note_not_like_DD_HEAL_SkyMap.pdf}
\plotone{plots/rolling16/rolling_3_0_8_Count_filter_night_gt_1278_375000_and_night_lt_1643_625000_and_note_not_like_DD_HEAL_SkyMap.pdf}
\plotone{plots/rolling16/rolling_3_0_9_Count_filter_night_gt_1278_375000_and_night_lt_1643_625000_and_note_not_like_DD_HEAL_SkyMap.pdf}
\plotone{plots/rolling16/rolling_3_1_0_Count_filter_night_gt_1278_375000_and_night_lt_1643_625000_and_note_not_like_DD_HEAL_SkyMap.pdf}
\plotone{plots/rolling16/rolling_6_0_8_Count_filter_night_gt_1278_375000_and_night_lt_1643_625000_and_note_not_like_DD_HEAL_SkyMap.pdf}
\plotone{plots/rolling16/rolling_6_0_9_Count_filter_night_gt_1278_375000_and_night_lt_1643_625000_and_note_not_like_DD_HEAL_SkyMap.pdf}
\plotone{plots/rolling16/rolling_6_1_0_Count_filter_night_gt_1278_375000_and_night_lt_1643_625000_and_note_not_like_DD_HEAL_SkyMap.pdf}
\epsscale{1}
\caption{Rolling cadence simulations with 2 (top), 3 (middle), and 6 (bottom) rolling stripes. Here we show the observations taken from 3.5-4.5 years in the survey, excluding the DDF observations.}
\end{figure}


Figure~\ref{fig:rolling_radar} shows the science impact of the different rolling cadence simulations. Overall, the rolling has fairly negligible impact on the science metrics. Metrics from DESC show rolling can be beneficial to SNe lightcurves. 

\begin{figure}
\epsscale{0.65}
\plotone{plots/radar_plots/rolling_radar.pdf}
\epsscale{1}
\caption{Science impact of different rolling simulations. The overall impact seems to be very small. While fast microlensing events can be impacted, that can be made up for by including more of the bulge in the WFD footprint. }\label{fig:rolling_radar}
\end{figure}


%############ Even Filters ############
\subsection{Even Filters}

The baseline simulation is fairly aggressive in switching to redder filters in bright time. This can create long gaps in light curves with no bluer observations. We have run a simulation where only the $u$, and $g$\ filters avoid bright time, and a simulation where only $u$\ avoids bright time. Figure~\ref{fig:even_filt_hourglass} shows the resulting filter distributions in year one. Unlike the baseline simulations, there are no longer sections of several days where only $y$\ is observed.

While the goal of these simulations was to improve SNe Ia lightcurves, the gains appear to be minimal over the baseline strategy.


\begin{figure}
\label{fig:even_filt_hourglass}
\plottwo{plots/pulled_plots/even_filters_g_v1_6_10yrs_Hourglass_year_0-1_HOUR_Hourglass.pdf}{plots/pulled_plots/even_filtersv1_6_10yrs_Hourglass_year_0-1_HOUR_Hourglass.pdf}
\caption{The filter distribution for the even filter simulations. Unlike the baseline simulations, bluer filters are observed in bright time.}
\end{figure}



\begin{figure}
\epsscale{0.65}
\plotone{plots/radar_plots/even_filt_radar.pdf}
\epsscale{1}
\caption{Science performance for the Even Filters runs.  Taking bluer filters in bright time can improve SNe performance and fast transients, but is detrimental to Solar System science.  The loss of depth shows up in most of the other metrics as well.}\label{fig:even_filt_radar}
\end{figure}



%############## Greedy 
\subsection{Ecliptic Pairs}

This simulation prohibits the twilight greedy algorithm from observing near the ecliptic, thus ensuring that all observations near the ecliptic are taken in pairs. This results in modest gains for NEO detection. 


\begin{figure}
\epsscale{0.65}
\plotone{plots/radar_plots/greedy_radar}
\epsscale{1}
\caption{Science impact of not permitting greedy observations near the ecliptic. }
\end{figure}

%############ Aliasing ############
\subsection{Aliasing}

There was concern that if observations were too uniformly placed on the meridian, periodic sources would be aliased. Figure~\ref{fig:alias} shows the FFT of observations at a sample WFD point in the baseline simulation. There is some aliasing at $\sim1$\ day which is inevitable for any ground-based telescope.  The aliasing is much lower than the minion\_1016 simulation that was analyzed in the Bell et al. cadence white paper. 

\begin{figure}
\label{fig:alias}
\epsscale{0.65}
\plotone{plots/alias_plots/aliasing.pdf}
\epsscale{1}
\caption{Aliasing at a sample position in a baseline simulation. There are peaks at harmonics of 24 hours, but this is inevitable with a ground-based telescope. The aliasing seems much lower than earlier version of OpSim where harmonic peaks could be seen past 200 $\mu$Hz.}
\end{figure}

\section{FBS release v1.6: Candidate release runs}\label{sec:1.6}

Here we describe the runs done as part of the `candidate baselines' in the FBS 1.6 release.  This set of simulations is unlike the previous experiments, in that instead of varying a particular survey strategy option across a family, we have attempted to set up a limited number of simulations that attempt to strongly boost particular goals. They are examples of more extreme choices for survey strategies; many of these options have serious drawbacks when considering an overview of science. Calling these `candidate baselines' in no way implies they are better than some of the survey strategies explored in the FBS 1.5 release, nor that all of these would be suitable choices for an initial survey strategy; they are just intended to be exploratory examples. 

The radar plots for all of these FBS 1.6 simulations are show in Figure~\ref{fig:v16radar}. 

\begin{figure}
\plotone{plots/radar_plots/v16_radar}
\caption{The science impact for the different version 1.6 simulations.}\label{fig:v16radar}
\end{figure}

%#################### Baseline #############################
\subsection{FBS 1.6 Baseline}\label{ss:1.6baseline}

\begin{figure}
\epsscale{.5}
\plotone{plots/pulled_plots/baseline_nexp1_v1_6_10yrs_Count_observationStartMJD_HEAL_SkyMap.pdf}
\plotone{plots/pulled_plots/baseline_nexp1_v1_6_10yrs_Nvisits_as_function_of_Alt_Az_HEAL_SkyMap.pdf}
\plotone{plots/pulled_plots/baseline_nexp1_v1_6_10yrs_Hourglass_year_0-1_HOUR_Hourglass.pdf} \\
\plotone{plots/aa_linear/baseline_nexp1_v1_6_N_obs_note_like_DD_HEAL_SkyMap.pdf}
\plotone{plots/aa_linear/baseline_nexp1_v1_6_N_obs_note_not_like_DD_HEAL_SkyMap.pdf}
\epsscale{1}
\caption{The baseline v1.6 simulation. The top panels show the distribution of visits (all filters) in RA/dec and Alt/Az. The middle panel shows the first year of observations color-coded by what filter was loaded. White regions represent scheduled and unscheduled downtime as well as weather downtime. The black curve on the bottom shows the moon phase. The bottom panels show the Alt/Az distribution of pointings for DDF observations (left) and non-DDF (right) on a linear stretch.}\label{fig:baseline1.6}
\end{figure}


For the baseline strategy, we set the footprint to have 18,000 square degrees dedicated the the WFD survey. The WFD has a filter distribution of $u:g:r:i:z:y$ of 0.31:0.44:1.0:1.0:0.9:0.9. 
% WFD sum = 4.55
We include coverage of the Galactic Plane (GP) and South Celestial Pole (SCP). These areas are set to have 20\% the number of counts of the WFD (if a spot in the WFD has 900 visits, points in the GP and SCP will have 180 visits). The GP and SCP are set to have equal number of visits in all filters.  The North Ecliptic Spur (NES) is observed with only the $g$, $r$, $i$, and $z$ filters. The NES area is set to have one-third the number of visits of the WFD.  The filter distribution in the NES is set to $g:r:i:z$ of 0.2:0.46:0.46:0.4. 

The total breakdown of target observing time is 85\% for WFD, 6\% for the NES, 6\% for the GP and NES, and 5\% for DDFs.


While the different survey areas are covered to different depths, the baseline scheduler treats them identically and only tries to maintain the proper ratios of area coverage. This means blocks of observations can be scheduled that cover the different regions seamlessly. It also means we have no additional constraints on how the regions are observed. For example, we currently do not reserve ``good seeing" time for the WFD area. 

The baseline survey includes the 4 announced Deep Drilling Fields as well as a pair of fields that overlap the Euclid Deep Field South.  Each individual DDF is set to take a maximum of 1\% of the total visits (the Euclid pair of fields are set to a maximum of 1\% combined). The standard DDF sequence is $u$x8, $g$x20, $r$x10, $i$x20, $z$x26, and $y$x20, all with 30s exposures. For any given sequence, only the five filters loaded in the camera are executed. By default, we remove the $u$ filter when the moon is more than 40\% illuminated at the start of the night. 


We run 2 baseline simulations, one with 1x30s visits and one with 2x15s visits.  The main difference is the additional readout time in the 2x15s version drops the open shutter fraction from 77\% to 72\%. This puts the 2x15s simulation close to failing the SRD FO metric, with some parts of the WFD region only reaching 824 observations (the median is still 892). 

For the rest of the simulations in v1.6 we use 1x30s visits.  If 2x15s visits are required there will be a significant drop in the number of visits, and areas outside of the WFD may need to be scaled back to still meet SRD requirements.

When it is non-twilight time and we are not observing DDFs, we use a Markov Decision Process to dynamically build a queue of observations.  Observations are planned in 44 minute blocks (22 minutes for an initial area, 22 minutes to repeat the area). The size of the blocks can scale slightly to try and fill time before twilight (e.g., it will expand to a pair gap of 25 minutes if there are 50 minutes until morning twilight begins). All observations are taken in pairs, with potential combinations of $u+g$, $u+r$, $g+r$, $r+i$, $i+z$, $z+y$, or $y+y$. The ordering of the filter pairs can change depending on what filter is currently loaded (e.g., if the scheduler decides to observe a $g+r$ sequence, the $r$ observations will be taken first to eliminate a filter change if possible.)

The camera rotator angle (relative to the telescope) is randomly set each night between -80 and 80 degrees. The angle is set when the block is scheduled, so there can be a few degrees of drift between when the rotator angle is computed and when the observation is actually taken.

The MDP uses basis functions based on
\begin{itemize}
    \item{The 5-sigma depth (for both filters in the pair being taken)}
    \item{The footprint uniformity (again, in both filters)}
    \item{The slewtime}
    \item{Staying in the current filter}
    \item{Rewards taking 3 observations per year per filter over the entire survey footprint} 
\end{itemize}
The MDP also includes basis functions that are simple masks
\begin{itemize}
    \item{Zenith is masked (to avoid long azimuth slews)}
    \item{30 degrees around the moon is masked}
    \item{The bright planets (Venus, Mars, and Jupiter) are masked with a 3.5 degree radius}
\end{itemize}


If the sun is higher than -18 degrees altitude, or there is not enough time remaining to take observations in pairs, the scheduler reverts to a greedy algorithm and selects observations one at a time. We use a similar MDP for these greedy twilight observation decisions. 

Compared to many of the other FBS 1.6 candidate baseline simulations, the baseline spends a lot of time observing the WFD, with a median of 948 visits. The higher number of visits means a faster cadence and better sampled lightcurves for objects with durations comparable to a season length. Our baseline simulation also has very light coverage of the Galactic bulge, resulting in fewer fast microlensing events than other potential footprints. 

%#################### DDF Heavy #############################
\subsection{DDF Heavy}\label{ss:1.6ddfheavy}

\begin{figure}
\epsscale{0.5}
\plotone{plots/pulled_plots/ddf_heavy_v1_6_10yrs_Count_observationStartMJD_HEAL_SkyMap}
\plotone{plots/pulled_plots/ddf_heavy_v1_6_10yrs_Nvisits_as_function_of_Alt_Az_HEAL_SkyMap}
\plotone{plots/pulled_plots/ddf_heavy_v1_6_10yrs_Hourglass_year_0-1_HOUR_Hourglass}
\epsscale{1}
\caption{DDF Heavy simulation. Nearly identical to the baseline, but giving as much time as possible to DDF observations.}\label{fig:ddfheavy}
\end{figure}


This run is nearly identical to the baseline, but gives a large fraction of time to the deep drilling fields. Each of the five DDFs takes between 2.4 and 2.9\% of the survey, with 13.4\% of all visits being used for DDF observations. The baseline has 4.6\% of visits used for DDFs.  This is enough time that the WFD area near the DDFs fails to reach 825 visits over 10 years, but the SRD requirement is formally still met because the median WFD point is observed 875 times.

As expected, the majority of non-DDF science cases suffer if we dedicate such a large fraction of time to the DDFs. It is worth noting that most metrics within MAF are not tailored for DDF purposes; this is an area that is missing science metrics.

%#################### Barebones #############################
\subsection{Barebones}\label{ss:1.6barebones}

\begin{figure}
\epsscale{0.5}
\plotone{plots/pulled_plots/barebones_v1_6_10yrs_Count_observationStartMJD_HEAL_SkyMap}
\plotone{plots/pulled_plots/barebones_v1_6_10yrs_Nvisits_as_function_of_Alt_Az_HEAL_SkyMap}
\plotone{plots/pulled_plots/barebones_v1_6_10yrs_Hourglass_year_0-1_HOUR_Hourglass}
\epsscale{1}
\caption{The barebones simulation covering just the WFD area as efficiently and deeply as possible.}\label{fig:barebones}
\end{figure}


The barebones simulation is not a viable survey strategy, but provides an extreme example where we focus exclusively on meeting the SRD requirements, with little optimization for science.

The survey footprint is restricted to the baseline 18,000 square degree WFD area only. Deep drilling fields are included, but capped at $\sim2.5$\% of the total visits. Visits in $u$ and $y$ are unpaired, while the rest of the filters are paired in the same filter. This results in very few filter changes in a night. 

There are a wide number of reasons why this would be a terrible survey strategy -- detected transients would have no color information, photometric uber-calibration could be difficult with the galactic plane gap, a lack of solar system object because the NES is not included, etc.  The main purpose is to show the scheduler can run very near the theoretical maximum for open shutter fraction, with this run reaching 80\%. Also, we can note the fONv MedianNvisits metric reaches 1155, which is 40\% higher than the SRD requirement of 825. This also implies that we can observe a maximum of $\sim115$\ WFD visits per year in the event we want to adjust the scheduler to attempt to catch up on the WFD progress. 

The total lack of bulge coverage means the barebones simulation contains virtually no fast microlensing events. Taking pairs in the same filter also radically reduces the number of SNe Ia that are well measured. 

%#################### Data Management Heavy #############################
\subsection{Data Management Heavy}\label{ss:1.6dmheavy}

\begin{figure}
\epsscale{0.5}
\plotone{plots/pulled_plots/dm_heavy_v1_6_10yrs_Count_observationStartMJD_HEAL_SkyMap}
\plotone{plots/pulled_plots/dm_heavy_v1_6_10yrs_Nvisits_as_function_of_Alt_Az_HEAL_SkyMap}
\plotone{plots/pulled_plots/dm_heavy_v1_6_10yrs_Hourglass_year_0-1_HOUR_Hourglass}
\epsscale{1}
\caption{The DM heavy simulation. Similar to the baseline, but the alt/az plot shows how some observations are being taken at high airmass to support DCR modeling.}\label{fig:dmheavy}
\end{figure}


This simulation is similar to the baseline, but includes various modifications that may be helpful for Data Management purposes. Across the WFD region, $u$, $g$, and $r$ a few images per year are taken at high airmass so that DCR correction models can be made. The camera rotator angle is set so that diffraction spikes fall along CCD rows and columns. This helps with difference imaging so the maximum possible area can be used, but may result in weak lensing systematics.  Each year, the scheduler prioritizes taking $g$, $r$, and $i$ images of the whole sky in good seeing conditions (defined as 0.7\arcsec effective FWHM or better).  The DDF fields use larger dithers, up to 1.5 degrees, compared to the default 0.7 degree maximum.

The addition of images taken at high airmass has a small negative impact on most science cases. 

%#################### Rolling Extragalactic #############################
\subsection{Rolling Extragalactic}\label{ss:1.6extragalactic}

\begin{figure}
\epsscale{0.5}
\plotone{plots/pulled_plots/rolling_exgal_mod2_dust_sdf_0_80_v1_6_10yrs_Count_observationStartMJD_HEAL_SkyMap}
\plotone{plots/pulled_plots/rolling_exgal_mod2_dust_sdf_0_80_v1_6_10yrs_Nvisits_as_function_of_Alt_Az_HEAL_SkyMap}
\plotone{plots/pulled_plots/rolling_exgal_mod2_dust_sdf_0_80_v1_6_10yrs_Hourglass_year_0-1_HOUR_Hourglass}
\epsscale{1}
\caption{The Rolling Exgal simulation. The WFD area is set to be 18,000 square degrees of low extinction area.}\label{fig:rollingexgal}
\end{figure}


\begin{figure}
\plottwo{plots/rolling_plot/baseline_nexp1_v1_6_Count_filter_note_not_like_DD_HEAL_SkyMap.pdf}{plots/rolling_plot/rolling_exgal_mod2_dust_sdf_0_80_v1_6_Count_filter_note_not_like_DD_HEAL_SkyMap.pdf}
\plottwo{plots/rolling_plot/baseline_nexp1_v1_6_Count_filter_night_gt_1278_375000_and_night_lt_1643_625000_and_note_not_like_DD_HEAL_SkyMap.pdf}{plots/rolling_plot/rolling_exgal_mod2_dust_sdf_0_80_v1_6_Count_filter_night_gt_1278_375000_and_night_lt_1643_625000_and_note_not_like_DD_HEAL_SkyMap.pdf}
\caption{Illustration of rolling cadence. The top panels show the number of observations after 10 years (all filters) for the Baseline and Rolling Exgal simulations (excluding DDF observations). Both simulations have very smooth WFD coverage, with $\sim$900 observations.  The lower panels show the number of observations taken between 3.5 and 4.5 years.  The baseline WFD remains smooth, while the Rolling Exgal simulation has declination stripes of high and low counts.}\label{fig:exgalroll}
\end{figure}


The rolling extragalactic is motivated by cosmological drivers. The footprint is modified so the 18,000 square degrees of the WFD are placed in low-extinction regions. The simulation also executes a half-sky rolling scheme, which should result in better sampled lightcurves for extragalactic transients.

This simulation divides the sky into quarters, and has one northern stripe and one southerns stripe with a rolling emphasis at a time. This could be preferable to a simple two-band rolling scheme, because with the quarters a region of emphasis will always be available to northern telescopes. If we rolled with an emphasis purely on the southern half of the WFD region, $\sim$80\% of the Rubin alert stream would become unavailable to northern hemisphere observatories for that season.

As expected, avoiding high extinction regions increases the number of galaxies. We expect the addition of rolling will show improvements in more sophisticated SNe Ia metrics from the community. The footprint covers some of the Magellanic Clouds, boosting the fast microlensing events.  The science gains come at the expense of some of the SRD metrics. 

%#################### Milky Way Heavy #############################
\subsection{Milky Way Heavy}\label{ss:1.6milkywayheavy}
\begin{figure}
\epsscale{0.5}
\plotone{plots/pulled_plots/mw_heavy_v1_6_10yrs_Count_observationStartMJD_HEAL_SkyMap}
\plotone{plots/pulled_plots/mw_heavy_v1_6_10yrs_Nvisits_as_function_of_Alt_Az_HEAL_SkyMap}
\plotone{plots/pulled_plots/mw_heavy_v1_6_10yrs_Hourglass_year_0-1_HOUR_Hourglass}
\epsscale{1}
\caption{The Milky Way heavy simulation. Similar to the Baseline, but the bulge and Magellanic Clouds are added to the WFD area. }\label{fig:mwheavy}
\end{figure}

The Milky Way heavy simulation covers the Galactic bulge, LMC, and SMC as part of the WFD area.  

There is very little change in the overall median coadded depths compared with the baseline since the extra WFD area is added to a region of the sky that is under-subscribed in the baseline.  In the baseline simulation, there are an excess of observations in the WFD on either side the galactic plane, so covering the bulge is ``free", in the sense that it uses these excess pointings to cover the bulge. 

There is a large boost in microlensing events and number of stars, with little impact on the other metrics. We would benefit from other metrics for bulge-specific science cases to explore using a different filter distribution for the bulge region.


%#################### Solar System Heavy #############################
\subsection{Solar System Heavy}\label{ss:1.6solarsystemheavy}
\begin{figure}
\epsscale{0.5}
\plotone{plots/pulled_plots/ss_heavy_v1_6_10yrs_Count_observationStartMJD_HEAL_SkyMap}
\plotone{plots/pulled_plots/ss_heavy_v1_6_10yrs_Nvisits_as_function_of_Alt_Az_HEAL_SkyMap}
\plotone{plots/pulled_plots/ss_heavy_v1_6_10yrs_Hourglass_year_0-1_HOUR_Hourglass}
\epsscale{1}
\caption{The Solar System heavy simulation. The high airmass observations are twilight NEO observations.}\label{fig:ssheavy}
\end{figure}

For the Solar System Heavy simulation, the baseline survey footprint is modified to include ecliptic plane coverage through the galactic plane.

A fraction of twilight time is used for a NEO survey in $r$ band. The NEO survey uses very short (1 second) exposures at high airmass (toward the sun in evening or morning twilight). This simulation only uses $i$, $z$, $y$ in the remainder of the twilight time, making sure we observe more $r$-band in non-twilight and in pairs. It also includes $r+r$ pairs in non-twilight time.  For regular 1x30s visit twilight observations, we avoid observing the ecliptic, thereby ensuring they are always taken in pairs in non-twilight time.

The simulation shows a slight improvement in the discovery of bright NEOs and TNOs, with a slight decrease in discovery of faint objects of all populations, while significantly impacting SNe Ia discovery due to the addition of pairs in the same filter. Solar system metrics, particularly for bright objects, are most sensitive to footprint (they tend to get enough visits to discover objects, so need to explore more area of sky) so a larger footprint (such as the big sky style with galactic plane and extended northern sky coverage) works better. For fainter objects, visits in redder filters and in the same filter are ideal; beyond that, more visits are important as the timing of discovery is more critical. The Solar System Heavy simulation adds enough twilight visits that the overall number of long exposure ($>1$s) visits in the traditional WFD footprint is reduced by about 6\%; this has an impact on the detection of faint objects. Further optimization toward solar system objects would likely produce a slightly different simulation to this, however this is a reasonable example of the trades with other science. 


%#################### Combo Dust #############################
\subsection{Combo Dust}\label{ss:1.6combodust}

\begin{figure}
\epsscale{0.5}
\plotone{plots/pulled_plots/combo_dust_v1_6_10yrs_Count_observationStartMJD_HEAL_SkyMap}
\plotone{plots/pulled_plots/combo_dust_v1_6_10yrs_Nvisits_as_function_of_Alt_Az_HEAL_SkyMap}
\plotone{plots/pulled_plots/combo_dust_v1_6_10yrs_Hourglass_year_0-1_HOUR_Hourglass}
\epsscale{1}
\caption{The Combo Dust simulation. Similar to the Rolling Exgal simulation, but the WFD is expanded to include the bulge and ecliptic, Magellanic Clouds, and an anti-center bridge.  }\label{fig:combodust}
\end{figure}


This simulation attempts to improve several science cases compared to the baseline simultaneously. The footprint used here starts with defining the WFD area as 18,000 square degrees with low extinction. Then an additional 2,000 square degrees are added to WFD to cover the bulge, the ecliptic through the galactic plane, the LMC and SMC, and an outer Galactic plane region. Dusty areas of the sky and the South Celestial Pole are covered at about one-quarter the WFD depth. The NES is covered in $g$, $r$, $i$, and $z$. The footprint also includes very light coverage to the northern limit of the telescope in $g$, $r$, and $i$ so there can be templates for ToO events on the entire accessible sky. This simulation includes the same half-footprint rolling scheme as Rolling Extragalactic.

The footprint has 35 free parameters for setting the various region locations and filter ratios. Many of these have have been set by eye or use historical values; it is quite likely these parameters could be improved. 

This simulation manages to boost nearly all the science metrics at the expense of reducing margin in the SRD metrics. When we run the $combo\_dust$ with 2x15s visits, the fO metric drops below the SRD requirement of 825 visits to 817 visits. The footprint can be adjusted to meet the SRD requirement, but it does imply there will be very little contingency if we use 2-snap visits. The 1x30s visit $combo\_dust$ has a median of 885 visits in the WFD region, meeting SRD requirements.

\begin{table}
\begin{centering}
\begin{tabular}{lrrrrrrrrr}
\toprule
filter &  Baseline &  Baseline &  Barebones &  DDF  &  DM  &  MW  &  Rolling  &  SS  &  Combo  \\
 & & 2 snaps & & Heavy & Heavy & Heavy & Exgal & Heavy & Dust \\
 & (mags) &  \multicolumn{8}{c}{$m_{\rm{Baseline}} - m_{\rm{Sim}}$} \\ 
\hline
     u &     25.86 &              0.24 &      -0.13 &       0.08 &      0.11 &      0.02 &           0.11 &     -0.02 &        0.12 \\
     g &     26.97 &              0.11 &      -0.15 &       0.09 &      0.12 &      0.01 &           0.10 &      0.07 &        0.14 \\
     r &     26.95 &              0.08 &      -0.12 &       0.08 &      0.07 &      0.01 &           0.10 &      0.05 &        0.14 \\
     i &     26.40 &              0.07 &      -0.17 &       0.11 &     -0.01 &      0.01 &           0.11 &      0.11 &        0.15 \\
     z &     25.67 &              0.06 &      -0.12 &       0.08 &     -0.01 &      0.01 &           0.11 &      0.02 &        0.11 \\
     y &     24.90 &              0.06 &      -0.14 &       0.06 &      0.04 &      0.01 &           0.09 &      0.03 &        0.09 \\
\end{tabular}
\caption{Difference in median coadded five sigma depths compared to Baseline for v1.6 simulations. Negative values indicate deeper depths.}\label{table:depths}
\end{centering}
\end{table}

%\begin{figure}
%\plotone{plots/radar_plots/v16_radar.pdf}
%\caption{Science impact for the v1.6 runs. The barebones survey performs very well on SRD requirements at the expense of almost all other science. The combo\_dust %run is the opposite, with low SRD scores and high science performance.}\label{fig:v16radar}
%\end{figure}



\section{Individual Visit Length}
What to do - 1x30s vs. 2x15s? 1x30s much more efficient (show rough calculation of overhead) than 2x15s, but may have drawbacks due to cosmic ray rejection and potential to miss very rapid transients (or WD detection .. ref white paper). Subtle drawback that 2x15s gives the same "midpoint exposure time" across FOV, 1x30s does not. 

Show difference in 1x30s vs. 2x15s in whatever is our 'standard baseline' at this point. 

There has been thought of using a variety of exposure times if we use two snaps (e.g., 5s + 25s). Because there are not plans to release catalogs from individual snaps, it's not clear if this would enable much new science.

Show effect of 7\% loss in efficiency when attempting to combine minisurveys in various configurations (assume we will find some combinations possible with single exposure visits that are impossible with two snaps). 

Also possible to use variable exposure time depending on seeing and sky brightness conditions. Shorter exposures in good conditions keeps us from observing ``wasted" depth, letting us take longer exposures in poor conditions. This does introduce a host of new free parameters (an ideal target depth for each filter and minimum and maximum exposure times).  This would might require rewording the SRD to ensure, e.g., that 20s visits in good conditions count for the number of visit requirement.

Relevant metrics: total number of visits, number of visits per field/filter

\section{Intra-night Cadence}
What to do for visit sequence within a night? White paper support for multiple filters within a night (except TNOs maybe?). Potential drawbacks - less efficient (show effect on efficiency). This applies to WFD primarily, but we've applied to any survey that did not have their own specifications (so, everywhere). 

Extension of pairs to $u$ band and $y$ band (show effect). 

Relevant metrics: inter-night visit gaps and SN discovery, SSO discovery/characterization, transient and variable discovery (??), number of visits

\section{Wide-Fast-Deep Footprint}
What to do for WFD footprint? SRD not specific, DESC want low-extinction sky (and depth), but WFD is generally the area of sky that receives the most visits, so generally other science will also benefit from more visits to their relevant areas (particularly galactic plane .. for time-domain studies primarily, not depth)

Relevant metrics: area of sky with 825 visits (under particular restrictions, like total coadded depth and individual image seeing and dust extinction), number of galaxies, number of resolved galaxies, SSO discovery, transient and variable star discovery, astrometry in the galactic plane (?)

\section{Dithering}

The spatial dithering strategy for the large area surveys seems to work well.

We need to decide on a rotational dithering strategy. Should we try to randomize, or keep diffraction spikes aligned along rows and columns?

The dithering strategy for the DDFs pose a tension. DDF science is best served by small spatial dithers, but DM would prefer large spatial dithers. Also need to do something unique for the Euclid pair of fields, and any fields with nearby bright stars.


\section{Rolling cadence}
Motivation for a rolling cadence (more frequent visits in some years)

Different options for rolling and explanation of how implemented

Should really include discussion of recovery from bad weather years and simulation of same

Relevant metrics: Maintain astrometry requirements, SN discovery, SSO discovery and characterization,  Transient and variable discovery, uniformity of coadded depth / number of visits, 

\section{Northern minisurveys}
Add extension to cover Euclid/DESI with various numbers of visits

Observing NES 

Effect of adding or removing these minisurveys

Relevant metrics: SSO discovery and characterization (particularly active asteroids), depth and number of visits through remainder of North

\section{Southern minisurveys}
Add extension over south celestial pole, LMC/SMC with various numbers of visits

Effect of adding or removing these minisurveys

Relevant metrics: number of visits and coadded depth over SCP, discovery of variables in LMC/SMC (see Olsen white paper for metrics?)

\section{Low Galactic Latitudes}
Discussion of definitions from SAC and recommendations for visits

Effect of adding or removing these minisurveys

Relevant metrics: number of visits, astrometry in bulge, discovery of variables/transients/microlensing in bulge (?)

\section{Twilight Observing}
Discuss need for twilight observing to meet SRD goals (weather, total amount of time available)

Add NEO twilight survey, add DCR white paper (season extension visits?)

Effect of adding or removing these minisurveys

Relevant metrics: NEO discovery, number of visits and coadded depth (and uniformity) in WFD, measurement of DCR, season length

\section{Deep Drilling Fields}
Discuss purpose and how these are scheduled (very different from other fields)

Discuss potential cadences (AGN/ DESC) and how these differ, and our combination of the two

Discuss timing issues with oversubscription (and how much of a problem this could be, what if worse weather?) -- include location of fifth DD field

Effect of adding or removing these minisurveys

Relevant metrics: number of visits and coadded depth for DD, SN detection in DDFs, AGN detection in DDFs
*[solar system minisurvey DDF?]

\section{ToO modes}
Discuss impact of ToO, and how we could implement ToOs in scheduler (various modes: straight to queue by hand or set up known program and supply trigger, etc. -- that we're evaluating the second?)

Any ToO survey should also take into account that chip and raft gaps mean full sky coverage will require multiple images with spatial dithering.

Discuss how we can have a low coverage region to the north to maintain templates for all possible ToOs, or we could decide ot only search for ToOs that are likely to be in the WFD area.

Relevant metrics: frequency of achieving ToO observations, number of visits and coadded depth in other surveys (WFD or other minisurveys that may be in particular contention)

\section{Making it all work}
Discuss combinations of the above that work together or don't 

Relevant metrics: all

\section{Optimizing parameters}
Somewhere in here we probably ought to talk about optimizing the parameters for each run, and doing bigger sweeps across parameter space. That would easily expand each of the above options by many factors.

\section{Conclusions}
Hopefully here we pare down the evaluation of 100s of runs (like promised) to a set of between 10 to 20 (if this is possible, after combining along different axes). 
The results should come with some basic comments about what's particularly good or bad in each of these areas and how we arrived at these general options. 


% Make sure lsst-texmf/bin/generateAcronyms.py is in your path
\section{Acronyms} \label{sec:acronyms}
\input{acronyms.tex}
