\section{Introduction}

Vera C. Rubin Observatory (Rubin) will carry out the Legacy Survey of Space and Time (LSST) over the first ten years of its lifetime. The LSST is intended to meet four core science goals:
\begin{itemize}
\item constraining dark energy and dark matter
\item taking an inventory of the Solar System
\item exploring the transient optical sky, and
\item mapping the Milky Way.
\end{itemize}

The basic requirements for these goals are described in the LSST Science Requirements Document (\href{http://ls.st/srd}{SRD}\footnote{The LSST Science 
Requirements Document (SRD) is available at \href{http://ls.st/srd}{http://ls.st/srd}}). In practice, the SRD intentionally places minimal quantitative constraints 
on the observing strategy, primarily requiring:
\begin{itemize} 
\item A footprint for the `main survey' of at least 18,000 deg$^2$, which must be uniformly covered to 
a median of 825 30-second visits per 9.6 deg$^2$ field, summed over all six filters, $ugrizy$ (see SRD 
Tables 22 and 23). This places a minimum constraint on the time required to complete 
the main survey. Simulated surveys indicate that the main survey typically requires 80--90\% 
of the available time (10 years) to reach this benchmark; even with scheduling improvements, it is unlikely 
that the goals of the main survey could be met with a time allocation significantly below 80\%. 
\item Parallax and proper motion $1\sigma$ accuracies of 3~mas and 1~mas/yr per coordinate at $r=24$, 
respectively, in the main survey (see SRD Table 26), which places
a weak constraint on how visits are distributed throughout the lifetime of the survey and throughout a season.
\item Rapid revisits (40 seconds to 30 minutes) must be acquired over an area of at least 2000 deg$^2$ (see SRD table 25) for
very fast transient discovery; this requirement can usually be satisfied via simple field overlaps when surveying contiguous areas of sky. 
\end{itemize}
This leaves significant flexibility in the detailed cadence of observations within
the main survey footprint, including the distribution of visits within a year (or between seasons), the distribution between filters and 
the definition of a `visit' itself. Furthermore, these constraints apply to the main survey; the use of the 
remaining time (i.e., in mini surveys) is not constrained by the SRD.

In order to maximize the overall science impact of the LSST, in 2018 the project issued a \href{ http://ls.st/doc-28382}{call for white papers} requesting survey strategy input\footnote{The call for white papers is available at \href{ http://ls.st/doc-28382}{http://ls.st/doc-28382}}. The 46 \href{https://www.lsst.org/submitted-whitepaper-2018}{submitted white papers} represent a wide swath of the astronomical community, and work together with the \href{https://github.com/LSSTScienceCollaborations/ObservingStrategy/raw/master/whitepaper/releases/LSST_Observing_Strategy_White_Paper_v1.0.pdf}{Community Observing Strategy Evaluation Paper (COSEP)}\footnote{The github repository containing the living source for the COSEP is \href{https://github.com/LSSTScienceCollaborations/ObservingStrategy}{https://github.com/LSSTScienceCollaborations/ObservingStrategy}} to shape the next stage of the survey strategy evaluation. The contents of these white papers were distilled into several areas for investigation by the LSST Science Advisory Council (SAC) in their advisory \href{http://ls.st/doc-32816}{response} to the project\footnote{The SAC white paper report is available at \href{http://ls.st/doc-32816}{http://ls.st/doc-32816}}.

This survey strategy optimization work is starting from an existing candidate baseline strategy, driven by the basic science goals. A brief introduction to the baseline survey strategy, expanded background of the primary LSST science goals, and concise descriptions of how these goals drive the basic survey strategy and data processing requirements are provided in the \href{http://ls.st/lop}{LSST Overview paper}\footnote{The LSST Overview paper is a living document available at \href{http://ls.st/lop}{http://ls.st/lop}.}. A reference survey simulation (baseline2018a), generated by an earlier version of the LSST survey simulation tools (see Section~\ref{section:simulator}), provided an implemented example of this strategy. This starting point for the survey strategy can be described extremely briefly as follows:
\begin{itemize}
\item The {\bf main ``wide-fast-deep''  (WFD) survey}, which covers $\sim$18,000 deg$^2$ of sky within
the equatorial declination
range $-62^\circ < \delta < +2^\circ$, and excluding the central portion of the Galactic 
plane. Within the main survey, two visits\footnote{A `visit' here is an LSST default visit, which 
consists of two back-to-back 15 sec exposures, for a total of 30 sec of on-sky exposure time. These back-to-back exposures are always
in the same filter, separated only by the 2 second readout time.} 
per 9.6 deg$^2$ field (in either the same or different filters) are acquired in each night, to allow identification of moving objects and rapidly varying transients, and to improve
the reliability of the alert stream. These pairs of visits are repeated every three to four nights throughout the period the field is visible in each year (other nights are used to maximize the sky coverage). Each
field in the main survey receives about 825 visits throughout the ten years of the LSST, spread over the six LSST filters 
$ugrizy$. The quantitative SRD constraints on area coverage, number of visits, parallax and proper motion errors, and 
rapid-revisit rate (40 seconds -- 30 minutes) apply to visits obtained in the main survey. 
\item The set of five {\bf Deep Drilling Field candidate mini surveys}, consisting of five specific field pointings for a total of $\sim$ 50 deg$^2$, 
which are observed with a much denser sampling rate. These mini surveys use a similar sequence of visits; the fields
are observed every three to four days, but in a sequence of multiple $grizy$ exposures during gray and bright time, and then
multiple sequential $u$ band exposures during dark time. The current deep drilling mini survey fields are aimed at extragalactic
science, providing a `gold sample' to calibrate the main survey, and to discover Type Ia supernovae. 
\item The {\bf Galactic Plane candidate mini survey} covers the central portion of the Galactic plane that is not included in the main survey, 
centered around $|l| = 0^\circ$ and covering $\sim$ 1860 deg$^2$.  It is observed at a much reduced rate compared to the main survey, 
and with a smaller total number of observations per field (30 visits per field and per filter, in $ugrizy$), so as to
provide astrometry and photometry of stars toward the Galactic center but without reaching the confusion limit in the coadded images.
There is no requirement for pairs of visits in each night in this area.
\item The {\bf North Ecliptic Spur candidate mini survey} covers the area north of $\delta = +2^\circ$ to $10^\circ$ north of the Ecliptic plane
and is intended to observe the entire Ecliptic plane for the purpose of inventorying the minor bodies in the Solar System. This area ($\sim$ 4160 deg$^2$) 
is observed on a schedule similar to the main survey, although with a smaller total number of visits per field and only in filters $griz$. 
\item The {\bf South Celestial Pole candidate mini survey} covers the region south of the main survey, to the South Celestial Pole, $\sim$ 2315 deg$^2$,
including the Magellanic Clouds. 
This mini survey is observed with a strategy similar to the Galactic Plane mini survey, with 30 visits per field per filter in $ugrizy$, 
and without requiring pairs of visits. This provides coverage of the Magellanic clouds, but without committing extensive time as these fields are
at high airmasses from the LSST site.
\end{itemize}

This report covers the LSST Survey Strategy team's experiments with the LSST scheduler to address the optimization questions raised by the SAC. These questions include:
\begin{itemize}
\item How should the WFD footprint be defined?
\item What should the cadence of visits within the WFD look like? This includes both the intra-night cadence and the inter-night cadence throughout the season.
\item What is the impact of varying the footprint for mini-surveys?
\item Can we leverage twilight observing?
\item How should the Deep Drilling fields be distributed and what cadence should be used for their observations?
\item What are the impact of ToO proposals, particularly gravitational wave followup?
\end{itemize}

These questions are aimed at ensuring the best possible science return from the LSST. 

