
\section{Simulation Analysis}

Here we present the metrics we use to illustrate the performance of different simulations. By nesessity, we use a subset of all the metics we typically run.

\subsection{Parallax}


\subsection{Proper Motion}

\subsection{fO}

\subsection{SNe Ia}

\subsection{Tidal Disruption Events (TDE)}

\subsection{Weak Lensing}

Contributed by DESC.

\subsection{3x2 point Figure of Merit}

Contributed by DESC.

\subsection{Fast Microlensing}

Light curves contributed from the community.

While we also have a slow microlensing metric, we find very little variation over different siumulations.

Because the baseline footprint has sparse coverage of the Galactic bulge, the baseline value of this metric is relatively small, making the normalized values more volitile than the other metrics here.

\subsection{Number of Galaxies}

\subsection{Number of Stars}

\subsection{Bright Near Earth Objects}

\subsection{Faint Near Earth Objects}

\subsection{Trans Neptunian Objects}

\subsection{Radar Plots}

Explain that the radar plots have values normalized (typically to a relevant baseline run). For the parallax and proper motion metrics, the inverse of the errors are compared. For magnitudes, we plot madnitude difference (with larger values indicating deeper depths).

XXX--Maybe put in a table of the raw values for the 1.5 and 1.6 baseline runs.