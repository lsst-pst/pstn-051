\subsection{Dithering}

\subsubsection{Rotational dithering}

By default, we select a random camera rotation angle (wrt the telescope) nightly. This creates minimal additional slewtime, and seems to provide adequate angular randomization.  We currently have no science metrics that depend on the angular distribution, and this should be something very important to weak lensing science (although we do not have a metric to measure this).

We have also experimented with setting the camera rotation angle to ensure stellar diffraction spikes fall preferentially along rows and columns. 

\begin{itemize}
    \item{How should we rotationally dither visits?}
\end{itemize}

Need metrics sensitive to the rotation angle of the camera. 

\subsubsection{Spatial Dithering}

For the wide area regions we have had excellent results randomizing the tessellation orientation nightly. This does result in a small percent of time being spent observing outside the desired survey footprint. The alternative would be to limit the amount one dithers out of the footprint, but then one risks imprinting systematics on objects near the footprint border (e.g., an object is never observed in the center of the focal plane, only by outer rafts).

Primary metric is survey uniformity?


\subsection{Image Differencing Templates, DCR}

Do we need to do anything special to ensure we have adequate image templates? A certain number of observations per year? A certain fraction of images taken in good seeing conditions? 

If we need to start considering image quality, that makes it more difficult to simulate a night ahead of time and maintain the list of upcoming observations.

Should we intentionally extend to high airmass to facilitate DCR modeling for templates? Note that in the baseline, we only image a location in the WFD region $\sim$9 times per year in $g$ and $\sim$6 times in $u$. Also, we have chip and raft gaps, so if we want to build a DCR model for the entire sky in $g$, we might be dedicating 1/3 of the $g$ observations in a year to DCR. If we switch to 60s $u$ band exposures, there would be no observations beyond building the DCR model. (We don't know that DM needs this).

There have been claims that measuring DCR can be used for science.  We do not have any metrics that demonstrate any gains, and the loss of depth is noticeable. In theory, we could combine the DCR measurements to extend the season length of observations as well (e.g., only take DCR template images near twilight in the direction of the sun).


\subsection{Satellite Megaconstellations}

Starlink (followed by Amazon) is poised to launch thousands of LEO satellites. Observations so far imply that final-orbit Starlink satellites should not saturate Rubin exposures, and thus can be masked fairly easily in the image reduction pipeline. 

Do we need any further satellite mitigations? Will NEO twilight surveys still be viable in the presence of megaconstellations, or should we use twilight strategies that avoid the horizon?

Figure~\ref{fig:megasat} shows how illuminated megaconstellations in LEO would leave numerous streaks on Rubin images.

% from https://github.com/yoachim/satellite_collisions
\begin{figure}
\label{fig:megasat}
\plottwo{plots/sat_plots/ten_min_12k.pdf}{plots/sat_plots/tenmin_example.pdf}
\caption{Alt/az projection of simulated satellite megaconstellations as seen from the Rubin Observatory site after twilight has ended. } 
\end{figure}


