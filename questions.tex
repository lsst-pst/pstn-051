
\section{Outstanding Questions}\label{sec:questions}

Here we go through some of the outstanding questions that the SCOC and scientific community can help resolve in order to converge on a final scheduler strategy for the Rubin Observatory. 

\subsection{Exposure Time(s)}

We will probably need on-sky data to make a final answer to this question, but we need to eventually decide how many snaps to take in a visit.

XXX--1 or two snaps
XXX--variable exposure time?

xxx-Should we change the u-band to default to 60 second exposures to ensure they are not readnoise dominated? 

xxx--should we include some very short exposure time exposures. That would let us have better tie-in with other surveys (e.g., Gaia).  It is relatively little exposure time, but the readout time means it is a low-efficieny way to operate the telescope.

Should we decrease the exoposure time in twilight to keep the saturation level reasonable?

\subsection{Survey Contingency}

How much continency should we aim for when designing the survey strategy?  Currently, with what we believe is a conservative weather closure policy, we can meet SRD requirements with 2x15s visits, but can cover a larger footprint and do more science cases with 1x30s snaps.  

\subsection{Deep Drilling Fields}

We have run a number of Deep Drilling strategies. The DDF strategy is largely seperable from the rest of the survey design.

\begin{itemize}
    \item{What fraction of the survey should be dedicated to the DDFs?}
    \item{Should DDFs be preferntially executed in dark time, or is it more important to maintain cadence?}
    \item{Where should the DDFs be placed (can we finalize the 5th DDF as a Euclid double-pointing DDF)?}
    \item{What is the prefered dithering strategy (spatially and rotationally) for the DDFs? There is tension in that DM generally prefers larger dithers for calibration and co-addition purposes, while science cases prefer smaller dithers to preserve the area that reaches the deepest levels.}
    \item{Should we try ``rolling" the DDFs, completing DDF observations in a field in only a few years?}
\end{itemize}

\subsection{Rotational Dithering}

By default, we select a random camera rotation angle (wrt the telescope) nightly. This creates minimal additional slewtime, and seems to provide adequate angular randomization.  We currently have no science metrics that depend on the angular distribution, and this should be something very important to weak lensing science.

We have also experimented with setting the camera rotation angle to ensure stellar diffraction spikes fall preferentially along rows and columns. 

\subsection{Survey Footprint}

Perhaps the biggest question, what should we set the final survey footprint to be?

\begin{itemize}
    \item{How should we cover the galactic plane?}
    \item{How should we observe the bulge?}
    \item{Should we avoid areas of high dust extinction for the WFD area?}
    \item{What is the ideal filter distribution to use? It would be nice to have a photo-z metric to help make this decision.}
    \item{Should we cover the LMC and SMC as part of the WFD survey? As their own DDF-like survey?}
\end{itemize}

\subsection{Best Use of Twilight Time}

Our baseline simulation uses twilight time to fill in observations in redder filters ($rizy$). We can use some of the time to conduct a NEO survey. We can also vary which filters get used in twilight time. The baseline greedy algorithm used in twilight is known to be rather unstable, so we could also try running more contiguous blocks in twilight. We could also emphasize targeting areas that have already been observed 4 or more times in the night, potentially gathering important color information for a small number of transients.


\subsection{Target of Opportunity}

Currently, the only expected ToO use of Rubin observatory is followup of gravitational wave detections.

\begin{itemize}
    \item{When should Rubin inturrupt observations to look for GW optical counterparts?}
    \item{When the detection is in the WFD area, or anywhere on the sky?}
    \item{Should we expand the survey footprint so we have image differencing templates over the entire accessible sky, in at least a few filters?}
    \item{Should Rubin plan on observing the entire light curve of ToO events, or make observations that allow for detection and leave detailed follow up to other observatories?}
    \item{What filter combination and dither strategy should be used for observing ToO triggers?} 
\end{itemize}


\subsection{Image Differencing Templates}

Do we need to do anything special to ensure we have adequate image templates? A certain number of observations per year? A certain fraction of images taken in good seeing conditions?

Should we intentionally extend to high airmass to facilitate DCR modeling? Note that in the baseline, we only image a location in the WFD region $\sim$9 times per year in $g$ and $\sim$6 times in $u$. Also, we have chip and raft gaps, so if we want to build a DCR model for the entire sky in $g$, we might be dedicating 1/3 of the $g$ observations in a year to DCR. If we swtich to 60s $u$ band exposures, there would be no observations beyond building the DCR model. 

\subsection{Satellite Megaconstellations}

xxx -- starlink is poised to launch thousands of LEO satellites. Observations so far imply that on orbit Starlink satellites should not saturate Rubin exposures, and thus can be masked fairly easily in the image reduction pipeline. 

Do we need any further satellite mitigations? Will NEO twilight surveys still be viable in the presence of megaconstellations, or should we use twilight strategies that avoid the horizon?

