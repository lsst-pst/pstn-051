
\section{FBS release v1.6}

Here we desciribe the runs done as part of the FBS 1.6 release.  Unlike previous releases, here we do a select few releases that combine various aspects of previous experiements.



\subsection{Barebones}

The barebones simulation is an example of a survey where we focus exclusively on meeting the SRD requirements, with little optimization for science.

The survey footprint is restricted to the standard WFD area only. Deep drilling fields are included, but capped at $\sim2.5$\% of the total visits. Visits in u and y are unpaired, while the rest of the filters are paired in the same filter. This results in very few filter changes in a night.

There are a wide number of reasons why this would be a terrible survey strategy--detected transients would have no color information, photometric callibration would be difficult with the galactic plane gap, etc.  The main purpose it to show the scheduler can reach very near the theoritical maximum for open shutter fraction, with this run getting to 80\% OSF. Also, we can note the fONv metric reaches 1,151 which is 40\% higher than the SRD requirement on 825.


\subsection{Baseline}

We run 2 baseline simulations, one with 1x30s visits and one with 2x15s visits.  The main difference is the additional readout time in the 2x15 verion drops the open shutter fraction from 77\% to 72\%. This puts the 2x15s simulation incredibly close to failing the SRD metric, with some parts of the WFD region only reaching 824 observations (the median is still 892). 

For the rest of the runs, we use 1x30s visits, but note that if 2x15s is required there will be a significant drop in the number of visits, and areas outside of the WFD may need to be scaled back to still meet SRD requirements.

The baseline surveys use the following strategies:

XXX-DDF strategy:


XXX--main darktime strategy: 

Observations are taken in 44 minute blocks (22 minutes for an initial area, 22 minutes to repeat the area). The size of the blocks can scale slightly to try and fill time before twlight (e.g., it will expand to a pair gap of 25 minutes if there are 50 minutes until twilight). All observations are taken in pairs, with potential combinations of u+g, u+r, g+r, r+i, i+z, z+y, or y+y. The order of observations can change deppending on what filter is currently being used (e.g., if the scheduler decides to observe a g+r seqeunce, the r observations will be taken first to eliminate a filter change.)

The camera rotator angle (relative to the telescope) is randomly set each night between -80 and 80 degrees.  XXX-the angle is set when the block is scheduled, so there can be a few degrees of drift between when the rotator angle is computed and when the observation is actually taken.

The basis functions used are:

The 5-sigma depth (for both filters in the pair being taken), the footprint uniformity (again, in both fitlers), the slewtime, and a basis fucntion that rewards staying in the current filter.  We also include a basis function that rewards taking 3 observations per year per filter over the entire survey footprint.  

The zenith is masked (to avoid long azimuth slews), and a region 30 degrees around the moon is masked. The bright planets (Venus, Mars, and Jupitor) are masked with a 3.5 degree radius. 

XXX--twilight strategy

\subsection{DDF Heavy}

This run is very similar to the baseline, but gives a large fraction of time to the deep drilling fields.

g,r,i images of the whole sky in good seeing

large dithers for DDF fields

\subsection{Data Management Heavy}

This is simulations exercises various strategies that may be helpful for Data Management purposes. 

For the WFD region in u,g, and r a few images per year are taken at high airmass so that DCR correction models can be made.

The camera rotator angle is set so that diffraction spikes fall along CCD rows and columns. This helps with difference imaging so the maximum possible area can be used.

\subsection{Milky Way Heavy}

The Milky Way heavy simulation covers the bulge, LMC, and SMC as part of the WFD area.  

\subsection{Rolling Extragalactic}

Use a footprint that de-emphasizes the dusty plane of the galaxy. Use a rolling cadence strategy on the WFD area to generate more densly sampled light curves. Standard DDF strategy.

\subsection{Solar System Heavy}

Include ecliptic coverage through the galactic plane

NEO survey at twilight

only i,z,y in twilght

include r+r pairs
