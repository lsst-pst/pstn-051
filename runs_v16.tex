
\section{FBS release v1.6}

Here we describe the runs done as part of the FBS 1.6 release.  Unlike previous releases, here we do a select few releases that combine various aspects of previous experiments.


%#################### Baseline #############################
\subsection{Baseline}

\begin{figure}
\epsscale{.5}
\plotone{plots/pulled_plots/baseline_nexp1_v1_6_10yrs_Count_observationStartMJD_HEAL_SkyMap.pdf}
\plotone{plots/pulled_plots/baseline_nexp1_v1_6_10yrs_Nvisits_as_function_of_Alt_Az_HEAL_SkyMap.pdf}
\plotone{plots/pulled_plots/baseline_nexp1_v1_6_10yrs_Hourglass_year_0-1_HOUR_Hourglass.pdf}
\epsscale{1}
%XXX ra dec map, alt az, an hourglass
\caption{The baseline v1.6 simulation. The top panels show the distribution of visits (all filters) in RA/dec and Alt/Az. The bottom panel shows the first year of observations color-coded by what filter was loaded. White regions represent scheduled and unscheduled downtime as well as weather downtime. The black curve on the bottom shows the moon phase.}\label{fig:baseline1.6}
\end{figure}


For the baseline strategy, we set the footprint to have 18,000 square degrees dedicated the the WFD survey. The WFD has a filter distribution of u:g:r:i:z:y of 0.31:0.44:1.0:1.0:0.9:0.9.
% WFD sum = 4.55

We include coverage of the Galactic Plane (GP) and South Celestial Pole (SCP). These areas are set to have 20\% the number of counts of the WFD (if a spot in the WFD has 900 visits, points in the GP and SCP will have 180 visits). The GP and SCP are set to have equal number of visits in all filters.

The total breakdown of target observing time is 85\% for WFD, 6\% for the NES, 6\% for the GP and NES, and 5\% for DDFs.

The North Ecliptic Spur (NES) is observed with only the g, r, i, and z filters. The NES area is set to have one-third the number of visits of the WFD.  The filter distribution is set to g:r:i:z of 0.2:0.46:0.46:0.4. 

While the different survey areas are covered to different depths, the baseline scheduler treats them identically and only tries to maintain the proper ratios of area coverage. This means blocks of observations can be scheduled that cover the different regions seamlessly. It also means we have no additional constraints on how the regions are observed. For example, we currently do not reserve ``good seeing" time for the WFD area. 

The baseline survey includes the 4 announced Deep Drilling Fields as well as a pair of fields that overlap the Euclid Deep Field South. (XXX--maybe put in a table?). Each individual DDF is set to take a maximum of 1\% of the total visits (the Euclid fields are set to take 1\% combined). The DDF sequence is ux8, gx20, rx10, ix20, zx26, and yx20, all with 30s exposures. For any given sequence, only the five filters loaded in the camera are executed. By default, we remove the u filter when the moon is more than 40\% illuminated at the start of the night.


We run 2 baseline simulations, one with 1x30s visits and one with 2x15s visits.  The main difference is the additional readout time in the 2x15 version drops the open shutter fraction from 77\% to 72\%. This puts the 2x15s simulation close to failing the SRD FO metric, with some parts of the WFD region only reaching 824 observations (the median is still 892). 

For the rest of the simulations in v1.6 we use 1x30s visits.  If 2x15s is required there will be a significant drop in the number of visits, and areas outside of the WFD may need to be scaled back to still meet SRD requirements.

The baseline surveys use the following strategies:

XXX-DDF strategy:


XXX--main non-twilight time strategy: 

Observations are taken in 44 minute blocks (22 minutes for an initial area, 22 minutes to repeat the area). The size of the blocks can scale slightly to try and fill time before twilight (e.g., it will expand to a pair gap of 25 minutes if there are 50 minutes until twilight). All observations are taken in pairs, with potential combinations of u+g, u+r, g+r, r+i, i+z, z+y, or y+y. The order of observations can change depending on what filter is currently loaded (e.g., if the scheduler decides to observe a g+r sequence, the r observations will be taken first to eliminate a filter change if possible.)

The camera rotator angle (relative to the telescope) is randomly set each night between -80 and 80 degrees.  XXX-the angle is set when the block is scheduled, so there can be a few degrees of drift between when the rotator angle is computed and when the observation is actually taken.

The basis functions used are:

The 5-sigma depth (for both filters in the pair being taken), the footprint uniformity (again, in both filters), the slewtime, and a basis function that rewards staying in the current filter.  We also include a basis function that rewards taking 3 observations per year per filter over the entire survey footprint.  

The zenith is masked (to avoid long azimuth slews), and a region 30 degrees around the moon is masked. The bright planets (Venus, Mars, and Jupiter) are masked with a 3.5 degree radius. 

XXX--twilight strategy
If the sun is higher than XXX degrees, or there is not enough time remaining to take observations in pairs, 

%#################### DDF Heavy #############################
\subsection{DDF Heavy}

\begin{figure}
\epsscale{0.5}
\plotone{plots/pulled_plots/ddf_heavy_v1_6_10yrs_Count_observationStartMJD_HEAL_SkyMap}
\plotone{plots/pulled_plots/ddf_heavy_v1_6_10yrs_Nvisits_as_function_of_Alt_Az_HEAL_SkyMap}
\plotone{plots/pulled_plots/ddf_heavy_v1_6_10yrs_Hourglass_year_0-1_HOUR_Hourglass}
\epsscale{1}
\caption{DDF Heavy simulation. Nearly identical to the baseline, but giving as much time as possible to DDF observations.}\label{fig:ddfheavy}
\end{figure}


This run is nearly identical to the baseline, but gives a large fraction of time to the deep drilling fields. Each of the five DDFs takes between 2.4 and 2.9\% of the survey, with 13.4\% of all visits being used for DDF observations. The baseline has 4.6\% of visits used for DDFs.  This is enough time that the WFD area near the DDFs fails to reach 825 visits over 10 years, but the SRD requirement is formally still met because the median WFD point is observed 875 times.

XXX--For each of these 1.6 runs, maybe an include a science impact recap? Maybe a table to compare median coadded depth in each filter, a radar relative to baseline (and same scale across all of them), and a ew lines of explination of what we think happened, what metrics we need (e.g., here we could say we need AGN and other DDF relevant metrics).

%#################### Barebones #############################
\subsection{Barebones}

\begin{figure}
\epsscale{0.5}
\plotone{plots/pulled_plots/barebones_v1_6_10yrs_Count_observationStartMJD_HEAL_SkyMap}
\plotone{plots/pulled_plots/barebones_v1_6_10yrs_Nvisits_as_function_of_Alt_Az_HEAL_SkyMap}
\plotone{plots/pulled_plots/barebones_v1_6_10yrs_Hourglass_year_0-1_HOUR_Hourglass}
\epsscale{1}
\caption{The barebones simulation essentially covering just the WFD area as efficiently and deeply as possible.}\label{fig:barebones}
\end{figure}


The barebones simulation is an example of a survey where we focus exclusively on meeting the SRD requirements, with little optimization for science.

The survey footprint is restricted to the standard 18,000 square degree WFD area only. Deep drilling fields are included, but capped at $\sim2.5$\% of the total visits. Visits in u and y are unpaired, while the rest of the filters are paired in the same filter. This results in very few filter changes in a night. 

There are a wide number of reasons why this would be a terrible survey strategy--detected transients would have no color information, photometric uber-calibration would be difficult with the galactic plane gap, a lack of solar system object because the NES is not included, etc.  The main purpose is to show the scheduler can reach very near the theoretical maximum for open shutter fraction, with this run reaching 80\%. Also, we can note the fONv metric reaches 1,148 which is 40\% higher than the SRD requirement of 825. This also implies that we can observe a maximum of $\sim115$\ WFD visits per year in the event we want to adjust the scheduler to attempt to catch up on the WFD progress. 



%#################### Data Management Heavy #############################
\subsection{Data Management Heavy}

\begin{figure}
\epsscale{0.5}
\plotone{plots/pulled_plots/dm_heavy_v1_6_10yrs_Count_observationStartMJD_HEAL_SkyMap}
\plotone{plots/pulled_plots/dm_heavy_v1_6_10yrs_Nvisits_as_function_of_Alt_Az_HEAL_SkyMap}
\plotone{plots/pulled_plots/dm_heavy_v1_6_10yrs_Hourglass_year_0-1_HOUR_Hourglass}
\epsscale{1}
\caption{The DM heavy simulation. Similar to the baseline, but the alt/az plot shows how some observations are being taken at high airmass to support DCR modeling.}\label{fig:dmheavy}
\end{figure}


This is simulations includes various modifications that may be helpful for Data Management purposes. For the WFD region in u, g, and r a few images per year are taken at high airmass so that DCR correction models can be made.

The camera rotator angle is set so that diffraction spikes fall along CCD rows and columns. This helps with difference imaging so the maximum possible area can be used, but may result in weak lensing systematics.

Each year, the scheduler prioritizes taking g,r,i images of the whole sky in good seeing conditions (defined as 0.7\arcsec effective FWHM or better).

The DDF fields use larger dithers, up to 1.5 degrees, compared to the default 0.7 degree maximum.


%#################### Rolling Extragalactic #############################
\subsection{Rolling Extragalactic}

\begin{figure}
\epsscale{0.5}
\plotone{plots/pulled_plots/rolling_exgal_mod2_dust_sdf_0_80_v1_6_10yrs_Count_observationStartMJD_HEAL_SkyMap}
\plotone{plots/pulled_plots/rolling_exgal_mod2_dust_sdf_0_80_v1_6_10yrs_Nvisits_as_function_of_Alt_Az_HEAL_SkyMap}
\plotone{plots/pulled_plots/rolling_exgal_mod2_dust_sdf_0_80_v1_6_10yrs_Hourglass_year_0-1_HOUR_Hourglass}
\epsscale{1}
\caption{The Rolling Exgal simulation. }\label{fig:rollingexgal}
\end{figure}


\begin{figure}
\plottwo{plots/rolling_plot/baseline_nexp1_v1_6_Count_filter_note_not_like_DD_HEAL_SkyMap.pdf}{plots/rolling_plot/rolling_exgal_mod2_dust_sdf_0_80_v1_6_Count_filter_note_not_like_DD_HEAL_SkyMap.pdf}
\plottwo{plots/rolling_plot/baseline_nexp1_v1_6_Count_filter_night_gt_1278_375000_and_night_lt_1643_625000_and_note_not_like_DD_HEAL_SkyMap.pdf}{plots/rolling_plot/rolling_exgal_mod2_dust_sdf_0_80_v1_6_Count_filter_night_gt_1278_375000_and_night_lt_1643_625000_and_note_not_like_DD_HEAL_SkyMap.pdf}
\caption{Illustration of how rolling cadence works. The top panels show the number of observations after 10 years (all filters) for the baseline and rolling exgal simulations (excluding DDF observations). Both simulations have very smooth WFD coverage, with $\sim$900 observations.  The lower panels show the number of observations taken between 3.5 and 4.5 years into the survey.  The baseline WFD remains smooth, while the rolling exgal simulation has declination stripes of high and low counts.  }\label{fig:exgalroll}
\end{figure}


The rolling extragalactic is motivated by cosmological drivers. The footprint is modified so the 18,000 square degrees of the WFD are placed in low-extinction regions. The simulation also executes a half-sky rolling scheme, which should result in well sampled lightcurves for extragalactic transients.

xxx--add a plot showing the lines of how rolling works.  

xxx--mention that doing rolling in quarters ensures that half the alert stream is always available to northern telescopes for follow up. 


%#################### Milky Way Heavy #############################
\subsection{Milky Way Heavy}
\begin{figure}
\epsscale{0.5}
\plotone{plots/pulled_plots/mw_heavy_v1_6_10yrs_Count_observationStartMJD_HEAL_SkyMap}
\plotone{plots/pulled_plots/mw_heavy_v1_6_10yrs_Nvisits_as_function_of_Alt_Az_HEAL_SkyMap}
\plotone{plots/pulled_plots/mw_heavy_v1_6_10yrs_Hourglass_year_0-1_HOUR_Hourglass}
\epsscale{1}
\caption{The Milky Way heavy simulation. }\label{fig:mwheavy}
\end{figure}

The Milky Way heavy simulation covers the Galactic bulge, LMC, and SMC as part of the WFD area.  


%#################### Solar System Heavy #############################
\subsection{Solar System Heavy}
\begin{figure}
\epsscale{0.5}
\plotone{plots/pulled_plots/ss_heavy_v1_6_10yrs_Count_observationStartMJD_HEAL_SkyMap}
\plotone{plots/pulled_plots/ss_heavy_v1_6_10yrs_Nvisits_as_function_of_Alt_Az_HEAL_SkyMap}
\plotone{plots/pulled_plots/ss_heavy_v1_6_10yrs_Hourglass_year_0-1_HOUR_Hourglass}
\epsscale{1}
\caption{The Solar System heavy simulation. }\label{fig:ssheavy}
\end{figure}

Include ecliptic coverage through the galactic plane

NEO survey at twilight--Note, a NEO survey taking short exposures will drastically increase the data throughput. DM needs to check if this would be feasible.  We also need to check with the camera team that taking short exposures for an extended time will not be a thermal issue.

only i,z,y in twilight, making sure we observe more r-band in non-twilight and in pairs.

include r+r pairs

For regular (30s visit) twilight observations, avoid the ecliptic (ensuring they are always taken in pairs in non-twilight time)

%#################### Combo Dust #############################
\subsection{Combo Dust}

\begin{figure}
\epsscale{0.5}
\plotone{plots/pulled_plots/combo_dust_v1_6_10yrs_Count_observationStartMJD_HEAL_SkyMap}
\plotone{plots/pulled_plots/combo_dust_v1_6_10yrs_Nvisits_as_function_of_Alt_Az_HEAL_SkyMap}
\plotone{plots/pulled_plots/combo_dust_v1_6_10yrs_Hourglass_year_0-1_HOUR_Hourglass}
\epsscale{1}
\caption{The Combo Dust simulation. }\label{fig:combodust}
\end{figure}


This simulation attempts to improve several science cases compared to the baseline simultaneously. The footprint used here starts with defining the WFD area as 18,000 square degrees with low extinction. Then an additional 2,000 square degrees are added to WFD to cover the bulge, the ecliptic through the galactic plane, the LMC and SMC, and an outer disk region. Dusty areas of the sky and the South Celestial Pole are covered at about one-quarter the WFD depth. The NES is covered in $g$, $r$, $i$, and $z$. The footprint also includes very light coverage to the northern limit of the telescope so there can be templates for ToO events on the entire visible sky. 

The footprint has 35 free parameters for setting region locations and filter ratios. Many of these have have been set by eye or use historical values of questionable providence. Moving forward with such a footprint 

XXX--this one is also rolling.

% Let's make a table comparing everything. Maybe one for the basics, then one for science?
