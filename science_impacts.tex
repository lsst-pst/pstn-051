\section{Broad Survey Strategy Choices}

The families of simulations described in the previous sections (FBS 1.5 and 1.6 releases) maintained the approach of varying a single kind of parameter (such as the amount of time devoted to triplets of visits in the {\tt third\_visit} runs), but the underlying survey strategy choices can cover multiple families (such as the runs including pairs in mixed or the same filters and the triplet visits runs). In this section, we will consolidate and summarize some of the results from the previous sections, identifying tensions and trades as possible. 

\subsection{Visit Exposure Time}\label{sec:visitexposuretime}

There are several choices related to the individual visit exposure time:
 \begin{itemize}
 \item Should visits be made of a single exposure (1x30s) or two (2x15s)? 
 \item Should $u$ band visits be longer exposures (60s)?
 \item Should the visit exposure time be variable?
 \end{itemize}
 
For the first of these questions -- 1x30s visits or 2x15s visits -- is perhaps the most important, yet the final decision cannot be made until the camera is on the sky and acquiring data, to verify cosmic ray rejection will work and that the two snaps are not required. In adidition, there are still some rapid transient or variable science cases that may still benefit from 2x15s visits. However, while the decision cannot be finalized yet, the decision does weigh in on other the survey strategy choices, so it must be considered already. The decrease in overhead going from 2x15s visits to 1x30s visits allows 8\% more visits to be obtained in the same time, which allows significantly more options for footprint coverage and addition of minisurveys. 

The second question -- increasing the $u$ band exposure time -- was addressed in Section~\ref{ss:u60}. The tension comes between increasing coadded depth in $u$ versus a dramatically decreased number of $u$ band visits making it more difficult to obtain $u$ photometry for transients. Extending the visit exposure time while also maintaining the number of $u$ band visits would require adding about 3\% more visits to WFD by removing them from other minisurveys. 

The third question -- variable exposure time -- was addressed in Section~\ref{ss:var_expt}. This decreases the scatter in the individual image five sigma limiting magnitudes, but results in fewer overall visits. It had mixed effects on the typical limiting magnitudes per visit, and the coadded $u$ band depth became shallower by 0.3 magnitudes. In general, this seems to add complication to practical aspects of the scheduler when using real-world telemetry (due to the difficulty in predicting the proper exposure time when scheduling blocks of blobs) and may not have clear advantages to science. 

\subsection{Intra-night Cadence}\label{sec:intranight}

Within each night, there are some choices about how to acquire visits during the non-twilight, good weather time:
\begin{itemize}
\item Should pairs of visits be taken in the same filter or different filters?
\item Should additional visits (triplets) be acquired for some portion of the fields during the night?
\end{itemize}

Whether or not pairs should be taken in the same filter or different filters pre-supposes that visits should be taken in pairs at all. With the current baseline software and criteria for linking moving objects, the need for pairs of visits is clear. Even without the requirement of linking moving objects, there will be many unknown solar system objects in every image; taking pairs of visits is a reasonable way to quickly filter these detections out of the alert stream. 

Then the question is whether pairs should be taken in the same filter or in different filters. The simulations in Section~\ref{ss:baseline} show that same-filter pairs achieve about 4\% more visits over the survey than split-pairs, due to fewer filter changes. Same-filter pairs have a slight benefit to discovery for all solar system objects (2-3\%), but same-filter pairs have a big negative impact on transients (10-50\% depending on the population and metric requirements). Figure~\ref{fig:intranight} illustrates the response of these metrics across these simulations.

The final question is if some fraction of visits should be acquired in triplets. Section~\ref{ss:thirdobs} explores adding a third observation per night for some fraction of the available time. By adding a third visit, less sky is covered per night. This has a small negative impact on most solar system object discovery, with NEO completeness at $H\le22$ dropping by about 4\% for the simulation with 120 minutes per night spent on triplets. Surprisingly, the transient metrics did not show much improvement with the addition of triplets; detection of TDEs dropped by 8\% with 120 minutes of triplets. This is likely due to the timescales over which the objects change, the smaller amount of area covered per night, and the specific detection requirements in the metrics.  

It seems likely that we are also missing metrics; metrics which reflect classification confusion and detection requirements for transients which were the target of the Bianco et al white paper (`Presto-Color') may change this evaluation. 

\begin{figure}
\epsscale{0.75}
\plotone{plots/srd_intranight}
\plotone{plots/sso_intranight}
\plotone{plots/desc_intranight}
\plotone{plots/tvs_intranight}
\caption{SRD metrics, Solar System discovery metrics, DESC metrics (which are calculated over an extragalactic footprint), and a subset of metrics relating to transients and variables. The runs included here are those which together relate to the intra-night cadence.}
\label{fig:intranight}
\end{figure}


\subsection{Survey Footprint}\label{sec:bigfootprints}

What to do for WFD footprint? SRD not specific, DESC want low-extinction sky (and depth), but WFD is generally the area of sky that receives the most visits, so generally other science will also benefit from more visits to their relevant areas (particularly galactic plane .. for time-domain studies primarily, not depth)

We see that science effect depends strongly on how the other minisurveys are configured, and how they are configured will vary depending on what the WFD footprint looks like -- these aren't entirely independent questions. 

Top science metrics across footprint (including all kinds of coverage). 
What metrics are obviously missing?

\begin{itemize}
    \item{How should we cover the Galactic plane?}
    \item{How should we observe the Galactic bulge?}
    \item{How should we cover the north?}
    \item{How should we cover the SCP? (is this more specifically the LMC/SMC?)}
    \item{Should we avoid areas of high dust extinction for the WFD area?}
    \item{What is the ideal filter distribution to use? It would be nice to have a photo-z metric to help make this decision.}
    \item{What is the ideal filter distribution in the GP and SCP?}
    \item{Should we cover the LMC and SMC as part of the WFD survey? As their own DDF-like survey? We have few metrics that touch on LMC/SMC science directly.}
    \item{Should we add area in the north to overlap with Euclid, WFIRST, and/or DESI?}
\end{itemize}


\subsection{Rolling cadence}

Top science metrics across rolling cadence options. 
What metrics are obviously missing?


\subsection{Other minisurveys}

twilight survey, DCR, short\_exp minisurveys, (toO?)-- very different than main survey and represent additional time

Effect of adding or removing these minisurveys - number of long visits in WFD



Top metrics across these runs? Is it consistent (i.e. just like removing visits?)


\subsection{Deep Drilling Fields}


Discuss potential cadences (AGN/ DESC) and how these differ, and that combining the two is more expensive

Effect of adding or removing these minisurveys

We have run a variety of Deep Drilling strategies. The DDF strategy is largely separable from the rest of the survey design, and we have a number of proposals for DDFs that we have yet to explore (e.g., rolling DDFs where a single DDF is completed in one observing season).  We have started experimenting with pre-scheduling DDF observations. 

\begin{itemize}
    \item{What fraction of the survey should be dedicated to the DDFs?}
    \item{Should DDFs be preferentially executed in dark time, or is it more important to maintain cadence?}
    \item{Where should the DDFs be placed (can we finalize the 5th DDF as a Euclid double-pointing)?}
    \item{What is the preferred dithering strategy (spatially and rotationally) for the DDFs? There is tension in that DM generally prefers larger dithers for calibration and co-addition purposes, while science cases prefer smaller dithers to preserve the area that reaches the deepest levels.}
    \item{Should we try ``rolling" the DDFs, completing DDF observations in a field in only a few years?}
\end{itemize}

Relevant metrics: number of visits and coadded depth for DD, SN detection in DDFs, AGN detection in DDFs
*[solar system minisurvey DDF?]
Missing DDF metrics

\subsection{Dithering}

\subsubsection{Rotational dithering}

By default, we select a random camera rotation angle (wrt the telescope) nightly. This creates minimal additional slewtime, and seems to provide adequate angular randomization.  We currently have no science metrics that depend on the angular distribution, and this should be something very important to weak lensing science (although we do not have a metric to measure this).

We have also experimented with setting the camera rotation angle to ensure stellar diffraction spikes fall preferentially along rows and columns. 

\begin{itemize}
    \item{How should we rotationally dither visits?}
\end{itemize}

Need metrics sensitive to the rotation angle of the camera. 

\subsubsection{Spatial Dithering}

For the wide area regions we have had excellent results randomizing the tessellation orientation nightly. This does result in a small percent of time being spent observing outside the desired survey footprint. The alternative would be to limit the amount one dithers out of the footprint, but then one risks imprinting systematics on objects near the footprint border (e.g., an object is never observed in the center of the focal plane, only by outer rafts).

Primary metric is survey uniformity?


\subsection{Image Differencing Templates, DCR}

Do we need to do anything special to ensure we have adequate image templates? A certain number of observations per year? A certain fraction of images taken in good seeing conditions? 

If we need to start considering image quality, that makes it more difficult to simulate a night ahead of time and maintain the list of upcoming observations.

Should we intentionally extend to high airmass to facilitate DCR modeling for templates? Note that in the baseline, we only image a location in the WFD region $\sim$9 times per year in $g$ and $\sim$6 times in $u$. Also, we have chip and raft gaps, so if we want to build a DCR model for the entire sky in $g$, we might be dedicating 1/3 of the $g$ observations in a year to DCR. If we switch to 60s $u$ band exposures, there would be no observations beyond building the DCR model. (We don't know that DM needs this).

There have been claims that measuring DCR can be used for science.  We do not have any metrics that demonstrate any gains, and the loss of depth is noticeable. In theory, we could combine the DCR measurements to extend the season length of observations as well (e.g., only take DCR template images near twilight in the direction of the sun).


\subsection{Satellite Megaconstellations}

Starlink (followed by Amazon) is poised to launch thousands of LEO satellites. Observations so far imply that final-orbit Starlink satellites should not saturate Rubin exposures, and thus can be masked fairly easily in the image reduction pipeline. 

Do we need any further satellite mitigations? Will NEO twilight surveys still be viable in the presence of megaconstellations, or should we use twilight strategies that avoid the horizon?

Figure~\ref{fig:megasat} shows how illuminated megaconstellations in LEO would leave numerous streaks on Rubin images.

% from https://github.com/yoachim/satellite_collisions
\begin{figure}
\label{fig:megasat}
\plottwo{plots/sat_plots/ten_min_12k.pdf}{plots/sat_plots/tenmin_example.pdf}
\caption{Alt/az projection of simulated satellite megaconstellations as seen from the Rubin Observatory site after twilight has ended. } 
\end{figure}



\section{Conclusions}

Have laid out several questions and lots of families of simulations. Obvious dealbreakers for core science? 

Metrics we know we need to get from the community:
\begin{itemize}
    \item{Photometric redshift performance, especially as it relates to filter distribution}
    \item{Weak Lensing systematics, especially as related to camera rotator angle}
    \item{Deep Drilling Field metrics beyond coadded depth (e.g., AGN performance)}
    \item{Deep Drilling metrics that are sensitive to the spatial dither strategy}
    \item{Transient early classification metric}
    \item{More populations in the Galactic plane beyond the simple number of stars.}
\end{itemize}

