
Simulations completed starting in 2020 use a revised database for the
atmospheric seeing. The revised database, like its predecessor, is
based on seeing measurements from the Geminin South DIMM, located at
the same site as Rubin Observatory. We derived predicted delivered
imager FWHM from the reported DIMM measurements using the
approximation of the von K\'arm/'an turbulance model given in
Tokovinin (2002) and an outer scale of 30 meters, and validated this
relationship between DIMM measurements and seeing by comparing these
derived values to the image quality measured from the Gemini South
GMOS instrument. We also tested the DIMM data by deriving a seeing
using the Kolmogerov relationship and comparing the result to the
seeing measured by the DECam imager on the Blanco telescope at CTIO, a
few miles away.

For most time samples in the simulation database, we generated seeing
data by resampling seeing derived from the DIMM into 5 minute
intervals, and shifting it ahead 4748 days (13.000 tropical years). For
example, the seeing for 2022-01-01 in the simulation database is taken
from the DIMM seeing on 2009-01-01. Thus, most of the ten simulated
years use seeing values that replay ten historical years.

There is, however, significant time for which no DIMM data is available, for
example due to clouds or equipment failure. We used a model of
$log(r_{0})$ (where $r_{0}$ is the Fried parameter) derived
from the DIMM data to generate artifical seeing values for these
times. This model has several components:
\begin{itemize}
  \item a yearly sinusoidal variation in $log(r_{0})$ to include
    seasonal variation,
  \item a smooth (years timescale) fit to the residuals with respect
    to the seasonal variation to represent multi-year trends in
    seeing,
  \item a 1st-order autoregressive series (damped random walk) to
    represent variations in the nightly seeing, and
  \item another 1st-order AR series to represent variations on a
    5-minute timescale within a night.
\end{itemize}
Artificial data generated according to this model therefore maintains
the night to night and short term distributions and correlations
present in the DIMM data, and follows seasonal variations and longer
term trends in the DIMM data surrounding it. 
